\documentclass[10pt]{beamer}


\usetheme[progressbar=frametitle]{metropolis}
\usepackage{appendixnumberbeamer}

\usepackage{booktabs}
\usepackage[scale=2]{ccicons}

\usepackage{pgfplots}
\usepgfplotslibrary{dateplot}
\usepackage[utf8]{inputenc}

\usepackage[table]{xcolor}
\usepackage{colortbl}


\usepackage{xspace}
\newcommand{\themename}{\textbf{\textsc{metropolis}}\xspace}

%%%%%%%%%%%%%%%%%%%%%%%%%%%%
%% UNCC Theme Adjustments %%
%%%%%%%%%%%%%%%%%%%%%%%%%%%%
\definecolor{CanvasBG}{HTML}{FAFAFA}

% From the official style guide
\definecolor{UnccGreen}{HTML}{005035}
\definecolor{UnccLightGreen}{HTML}{C3D7A4}
\definecolor{UnccGold}{HTML}{c1d7fe}
\definecolor{UnccOrange}{HTML}{F3901D}
\definecolor{UnccLightYellow}{HTML}{899064}
\definecolor{UnccBlue}{HTML}{007377}
\definecolor{UnccPink}{HTML}{DE3A6E}
\definecolor{White}{HTML}{FFFFFF}
\definecolor{LightGray}{HTML}{F1E6B2}
\definecolor{Purple}{HTML}{6c5098}
\definecolor{Bordeau}{HTML}{570514}
\definecolor{ULB_blue}{HTML}{003087}

\setbeamercolor{frametitle}{bg=ULB_blue}
\setbeamercolor{progress bar}{bg=UnccGold, fg=ULB_blue}
\setbeamercolor{alerted text}{fg=UnccOrange}

\setbeamercolor{block title}{bg=UnccGreen, fg=White}
\setbeamercolor{block title example}{bg=UnccBlue, fg=White}
\setbeamercolor{block title alerted}{bg=UnccPink, fg=White}
\setbeamercolor{block body}{bg=LightGray}

\metroset{titleformat=smallcaps, progressbar=foot}

\makeatletter
\setlength{\metropolis@progressinheadfoot@linewidth}{2pt}
\setlength{\metropolis@titleseparator@linewidth}{2pt}
\setlength{\metropolis@progressonsectionpage@linewidth}{2pt}
%%%%%%%%%%%%%%%%%%%%%%%%%%%%
%% UNCC Theme Adjustments %%
%%%%%%%%%%%%%%%%%%%%%%%%%%%%


\title{STAGE DE RECHERCHE STAG-F015}
\subtitle{Étude de spectres infrarouges de géantes rouges évoluées}
% \date{\today}
\date{}
\author{\small Margaux Vandererven}
\institute{\small Supervisé par Sophie Van Eck}
 \titlegraphic{\hfill\includegraphics[height=1.5cm]{/Users/margauxvandererven/Documents/unif2023-2024/spectre_IR/rapport/figures/science.png}}

\begin{document}

\maketitle

%\metroset{titleformat frame=allcaps}
\begin{frame}[fragile]{Étoiles de type S \& étoiles à baryum}




\begin{columns}
        \begin{column}{0.5\textwidth}
            \begin{itemize}
                \item 17 étoiles : 
					\begin{itemize}
						\item[\circ] de type S intrinsèques
						\item[\circ] de type S extrinsèques
						\item[\circ] à baryum
						\item[] 
					\end{itemize} 
				\item Spectres infrarouges : IGRINS 
					\begin{itemize}
						\item[\circ] Bande H (1.45 - 1.80 $\mu m$)
						\item[\circ] Bande K (2.05 - 2.50 $\mu m$)
						\item[]
					\end{itemize}
            \end{itemize} 

		$\rightarrow$ BD-2217$^{\circ}$42 (4000K)
        \end{column}
        \begin{column}{0.5\textwidth}
            \centering
            \includegraphics[width=\textwidth]{/Users/margauxvandererven/Documents/unif2023-2024/spectre_IR/rapport/figures/agb_structure.jpeg}
			\caption{\textit{Structure interne d'une étoile AGB}}
        \end{column}
    \end{columns}





%  The \themename theme is a Beamer theme with minimal visual noise
%  inspired by the \href{https://github.com/hsrmbeamertheme/hsrmbeamertheme}{\textsc{hsrm} Beamer
%  Theme} by Benjamin Weiss.
%
%  Enable the theme by loading
%
%  \begin{verbatim}    \documentclass{beamer}
%    \usetheme{metropolis}\end{verbatim}
%
%  Note, that you have to have Mozilla's \emph{Fira Sans} font and XeTeX
%  installed to enjoy this wonderful typography.
\end{frame}


%\metroset{titleformat frame=allcaps}
\begin{frame}[fragile]{Contributions moléculaires}
  
\begin{table}
%    \caption{}
    \begin{tabular}{c|ccc}
      \toprule
	\midrule
       &Molécules & Bande H (\%) & Bande K (\%)\\
      \midrule
      \cellcolor{green!15}\textbf{Cat. I}&\cellcolor{green!15}$^{12}$C$^{14}$N &\cellcolor{green!15} 82.47 & \cellcolor{green!15}76.33  \\
      \small\cellcolor{green!15}($>$ 10\%)&\cellcolor{green!15}$^{13}$C$^{14}$N & \cellcolor{green!15}78.52 & \cellcolor{green!15}67.18  \\
      \cellcolor{green!15}&\cellcolor{green!15}$^{12}$C$^{16}$O &\cellcolor{green!15}71.92 &\cellcolor{green!15}71.01   \\
      \cellcolor{green!15}&\cellcolor{green!15}HF & \cellcolor{green!15}1.81 &\cellcolor{green!15}47.39   \\
	  \cellcolor{green!15}&\cellcolor{green!15}$^{12}$C$^{12}$C &\cellcolor{green!15} 81.40 & \cellcolor{green!15}77.39  \\
	\cellcolor{green!15}&\cellcolor{green!15}$^{12}$C$^{13}$C & \cellcolor{green!15}73.81  & \cellcolor{green!15}65.34   \\
	\cellcolor{yellow!15}\textbf{Cat. II}&\cellcolor{yellow!15}$^{13}$C$^{13}$C &\cellcolor{yellow!15} 7.84  & \cellcolor{yellow!15}3.51  \\
	\small\cellcolor{yellow!15}(1-10\%)&\cellcolor{yellow!15}$^{16}$OH &\cellcolor{yellow!15}2.20  &\cellcolor{yellow!15}0.56    \\
	\cellcolor{yellow!15}&\cellcolor{yellow!15}$^{56}$FeH &\cellcolor{yellow!15} 2.96  &\cellcolor{yellow!15}0.08   \\
	\cellcolor{yellow!15}&\cellcolor{yellow!15}$^{12}$CH & \cellcolor{yellow!15}5.97  & \cellcolor{yellow!15}8.55   \\
      \bottomrule
    \end{tabular}
  \end{table}
\textbf{Cat. III} ($<$ 1\%) : $^{13}$C$^{17}$O, $^{13}$CH, $^{14}$NH, $^{48}$TiO, C$_{2}$H$_2$, HCl, H$_{2}$O, $^{20}$CaH, $^{28}$SiH, $^{28}$SiO, VO, YO, $^{48}$TiO, $^{24}$MgH, AlH, $^{52}$CrH, H$^{12}$CN, H$^{13}$CN, $^{90-94}$ZrO et $^{96}$ZrO
\end{frame}

%\metroset{titleformat frame=allcaps}
\begin{frame}[fragile]{Abondances C, N, O \& macroturbulence}

\textbf{Abondances} : 
	\begin{table}
    		\begin{tabular}{c|cccc}
      		\toprule
			\midrule
       		&$[Fe/H]$ & $[C/Fe]$ & $[N/Fe]$&$ [O/Fe]$\\
      		\midrule
			\textit{Ce travail} & - & 0.41 &0.32 & 0.75  \\
      		\textit{Shetye et al. (2018)} &-0.30$\pm$0.09& 0.35 &-0.1 &-  \\
			      		\bottomrule
    		\end{tabular}
	  \end{table}


\textbf{Macroturbulence} : $\rightarrow$ entre 12 et 13 km s$^{-1}$ \newline

\textbf{\large Dans la suite...}
\begin{itemize}
\item Déterminer avec précision la macroturbulence pour BD-2217$^{\circ}$42
\item Déterminer avec précision l'abondance de C, N, O pour BD-2217$^{\circ}$42
\item Déterminer l'abondance d'éléments lourds pour BD-2217$^{\circ}$42
\item Recommencer avec les 16 autres étoiles !
\end{itemize}

\end{frame}


%\metroset{titleformat frame=allcaps}
%\begin{frame}[fragile]{Abondances C, N, O \& macroturbulence}
%
%\textbf{Abondances} : 
%	\begin{table}
%%    \caption{}
%    		\begin{tabular}{ccccc}
%      		\toprule
%		\midrule
%       		&$[Fe/H]$ & $[C/Fe]$ & $[N/Fe]$&$ [O/Fe]$\\
%      		\midrule
%			\textit{Ce travail} & - & 0.41 &0.32 & 0.75  \\
%      		\textit{Shetye et al. 2018} &-0.30$\pm$0.09& 0.35 &-0.1 &-  \\
%			      		\bottomrule
%    		\end{tabular}
%	  \end{table}
%
%
%\textbf{Macroturbulence} :
%
%\begin{figure}
%\includegraphics[width=0.98\linewidth]{/Users/margauxvandererven/Documents/unif2023-2024/spectre_IR/rapport/figures/MACTURB_test.png}
%\small $\rightarrow$ entre 12 et 13 km s$^{-1}$
%\end{figure}
%
%\end{frame}




%\begin{table}
%%    \caption{}
%	\begin{tabular}{ccccc}
%		\toprule
%		\midrule
%%       		[Fe/H] & [C/Fe] &[C/Fe] & [N/Fe] & [O/Fe]&&\\
%      	\midrule
%%      \cellcolor{green!15}\textbf{Cat. I}&\cellcolor{green!15}$^{12}$C$^{14}$N &\cellcolor{green!15} 82.47 & \cellcolor{green!15}76.33  \\
%      	\bottomrule
%    \end{tabular}
%\end{table}


\end{document}