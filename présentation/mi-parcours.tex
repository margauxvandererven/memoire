\documentclass[10pt]{beamer}
\usepackage[utf8]{inputenc}

\usetheme[progressbar=frametitle]{metropolis}
\usepackage{appendixnumberbeamer}

\usepackage{booktabs}
\usepackage[scale=2]{ccicons}
\usepackage{graphicx}
\usepackage{pgfplots}
\usepackage{tikz}
\usepgfplotslibrary{dateplot}

\usepackage[table]{xcolor}
\usepackage{xcolor}
\usepackage{colortbl}
\usepackage{tabularx}

\usepackage{xspace}
\newcommand{\themename}{\textbf{\textsc{metropolis}}\xspace}


\definecolor{CanvasBG}{HTML}{FAFAFA}

% From the official style guide
\definecolor{UnccGreen}{HTML}{005035}
\definecolor{UnccLightGreen}{HTML}{C3D7A4}
\definecolor{UnccGold}{HTML}{c1d7fe}
\definecolor{UnccOrange}{HTML}{F3901D}
\definecolor{UnccLightYellow}{HTML}{899064}
\definecolor{UnccBlue}{HTML}{007377}
\definecolor{UnccPink}{HTML}{DE3A6E}
\definecolor{White}{HTML}{FFFFFF}
\definecolor{LightGray}{HTML}{F1E6B2}
\definecolor{Purple}{HTML}{6c5098}
\definecolor{Bordeau}{HTML}{570514}
\definecolor{ULB_blue}{HTML}{003087}

\setbeamercolor{frametitle}{bg=ULB_blue}
\setbeamercolor{progress bar}{bg=UnccGold, fg=ULB_blue}
\setbeamercolor{alerted text}{fg=UnccOrange}

\setbeamercolor{block title}{bg=UnccGreen, fg=White}
\setbeamercolor{block title example}{bg=UnccBlue, fg=White}
\setbeamercolor{block title alerted}{bg=UnccPink, fg=White}
\setbeamercolor{block body}{bg=LightGray}

\metroset{titleformat=smallcaps}

% progressbar=foot}

\makeatletter
\setlength{\metropolis@progressinheadfoot@linewidth}{2pt}
\setlength{\metropolis@titleseparator@linewidth}{2pt}
\setlength{\metropolis@progressonsectionpage@linewidth}{2pt}




% \title{Présentation de mi-parcours mémoire}
\subtitle{Étude de spectres infrarouges de géantes rouges évoluées}
% \date{\today}
\date{}
\author{\small Margaux Vandererven}
\institute{\small Supervisé par Sophie Van Eck}
 \titlegraphic{\hfill\includegraphics[height=1.5cm]{/Users/margauxvandererven/Documents/unif2023-2024/spectre_IR/rapport/figures/science.png}}

\begin{document}

\maketitle

% \metroset{titleformat frame=allcaps}

\begin{frame}[fragile]{Étoiles de type S \& étoiles à baryum}

    \begin{columns}
            \begin{column}{0.6\textwidth}
                    T$_{\text{eff}}$ étoiles S $\sim$ T$_{\text{eff}}$ étoiles K et M 

                    Bandes ZrO \& enrichissement en éléments s

    					\begin{itemize}
    						\item de type S intrinsèques (Tc rich)
    						\item de type S extrinsèques (Tc poor)
    						\item à baryum
    						\item[] 
    					\end{itemize} 
            \end{column}
            \begin{column}{0.4\textwidth}
                \centering
                \includegraphics[width=\textwidth]{/Users/margauxvandererven/Documents/unif2023-2024/spectre_IR/rapport/figures/agb_structure.jpeg}
    			\textit{Structure interne d'une étoile AGB.} Tiré de Persson 2014.
            \end{column}
    \end{columns}
\end{frame}

% \metroset{titleformat frame=allcaps}
\begin{frame}[fragile]{Processus s}
    % \begin{columns}
    %     \begin{column}{0.3\textwidth}
    %         +50\% éléments plus lourds que le fer \\
    %     \end{column}
    %     \begin{column}{0.7\textwidth}
    %         \centering
    %         \includegraphics[width=8cm]{images/medium.png}
    %     \end{column}
    % \end{columns}

    \begin{center}
        \includegraphics[width=10cm]{images/medium.png} \\
        (Käppeler et al. 2011.)
        \vfill
        + de 50\% éléments plus lourds que le fer \\
    \end{center}
\end{frame}

% \metroset{titleformat frame=allcaps}
\begin{frame}[fragile]{Spectre observé}
% \begin{columns}
%     \begin{column}{0.3\textwidth}
             Spectre infrarouge : \\
             IGRINS (Immersion GRating INfrared Spectrometer) \\ 
             Haute résolution : R=$\frac{\lambda}{\Delta \lambda}$ $\sim$ 45000 \\
    					\begin{itemize}
    						\item Bande H (1.45 - 1.80 $\mu m$)
    						\item Bande K (2.05 - 2.50 $\mu m$)
    						\item[]
    					\end{itemize}
        % $\rightarrow$ BD-2217$^{\circ}$42 (4000K) \\
        \vfill
        Correction : 

        Réduction, correction tellurique, première normalisation par Chris Sneden. 
        
        Seconde normalisation et correction redshift par moi-même. \\
%     \end{column}
% \end{columns}
\end{frame}

\begin{frame}[fragile]{Série d'étoiles}
    % \begin{column}{0.7\textwidth}
        \begin{table}[h!]
        %     % \caption{Étoiles et paramètres d'atmosphère stellaire}
                \begin{center}
                \resizebox{\textwidth}{!}{
                \begin{tabular}{cccccc}
                    \hline
                    \hline
                     Étoile & Type spectral & T$_{\rm eff}$ (K) & log $g$ (cm $s^{-2}$) & $\xi_{\rm micro}$ (km s$^{-1}$) & [Fe/H] (dex) \\
                    \hline
                    HD 60197 & K3.5III:Ba3.5 & $3800\pm50^{(3)}$ & $2.00\pm0.50^{(3)}$ & $2.00^{(3)}$ & $-0.60\pm0.20^{(3)}$\\
                    HD 63733 & 	S3.5/3 & $3700^{(1)}$ & $1.00^{(1)}$ & - & $-0.10\pm0.13^{(1)}$ \\
                    CR Cir & S6,2  & - & - & - & - \\
                    HD 123949 & K1pBa & $4378\pm80^{(3)}$ & $1.78\pm0.53^{(3)}$ & $1.37^{(3)}$ & $-0.31\pm0.13^{(3)}$ \\
                    \cellcolor{red!15}BD-22$^{\circ}$1742 & \cellcolor{red!15}S3:*3& \cellcolor{red!15}$4000^{(1)}$ & \cellcolor{red!15}$1.00^{(1)}$ & \cellcolor{red!15}- & \cellcolor{red!15}$-0.30\pm0.09^{(1)}$ \\
                    CD-29$^{\circ}$5912 & S4,4 & $3600^{(4)}$ & $1.00^{(4)}$ & - & $-0.40\pm0.22^{(4)}$ \\
                    BD-18$^{\circ}$2608 & S & $3500^{(2)}$ & $1.00^{(2)}$ & - & $-0.31\pm0.16^{(2)}$ \\
                    HD 116869 & G8III:Ba1 & $4892\pm30^{(3)}$ & $2.59\pm0.07^{(3)}$ & $1.38\pm0.04^{(3)}$ & $-0.44\pm0.09^{(3)}$ \\
                    HD 120620 & K0III (Ba$^{(3)}$) & $4831\pm13^{(3)}$ & $3.03\pm0.30^{(3)}$ & $1.11\pm0.05^{(3)}$ & $-0.30\pm0.10^{(3)}$ \\
                    HD 121447 & K4III$^{(3)}$ (Ba$^{(3)}$) & $4000\pm50^{(3)}$ & $1.00\pm0.50^{(3)}$ & $2.00^{(3)}$ & $-0.90\pm0.13^{(3)}$\\
                    HD 100503 & G/KpBa & $4000\pm50^{(3)}$ & $2.00\pm0.50^{(3)}$ & $2.00^{(3)}$ & $-0.72\pm0.13^{(3)}$ \\
                    HD 119185 &  G8IIIpBa & - & - & - & - \\
                    HD 88562  & K1III (Ba$^{(3)}$)& $4000\pm50^{(3)}$ & $2.00\pm0.50^{(3)}$ & $2.00^{(3)}$ & $-0.53\pm0.12^{(3)}$\\
                    V812 Oph  & S5+/2.5 & $3500^{(2)}$ & $1.00^{(2)}$ & - & $-0.37\pm0.13^{(2)}$\\
                    19 Aql  & F0III-IV & - & - & - & -  \\
                    V915 Aql  & S5+/2 & $3400^{(1)}$ & $0.00^{(1)}$ & - & $-0.50\pm0.15^{(1)}$ \\
                    HD 165774 & S4,6 & - & - & - & - \\
                    \hline
                \end{tabular}}
                \end{center}
        %     % \textbf{Notes}. Le type spectral est majoritairement repris de SIMBAD ; la première lettre faisant référence au système Harvard (O,B,A, F, G, K, M), le chiffre arabe suivant à la nomenclature "early"/"late", le chiffre romain à la classe de luminosité, "Ba" pour étoile à baryum / "S" pour étoile de type S et la lettre minuscule faisant référence aux particularités du spectre (par exemple, "p" pour "particularité non spécifiée").
            
            % \textbf{Références}. $^{(1)}$\cite{shetye_s_2018}, $^{(2)}$\cite{shetye_s_2021} , $^{(3)}$\cite{karinkuzhi_when_2018}, $^{(4)}$\cite{shetye_observational_2019}.
            \textbf{Références}. $^{(1)}$Shetye et al. 2018, $^{(2)}$Shetye et al. 2021 , $^{(3)}$Karinkuzhi et al. 2018, $^{(4)}$Shetye et al. 2019
        %     %     \label{param_stellaire}
            \end{table}
    % \end{column}
\end{frame}

\begin{frame}[fragile]{Spectre synthétique}
    \begin{columns}
        \begin{column}{0.52\textwidth}
            \begin{center}
                \textbf{TurboSpectrum v20}  
                \begin{itemize}
                    \item [-] code qui résoud l'équation de transfert radiatif avec méthode Feautrier
                    \item [-] à la fois dans l'apporixmation ETL et non-ETL
                    \item [-] à la fois pour la géométrie plan-parallèle (log g $>$ 3.5) et sphérique (log g $<$ 3.5)
                    \item [-] élargissement : profil de Voigt, effet Stark linéaire, théorie ABO
                \end{itemize}
            \end{center}
        \end{column}
        \begin{column}{0.04\textwidth}
            \begin{tikzpicture}
                \draw[thick] (0,-1) -- (0,5);
            \end{tikzpicture}
        \end{column}
        \begin{column}{0.44\textwidth}
            \begin{center}
                \textbf{MARCS}
               \begin{itemize}
                \item [-] Model Atmospheres with a Radiative and Convective Scheme
                \item [-] 1D à équilibre hydrostatique
                \item [-] convection implémentée par théorie de longueur de mélange
                \item [-] turbulences implémentées par paramètres simples (micro et macro-turbulence)
               \end{itemize}
            \end{center}
        \end{column}
\end{columns}
\vfill 
$\rightarrow$ Minimisation $\chi^2$ entre spectres synthétiques et spectre observé
\end{frame}

% \metroset{titleformat frame=allcaps}
\begin{frame}[fragile]{Contributions moléculaires}
  
\begin{table}
    %    \caption{}
    \resizebox{7cm}{!}{
        \begin{tabular}{c|ccc}
            \toprule
        \midrule
            &Molécules & Bande H (\%) & Bande K (\%)\\
            \midrule
            \textbf{Cat. I}&$^{12}$C$^{14}$N & 55.14 & 44.35  \\
            \small($>$ 10\%)&$^{13}$C$^{14}$N & 32.00 & 14.51  \\
            &$^{12}$C$^{16}$O & 75.33 & 72.01   \\
            &HF & 17.79 & 57.16   \\
            &$^{12}$C$^{12}$C & 32.97 & 30.73  \\
            &$^{12}$C$^{13}$C & 14.12  & 12.26   \\
            &$^{12}$CH & 4.68  & 10.68   \\
            &$^{16}$OH & 59.68  & 31.59    \\
            \textbf{Cat. II}&$^{13}$C$^{13}$C & 7.84  & 3.51  \\
            \small(1-10\%)&$^{13}$C$^{17}$O& 0.04 & 1.96   \\
            &$^{56}$FeH & 3.12  & 0.08   \\
            &$^{14}$NH & 1.57  & 1.23   \\
            &H$_{2}$O& 1.75 & 6.80   \\ 
            \bottomrule
        \end{tabular}}
\end{table}
\textbf{Cat. III} ($<$ 1\%) : $^{13}$CH, $^{14}$NH, $^{48}$TiO, C$_{2}$H$_2$, HCl, $^{20}$CaH, $^{28}$SiH, $^{28}$SiO, VO, YO, $^{48}$TiO, $^{24}$MgH, AlH, $^{52}$CrH, H$^{12}$CN, H$^{13}$CN, $^{90-94}$ZrO et $^{96}$ZrO
\end{frame}


% \metroset{titleformat frame=allcaps}
\begin{frame}[fragile]{Abondances C, N, O}

Itération sur les abondances de C, N, O jusqu'à convergence

\begin{table}[h!]
      \vspace{0.3cm}
    \begin{center}
    	\begin{tabular}{ccccc}
            \hline
    		\hline
            log $\varepsilon_{\rm O}$ & log $\varepsilon_{\rm C}$ & log $\varepsilon_{\rm N}$ & $^{12}$C/$^{13}$C & Raie\\
            \hline
            \cellcolor{blue!15}{8.59 $\pm$ 0.01} & 8.44 & 7.38 & 40 & $^{16}$OH \\
        % % raie de bande H
        8.59 & \cellcolor{blue!15}{7.82 $\pm$ 0.03} & 7.38 & 40 & $^{12}$C$^{16}$O \\
        8.59 & 7.82 & 7.38 & \cellcolor{blue!15}{20} & $^{13}$C$^{17}$O \\
        \cellcolor{blue!15}{8.30 $\pm$ 0.02} & 7.82 & 7.38 & 20 & $^{16}$OH \\
        8.30 & \cellcolor{blue!15}{7.89 $\pm$ 0.02} & 7.38 & 20 & $^{12}$C$^{16}$O \\
        \cellcolor{blue!15}{8.33 $\pm$ 0.02} & 7.89 & 7.38 & 20 & $^{16}$OH \\
        8.33 & \cellcolor{blue!15}{7.86 $\pm$ 0.03} & 7.38 & 20 & $^{12}$C$^{16}$O \\
        8.33 & 7.86 & \cellcolor{blue!15}{7.84 $\pm$ 0.02} & 20 & $^{12}$C$^{14}$N \\
        8.33 & 7.86 & 7.84 & \cellcolor{blue!15}{12} & $^{13}$C$^{14}$N \\
        \cellcolor{blue!15}{8.31 $\pm$ 0.01} & 7.86 & 7.84 & 12 & $^{16}$OH \\
        8.31 & \cellcolor{blue!15}{7.88 $\pm$ 0.03} & 7.84 & 12 & $^{12}$C$^{16}$O \\
        8.31 & 7.88 & \cellcolor{blue!15}{7.84 $\pm$ 0.02} & 12 & $^{12}$C$^{14}$N \\
        \end{tabular}
    \end{center} 
    % \textbf{Notes.} 
    % Chaque synthèse est réalisée sur des raies de l'élement se trouvant en 4ème colonne. 
    % Les paramètres fixés sont en noir et le paramètre déterminé en bleu. 
    % \label{itération_CNO_ancien}
    \end{table}

\end{frame}

\begin{frame}[fragile]{Paramètres stellaires}
    % $[Fe/H]$ : abondance de fer 


    % T$_{\rm eff}$ : respect de l'équation de Boltzmann $\rightarrow$  abondance d'un élément ne varie pas en fonction de son potentiel d'excitation 


    % log g : respect de l'équation de Saha  $\rightarrow$ abondance identique pour l'élément neutre et ses différents états d'ionisation \\ isochrone et tracés évolutifs


    % $\xi_{\text{micro}}$ : abondance ne varie pas en fonction de la largeur équivalente réduite 

    % \begin{column}{0.7\textwidth}
\begin{table}[h!]
    \begin{center}
        \begin{tabularx}{\textwidth}{c|X|c|c}
            Paramètre & Détermination & Ce travail & Littérature \\
            \hline
            &&\\
            $[Fe/H]$ &  abondance de fer & -0.25 dex & -0.30 dex \\
            &&\\
            T$_{\rm eff}$ & respect de l'équation de Boltzmann $\rightarrow$  abondance d'un élément ne varie pas en fonction du potentiel d'excitation & 4000 $\pm$ ? K  & 4000 K\\
            &&\\
            log g & respect de l'équation de Saha  $\rightarrow$ abondance identique pour l'élément neutre et ses différents états d'ionisation & to do Ti lines & 1.00 km s$^{-1}$\\
        \end{tabularx}
    \end{center}
\end{table}

\end{frame}

\begin{frame}[fragile]{Paramètres stellaires suite}
    % $[Fe/H]$ : abondance de fer 


    % T$_{\rm eff}$ : respect de l'équation de Boltzmann $\rightarrow$  abondance d'un élément ne varie pas en fonction de son potentiel d'excitation 


    % log g : respect de l'équation de Saha  $\rightarrow$ abondance identique pour l'élément neutre et ses différents états d'ionisation \\ isochrone et tracés évolutifs


    % $\xi_{\text{micro}}$ : abondance ne varie pas en fonction de la largeur équivalente réduite 

    % \begin{column}{0.7\textwidth}
\begin{table}[h!]
    \begin{center}
        \begin{tabularx}{\textwidth}{c|X|c|c}
            Paramètre & Détermination & Ce travail& Littérature \\
            \hline
            &&\\
            log g & isochrones (estimer âge) & 1.04 & 1.00 \\
            &&\\
            & tracés évolutifs (estimer masse) & to do & 1.00 \\
            &&\\
            & ailes raies fortes & 0.31 & 1.00 \\
             &&\\
            $\xi_{\text{micro}}$ & abondance ne varie pas en fonction de la largeur équivalente réduite & to do & - \\
        \end{tabularx}
    \end{center}
\end{table}

\end{frame}

\begin{frame}[fragile]{Raies atomiques}
    % Ca I, Mg I, Al I, Si I, K I, Sc I, Ti I, Ti II, V I, Mn I, Fe I, Co I, Ni I, Cu I, Y I, Zr I, Ba I, Ce II, Ce III, Er II, Yb II

\begin{table}[h!]
    \begin{center}
        \begin{tabular}{c|c|c}
            & Élement & Nb. raies\\
            \hline
            &&\\
            Élements pic du fer & Sc I &115\\
             & Ti I&63\\
             & Ti II&7\\
             & V I&76\\
             & Cr I&20\\
             & Mn I&55\\
             &Fe I&81 \\
            & Co I&69\\
            & Ni I&58\\
            Élements-$\alpha$ & Mg I&12 \\
             & Si I&13 \\
             & Ca I& 5\\
        \end{tabular}
    \end{center}
\end{table}

 % Élements du pic du fer  : Mn I, Co I, Ni I, 

\end{frame}

\begin{frame}[fragile]{Raies atomiques suite}
    % Ca I, Mg I, Al I, Si I, K I, Sc I, Ti I, Ti II, V I, Mn I, Fe I, Co I, Ni I, Cu I, Y I, Zr I, Ba I, Ce II, Ce III, Er II, Yb II

    \begin{table}[h!]
        \begin{center}
            \begin{tabular}{c|c|c}
                &Élement&Nb. raies\\
                \hline
                &&\\ 
                Éléments Z impaire & Na I &19\\
                 & Al I &7\\
                 & K I &5 \\
                Éléments s & Cu I & 5\\
                 & Y I & 17\\
                 & Zr I & 2\\
                 & Ba I & 2\\
                 & Ce II & 9\\
                 & Ce III & 2\\
                 & Nd II & 7\\
                 & Yb II & 2\\
            \end{tabular}
        \end{center}
    \end{table}
 % Élements du pic du fer  : Mn I, Co I, Ni I, 

\end{frame}

\begin{frame}[fragile]{La suite ?}
    \begin{itemize}
        \item finir détermination paramètres stellaires
        \item comparaison avec abondances déterminées dans le visible à partir de spectre HERMES
        \item détermination  d'abondances d'éléments lourds ETL et non-ETL
        \begin{itemize}
            \item [-] besoin de listes non-ETL
            \item [-] besoin d'interpoler modèles et coefficients d'écarts non-ETL
        \end{itemize}
        \item comparaison profil d'abondances avec modèles de nucléosynthèse
        \item analyse d'une autre étoile ?
    \end{itemize} 
\end{frame}
% \begin{frame}[fragile]{Références}
%     % \bibliographystyle{plain}
%     % \bibliography{bibtex.bib}
% \end{frame}

\end{document}