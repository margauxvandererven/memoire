\documentclass{article}
\usepackage[utf8]{inputenc} % Encodage pour les caractères spéciaux
\usepackage[T1]{fontenc}  

\usepackage{graphicx}
\usepackage[margin=1.8cm]{geometry}
\usepackage{fancyhdr}
\usepackage{pgf}
\usepackage{lmodern}
\usepackage{pgfplots}
\usepackage{tikz}
\usepackage{todonotes}
\usepackage{setspace} 
\usepackage{titlesec}
\usepackage{gensymb}
\usepackage{amsmath,amsfonts,amssymb}
\usepackage[T1]{fontenc}
\usepackage{natbib} 
\usepackage[french]{babel}
\usepackage{hyperref}
\usepackage{lipsum}  
\usepackage{subcaption}
\usepackage{array}
\usepackage[table, dvipsnames]{xcolor}
\usepackage{textcomp}

\definecolor{darkblue}{rgb}{0.0, 0.0, 0.8}
\pgfplotsset{compat=1.9}
\usepgfplotslibrary{groupplots}
\usepgfplotslibrary{polar}
\usepgfplotslibrary{smithchart}
\usepgfplotslibrary{statistics}
\usepgfplotslibrary{dateplot}
\usepgfplotslibrary{ternary}
\setlength{\headheight}{12.80502pt}


\setcitestyle{authoryear,open={},close={}, color = 'blue'}

\hypersetup{colorlinks,linkcolor={black},citecolor={black},urlcolor={black}}  

\DeclareCaptionFormat{custom}
{%
    \textcs{#1{.}} \small{#3}
}
\captionsetup{format=custom}

\renewcommand{\contentsname}{Contenu}

\titleformat{\section}
  {\Large\bf\sc}{\thesection}{1em}{}
\titleformat{\subsection}
  {\large\sc}{\thesubsection}{1em}{}
\titleformat{\subsubsection}
  {\sc}{\thesubsubsection}{1em}{}

%\setlength{\arrayrulewidth}{1mm}
%\setlength{\tabcolsep}{9pt}
\renewcommand{\arraystretch}{1.5}
%\newcolumntype{s}{>{\columncolor[HTML]{AAACED}} p{3cm}}

\renewcommand{\abstractname}{\sc \large Résumé}

\fancypagestyle{main}{
  \fancyhf{}
  \fancyfoot[L, RO]{\thepage}
%  \fancyhead[LO]{\firstleftmark}
  \fancyhead[C]{\sc Étude de spectres infrarouges de géantes rouges évoluées}
}




\pagestyle{main}

\begin{document}


\begin{titlepage}

\begin{minipage}{.3\textwidth}%
        \centering
        \includegraphics[width=1.\textwidth]{LOGO_Universite__libre_bruxelles.png}\par\vspace{0.2cm}
        \end{minipage}%
        \hfill
        \begin{minipage}{.25\textwidth}%
        \centering
        \includegraphics[width=1.\textwidth]{sciences_logo}\par\vspace{0.2cm}
        \end{minipage}%

\centering	\vspace{4cm}
	{\large Mémoire présenté en vue de l'obtention du diplôme de Master en
Sciences Physiques}\par\vspace{0.8cm}
	\rule{0.5\linewidth}{1pt}
	\vspace{0.8cm}\par 
	{\huge \textsc{Étude de spectres infrarouges d'étoiles géantes rouges évoluées} \par}
	\vspace{0.8cm}
	\rule{0.5\linewidth}{1pt}
	\vspace{1cm} \par 
	{\Large \textsc{Département de physique}} \par 
	{\Large \textsc{Institut d'Astronomie et d'Astrophysique}}\par\vspace{0.8cm}\vspace{3cm} \par 
	réalisé par\par
        {\large Margaux \textsc{Vandererven}\par \vspace{1cm} \par         
        }
		supervisé par\par
	\large Sophie \textsc{Van Eck}
	\vfill

% Bottom of the page
	{\large Année académique 2024-2025 \par}

\end{titlepage}



\pagestyle{empty}
\vfill

\centering	
	{\LARGE \textsc{Étude de spectres infrarouges d'étoiles géantes rouges évoluées} \par}
	\vspace{0.5cm}
	{\textsc{Institut d'Astronomie et d'Astrophysique}}\par\vspace{0.1cm} \par 
	\rule{0.3\linewidth}{0.4pt} \par
	\vspace{0.5cm}
	Margaux \textsc{Vandererven}\par \vspace{0.7cm} \par 
	\vspace{0.cm}
	
	
\abstract{\noindent La poussière}

\vspace{0.9cm}

\tableofcontents

\clearpage

% \pagestyle{main}
\section{Introduction}

\section{Débroussailler}


\section{Molécules}

Synthèse avec modèle : 

\begin{table}[h!]
  \begin{center}
  \begin{tabular}{ccccccc}
      \hline
      \hline
      Modèle & $T_{\rm eff}$ (K) & log $g$ (cm $s^{-2}$) & [Fe/H] (dex) & Mass (M$_\odot$) & [s/Fe] (dex)\\
      \hline
      A &  $4000$ & $1.00$ & -0.50 & 1.0& +0.00 \\
      \hline
  \end{tabular}
  \end{center}
  \textbf{Notes.} 4000g1.0z-0.50m1.0t02a+0.20c+0.346n+0.00o+0.20r+0.00s+0.00
  \label{MARCS}
  \end{table}


\begin{table}[h!]
%\caption{Listes de molécules et contribution de chacune d'elle dans la bande H et K}
\vspace{0.3cm}
\begin{minipage}[t]{.4\linewidth}
\begin{center}
	\begin{tabular}{cccc}
        \hline
		\hline
        Molécules & Bande H & Bande K & Cat. \\
        \hline
        $^{12}$C$^{14}$N & 55.14 \% & 44.35 \% & I\\
        $^{13}$C$^{14}$N & 32.00 \%  & 14.51 \%& I \\
        $^{12}$C$^{16}$O & 75.33 \% & 72.01 \% & I\\
        $^{13}$C$^{17}$O & 0.04 \% & 1.96 \% & II\\
        $^{16}$OH & 59.68 \% & 31.59 \% & I\\
        $^{56}$FeH& 3.12 \% & 0.08 \% & II\\
        HF & 17.79 \%  & 57.16 \%  & I\\
        $^{12}$CH & 4.68 \% & 10.68 \% & I\\
        $^{13}$CH & 0.15 \% & 0.39 \% & III\\
        $^{14}$NH & 1.57 \% & 1.23 \% & II\\
        $^{12}$C$^{12}$C & 32.97 \% & 30.73 \%& I \\
        $^{12}$C$^{13}$C & 14.12 \% & 12.26 \% & I\\
        $^{13}$C$^{13}$C & 0.38 \% & 0.25 \% & III\\
        C$_2$H$_2$ &  0.00  \% & 0.00 \% & III\\
        HCl & 0.64 \% &  0.50 \% & III\\
		H$_2$O & 1.75 \% & 6.80 \% & II\\
        $^{20}$CaH & < 0.01  \% & < 0.01  \% & III\\
        \hline
    \end{tabular}
\end{center} 
\end{minipage} 
\hspace{2.cm}
\begin{minipage}[t]{.4\linewidth}
\begin{center}
   \begin{tabular}{cccc}
        \hline
		\hline
        Molécules & Bande H & Bande K & Cat. \\
        \hline
        $^{28}$SiH & < 0.01 \% & 0.02 \% & III\\
        $^{28}$SiO & < 0.01 \% & 0.04 \% & III\\
		VO &  0.03 \% &  < 0.01 \% & III\\
		YO & < 0.01 \% &  < 0.01 \%& III\\
		$^{48}$TiO &  0.06 \% &  < 0.01 \% & III\\
		$^{24}$MgH &  < 0.01 \% &  < 0.01 \% & III \\
		AlH &  < 0.01 \% &  < 0.01 \% & III \\
		$^{52}$CrH & < 0.01 \% & 0.00 \% & III\\
		H$^{12}$CN & < 0.01 \% & < 0.01 \% & III\\
		H$^{13}$CN &   < 0.01 \% & < 0.01 \% & III\\
		$^{90}$ZrO & 0.02  \% & < 0.01  \% & III\\
		$^{91}$ZrO &  < 0.01 \% &  < 0.01 \% & III\\
		$^{92}$ZrO & < 0.01 \% &  < 0.01 \% & III\\
		$^{93}$ZrO &  0.00 \% & 0.00 \% & III\\
		$^{94}$ZrO &  < 0.01 \% &  < 0.01 \% & III\\
		$^{96}$ZrO &  < 0.01 \% &  < 0.01 \% & III\\
		 &  &  \\
        \hline
    \end{tabular}
	\end{center}
\end{minipage}
\vspace{0.3cm}

\textbf{Notes.} La catégorie I concerne les molécules contribuant à plus de 10\%, la catégorie II celles à moins de 10\% et plus de 1\% et la catégorie III celles à moins de 1\%. Les valeurs dans le colonne "Bande H" et "Bande K" sont le pourcentage en absorption du plus grand extremum du tracé de la différence entre le spectre synthétique atom + mol et le spectre synthétique atomique simple. log$\epsilon_C$ = 9,44, log$\epsilon_O$ = 9,56 et log$\epsilon_N$ = 8,38, MACturb = 8,00 kms-2, MICROturb = 1,70 et z = -0,3, ratio C = , ratio C/O=0.75.
\label{contribution_mol}
\end{table}

\section{TS}

\section{Macroturbulence}
Nous commençons par déterminer la macroturbulence à l'aide de raies propres dans le bande H et K. Nous effectuons des synthèses avec le modèle A de la Table~\ref{MARCS} et les abondances CNO et paramètres stellaires de la littérature \cite{shetye_s_2018}. Nous procédons par minimisation $\chi^2$ sur différentes raies propres avec comme paramètre libre l'abondance de l'élément et la macroturubulence. 
Les autres abondances sont scale à partir de \cite{grevesse_solar_2007}.

\begin{table}[h!]
%\caption{Listes de molécules et contribution de chacune d'elle dans la bande H et K}
\vspace{0.3cm}

\begin{center}
	\begin{tabular}{ccccccc}
        \hline
		\hline
        Élément & $\lambda$ (Å) & $\lambda_{min}$ (Å) & $\lambda_{max}$ (Å)& v$_{\mathrm{macro}}$ ($km s^{-1}$) & log $\varepsilon$ & $\chi^2$\\
        \hline
    Fe I & 15964.865 &&& 8.58 && \\
    % raie de bande H
    Fe I & 22257.107 &22256.48&22257.87& 8.65 &7.10& 0.0270\\
    Fe I & 22260.18 &22259.67&22260.77& 8.64 & 7.09 & 0.0132\\
		Co I & 16757.704&& & 8.61 && \\
		Ti I & 22211.22 & 22210.52 & 22211.91 & 8.90 & 4.88 & 0.0004\\
		Ti I & 22232.844&& & 9.00 &&\\
    \end{tabular}
\end{center} 
\textbf{Notes.} 
\label{macro}
\end{table}

\section{Abondance CNO}
\subsection{1 step} 
Abondance d'oxygène avec raie de OH, Cabu=8.44 (litt) et Nabu=7.38 (litt) 
avec un ratio de 12C/13C à 40 (valeur standard pour géantes)

\begin{center}
  \includegraphics[width=10cm]{../output/OH/first/abu_O.png}
\end{center} 

-> Oabu = 8.60 $\pm$ 0.02 et avec exclusion 8.59 $\pm$ 0.01

Puis on repasse sur les raies de CO pour checker C avec donc Oabu = 8.59 et Nabu = 7.38 (litt)
tjrs ratio 12C/13C. 

\begin{table}[h!]
  \vspace{0.3cm}
\begin{center}
	\begin{tabular}{ccccc}
        \hline
		\hline
        log $\varepsilon_{\rm O}$ & log $\varepsilon_{\rm C}$ & log $\varepsilon_{\rm N}$ & $^{12}$C/$^{13}$C & Element\\
        \hline
    \textcolor{darkblue}{8.59 $\pm$ 0.01} & 8.44 & 7.38 & 40 & $^{16}$OH \\
    % raie de bande H
    8.59 & \textcolor{darkblue}{7.82 $\pm$ 0.03} & 7.38 & 40 & $^{12}$C$^{16}$O \\
    8.59 & 7.82 & 7.38 & \textcolor{darkblue}{20} & $^{13}$C$^{17}$O \\
    \textcolor{darkblue}{8.30 $\pm$ 0.02} & 7.82 & 7.38 & 20 & $^{16}$OH \\
    8.30 & \textcolor{darkblue}{7.89 $\pm$ 0.02} & 7.38 & 20 & $^{12}$C$^{16}$O \\
    \textcolor{darkblue}{8.33 $\pm$ 0.02} & 7.89 & 7.38 & 20 & $^{16}$OH \\
    8.33 & \textcolor{darkblue}{7.86 $\pm$ 0.03} & 7.38 & 20 & $^{12}$C$^{16}$O \\
    8.33 & 7.86 & \textcolor{darkblue}{7.84 $\pm$ 0.02} & 20 & $^{12}$C$^{14}$N \\
    8.33 & 7.86 & 7.84 & \textcolor{blue}{12} & $^{13}$C$^{14}$N \\
    \textcolor{darkblue}{8.31 $\pm$ 0.01} & 7.86 & 7.84 & 12 & $^{16}$OH \\
    8.31 & \textcolor{darkblue}{7.88 $\pm$ 0.03} & 7.84 & 12 & $^{12}$C$^{16}$O \\
    8.31 & 7.88 & \textcolor{darkblue}{7.84 $\pm$ 0.02} & 12 & $^{12}$C$^{14}$N \\
    \end{tabular}
\end{center} 
\textbf{Notes.} 
Chaque synthèse est réalisée sur des raies de l'élement se trouvant en 4ème colonne. 
Les paramètres fixés sont en noir et le paramètre déterminé en bleu. 
\label{itération_CNO}
\end{table}

% \newpage
% \begin{figure}[htbp]
%   \centering
  
%   % Première image (graphique 1)
%   \begin{subfigure}[b]{0.45\textwidth}
%       \centering
%       \scalebox{0.45}{%% Creator: Matplotlib, PGF backend
%%
%% To include the figure in your LaTeX document, write
%%   \input{<filename>.pgf}
%%
%% Make sure the required packages are loaded in your preamble
%%   \usepackage{pgf}
%%
%% Also ensure that all the required font packages are loaded; for instance,
%% the lmodern package is sometimes necessary when using math font.
%%   \usepackage{lmodern}
%%
%% Figures using additional raster images can only be included by \input if
%% they are in the same directory as the main LaTeX file. For loading figures
%% from other directories you can use the `import` package
%%   \usepackage{import}
%%
%% and then include the figures with
%%   \import{<path to file>}{<filename>.pgf}
%%
%% Matplotlib used the following preamble
%%   \def\mathdefault#1{#1}
%%   \everymath=\expandafter{\the\everymath\displaystyle}
%%   
%%   \ifdefined\pdftexversion\else  % non-pdftex case.
%%     \usepackage{fontspec}
%%     \setmainfont{Times.ttc}[Path=\detokenize{/System/Library/Fonts/}]
%%     \setsansfont{DejaVuSans.ttf}[Path=\detokenize{/Users/margauxvandererven/anaconda3/lib/python3.11/site-packages/matplotlib/mpl-data/fonts/ttf/}]
%%     \setmonofont{DejaVuSansMono.ttf}[Path=\detokenize{/Users/margauxvandererven/anaconda3/lib/python3.11/site-packages/matplotlib/mpl-data/fonts/ttf/}]
%%   \fi
%%   \makeatletter\@ifpackageloaded{underscore}{}{\usepackage[strings]{underscore}}\makeatother
%%
\begingroup%
\makeatletter%
\begin{pgfpicture}%
\pgfpathrectangle{\pgfpointorigin}{\pgfqpoint{8.000000in}{6.000000in}}%
\pgfusepath{use as bounding box, clip}%
\begin{pgfscope}%
\pgfsetbuttcap%
\pgfsetmiterjoin%
\definecolor{currentfill}{rgb}{1.000000,1.000000,1.000000}%
\pgfsetfillcolor{currentfill}%
\pgfsetlinewidth{0.000000pt}%
\definecolor{currentstroke}{rgb}{1.000000,1.000000,1.000000}%
\pgfsetstrokecolor{currentstroke}%
\pgfsetdash{}{0pt}%
\pgfpathmoveto{\pgfqpoint{0.000000in}{0.000000in}}%
\pgfpathlineto{\pgfqpoint{8.000000in}{0.000000in}}%
\pgfpathlineto{\pgfqpoint{8.000000in}{6.000000in}}%
\pgfpathlineto{\pgfqpoint{0.000000in}{6.000000in}}%
\pgfpathlineto{\pgfqpoint{0.000000in}{0.000000in}}%
\pgfpathclose%
\pgfusepath{fill}%
\end{pgfscope}%
\begin{pgfscope}%
\pgfsetbuttcap%
\pgfsetmiterjoin%
\definecolor{currentfill}{rgb}{1.000000,1.000000,1.000000}%
\pgfsetfillcolor{currentfill}%
\pgfsetlinewidth{0.000000pt}%
\definecolor{currentstroke}{rgb}{0.000000,0.000000,0.000000}%
\pgfsetstrokecolor{currentstroke}%
\pgfsetstrokeopacity{0.000000}%
\pgfsetdash{}{0pt}%
\pgfpathmoveto{\pgfqpoint{1.000000in}{0.660000in}}%
\pgfpathlineto{\pgfqpoint{7.200000in}{0.660000in}}%
\pgfpathlineto{\pgfqpoint{7.200000in}{5.280000in}}%
\pgfpathlineto{\pgfqpoint{1.000000in}{5.280000in}}%
\pgfpathlineto{\pgfqpoint{1.000000in}{0.660000in}}%
\pgfpathclose%
\pgfusepath{fill}%
\end{pgfscope}%
\begin{pgfscope}%
\pgfpathrectangle{\pgfqpoint{1.000000in}{0.660000in}}{\pgfqpoint{6.200000in}{4.620000in}}%
\pgfusepath{clip}%
\pgfsetbuttcap%
\pgfsetroundjoin%
\definecolor{currentfill}{rgb}{0.000000,0.000000,0.545098}%
\pgfsetfillcolor{currentfill}%
\pgfsetlinewidth{1.505625pt}%
\definecolor{currentstroke}{rgb}{0.000000,0.000000,0.545098}%
\pgfsetstrokecolor{currentstroke}%
\pgfsetdash{}{0pt}%
\pgfsys@defobject{currentmarker}{\pgfqpoint{-0.041667in}{-0.041667in}}{\pgfqpoint{0.041667in}{0.041667in}}{%
\pgfpathmoveto{\pgfqpoint{-0.041667in}{-0.041667in}}%
\pgfpathlineto{\pgfqpoint{0.041667in}{0.041667in}}%
\pgfpathmoveto{\pgfqpoint{-0.041667in}{0.041667in}}%
\pgfpathlineto{\pgfqpoint{0.041667in}{-0.041667in}}%
\pgfusepath{stroke,fill}%
}%
\begin{pgfscope}%
\pgfsys@transformshift{1.281818in}{3.010460in}%
\pgfsys@useobject{currentmarker}{}%
\end{pgfscope}%
\begin{pgfscope}%
\pgfsys@transformshift{2.534343in}{1.300169in}%
\pgfsys@useobject{currentmarker}{}%
\end{pgfscope}%
\begin{pgfscope}%
\pgfsys@transformshift{3.786869in}{0.871443in}%
\pgfsys@useobject{currentmarker}{}%
\end{pgfscope}%
\begin{pgfscope}%
\pgfsys@transformshift{5.039394in}{1.678055in}%
\pgfsys@useobject{currentmarker}{}%
\end{pgfscope}%
\begin{pgfscope}%
\pgfsys@transformshift{6.918182in}{5.058638in}%
\pgfsys@useobject{currentmarker}{}%
\end{pgfscope}%
\end{pgfscope}%
\begin{pgfscope}%
\pgfpathrectangle{\pgfqpoint{1.000000in}{0.660000in}}{\pgfqpoint{6.200000in}{4.620000in}}%
\pgfusepath{clip}%
\pgfsetbuttcap%
\pgfsetroundjoin%
\definecolor{currentfill}{rgb}{1.000000,0.000000,0.000000}%
\pgfsetfillcolor{currentfill}%
\pgfsetlinewidth{1.505625pt}%
\definecolor{currentstroke}{rgb}{1.000000,0.000000,0.000000}%
\pgfsetstrokecolor{currentstroke}%
\pgfsetdash{}{0pt}%
\pgfsys@defobject{currentmarker}{\pgfqpoint{-0.041667in}{-0.041667in}}{\pgfqpoint{0.041667in}{0.041667in}}{%
\pgfpathmoveto{\pgfqpoint{-0.041667in}{-0.041667in}}%
\pgfpathlineto{\pgfqpoint{0.041667in}{0.041667in}}%
\pgfpathmoveto{\pgfqpoint{-0.041667in}{0.041667in}}%
\pgfpathlineto{\pgfqpoint{0.041667in}{-0.041667in}}%
\pgfusepath{stroke,fill}%
}%
\begin{pgfscope}%
\pgfsys@transformshift{3.624102in}{0.870000in}%
\pgfsys@useobject{currentmarker}{}%
\end{pgfscope}%
\end{pgfscope}%
\begin{pgfscope}%
\pgfsetbuttcap%
\pgfsetroundjoin%
\definecolor{currentfill}{rgb}{0.000000,0.000000,0.000000}%
\pgfsetfillcolor{currentfill}%
\pgfsetlinewidth{0.803000pt}%
\definecolor{currentstroke}{rgb}{0.000000,0.000000,0.000000}%
\pgfsetstrokecolor{currentstroke}%
\pgfsetdash{}{0pt}%
\pgfsys@defobject{currentmarker}{\pgfqpoint{0.000000in}{0.000000in}}{\pgfqpoint{0.000000in}{0.138889in}}{%
\pgfpathmoveto{\pgfqpoint{0.000000in}{0.000000in}}%
\pgfpathlineto{\pgfqpoint{0.000000in}{0.138889in}}%
\pgfusepath{stroke,fill}%
}%
\begin{pgfscope}%
\pgfsys@transformshift{1.281818in}{0.660000in}%
\pgfsys@useobject{currentmarker}{}%
\end{pgfscope}%
\end{pgfscope}%
\begin{pgfscope}%
\pgfsetbuttcap%
\pgfsetroundjoin%
\definecolor{currentfill}{rgb}{0.000000,0.000000,0.000000}%
\pgfsetfillcolor{currentfill}%
\pgfsetlinewidth{0.803000pt}%
\definecolor{currentstroke}{rgb}{0.000000,0.000000,0.000000}%
\pgfsetstrokecolor{currentstroke}%
\pgfsetdash{}{0pt}%
\pgfsys@defobject{currentmarker}{\pgfqpoint{0.000000in}{-0.138889in}}{\pgfqpoint{0.000000in}{0.000000in}}{%
\pgfpathmoveto{\pgfqpoint{0.000000in}{0.000000in}}%
\pgfpathlineto{\pgfqpoint{0.000000in}{-0.138889in}}%
\pgfusepath{stroke,fill}%
}%
\begin{pgfscope}%
\pgfsys@transformshift{1.281818in}{5.280000in}%
\pgfsys@useobject{currentmarker}{}%
\end{pgfscope}%
\end{pgfscope}%
\begin{pgfscope}%
\definecolor{textcolor}{rgb}{0.000000,0.000000,0.000000}%
\pgfsetstrokecolor{textcolor}%
\pgfsetfillcolor{textcolor}%
\pgftext[x=1.281818in,y=0.611389in,,top]{\color{textcolor}{\rmfamily\fontsize{16.000000}{19.200000}\selectfont\catcode`\^=\active\def^{\ifmmode\sp\else\^{}\fi}\catcode`\%=\active\def%{\%}8.56}}%
\end{pgfscope}%
\begin{pgfscope}%
\pgfsetbuttcap%
\pgfsetroundjoin%
\definecolor{currentfill}{rgb}{0.000000,0.000000,0.000000}%
\pgfsetfillcolor{currentfill}%
\pgfsetlinewidth{0.803000pt}%
\definecolor{currentstroke}{rgb}{0.000000,0.000000,0.000000}%
\pgfsetstrokecolor{currentstroke}%
\pgfsetdash{}{0pt}%
\pgfsys@defobject{currentmarker}{\pgfqpoint{0.000000in}{0.000000in}}{\pgfqpoint{0.000000in}{0.138889in}}{%
\pgfpathmoveto{\pgfqpoint{0.000000in}{0.000000in}}%
\pgfpathlineto{\pgfqpoint{0.000000in}{0.138889in}}%
\pgfusepath{stroke,fill}%
}%
\begin{pgfscope}%
\pgfsys@transformshift{2.534343in}{0.660000in}%
\pgfsys@useobject{currentmarker}{}%
\end{pgfscope}%
\end{pgfscope}%
\begin{pgfscope}%
\pgfsetbuttcap%
\pgfsetroundjoin%
\definecolor{currentfill}{rgb}{0.000000,0.000000,0.000000}%
\pgfsetfillcolor{currentfill}%
\pgfsetlinewidth{0.803000pt}%
\definecolor{currentstroke}{rgb}{0.000000,0.000000,0.000000}%
\pgfsetstrokecolor{currentstroke}%
\pgfsetdash{}{0pt}%
\pgfsys@defobject{currentmarker}{\pgfqpoint{0.000000in}{-0.138889in}}{\pgfqpoint{0.000000in}{0.000000in}}{%
\pgfpathmoveto{\pgfqpoint{0.000000in}{0.000000in}}%
\pgfpathlineto{\pgfqpoint{0.000000in}{-0.138889in}}%
\pgfusepath{stroke,fill}%
}%
\begin{pgfscope}%
\pgfsys@transformshift{2.534343in}{5.280000in}%
\pgfsys@useobject{currentmarker}{}%
\end{pgfscope}%
\end{pgfscope}%
\begin{pgfscope}%
\definecolor{textcolor}{rgb}{0.000000,0.000000,0.000000}%
\pgfsetstrokecolor{textcolor}%
\pgfsetfillcolor{textcolor}%
\pgftext[x=2.534343in,y=0.611389in,,top]{\color{textcolor}{\rmfamily\fontsize{16.000000}{19.200000}\selectfont\catcode`\^=\active\def^{\ifmmode\sp\else\^{}\fi}\catcode`\%=\active\def%{\%}8.58}}%
\end{pgfscope}%
\begin{pgfscope}%
\pgfsetbuttcap%
\pgfsetroundjoin%
\definecolor{currentfill}{rgb}{0.000000,0.000000,0.000000}%
\pgfsetfillcolor{currentfill}%
\pgfsetlinewidth{0.803000pt}%
\definecolor{currentstroke}{rgb}{0.000000,0.000000,0.000000}%
\pgfsetstrokecolor{currentstroke}%
\pgfsetdash{}{0pt}%
\pgfsys@defobject{currentmarker}{\pgfqpoint{0.000000in}{0.000000in}}{\pgfqpoint{0.000000in}{0.138889in}}{%
\pgfpathmoveto{\pgfqpoint{0.000000in}{0.000000in}}%
\pgfpathlineto{\pgfqpoint{0.000000in}{0.138889in}}%
\pgfusepath{stroke,fill}%
}%
\begin{pgfscope}%
\pgfsys@transformshift{3.786869in}{0.660000in}%
\pgfsys@useobject{currentmarker}{}%
\end{pgfscope}%
\end{pgfscope}%
\begin{pgfscope}%
\pgfsetbuttcap%
\pgfsetroundjoin%
\definecolor{currentfill}{rgb}{0.000000,0.000000,0.000000}%
\pgfsetfillcolor{currentfill}%
\pgfsetlinewidth{0.803000pt}%
\definecolor{currentstroke}{rgb}{0.000000,0.000000,0.000000}%
\pgfsetstrokecolor{currentstroke}%
\pgfsetdash{}{0pt}%
\pgfsys@defobject{currentmarker}{\pgfqpoint{0.000000in}{-0.138889in}}{\pgfqpoint{0.000000in}{0.000000in}}{%
\pgfpathmoveto{\pgfqpoint{0.000000in}{0.000000in}}%
\pgfpathlineto{\pgfqpoint{0.000000in}{-0.138889in}}%
\pgfusepath{stroke,fill}%
}%
\begin{pgfscope}%
\pgfsys@transformshift{3.786869in}{5.280000in}%
\pgfsys@useobject{currentmarker}{}%
\end{pgfscope}%
\end{pgfscope}%
\begin{pgfscope}%
\definecolor{textcolor}{rgb}{0.000000,0.000000,0.000000}%
\pgfsetstrokecolor{textcolor}%
\pgfsetfillcolor{textcolor}%
\pgftext[x=3.786869in,y=0.611389in,,top]{\color{textcolor}{\rmfamily\fontsize{16.000000}{19.200000}\selectfont\catcode`\^=\active\def^{\ifmmode\sp\else\^{}\fi}\catcode`\%=\active\def%{\%}8.60}}%
\end{pgfscope}%
\begin{pgfscope}%
\pgfsetbuttcap%
\pgfsetroundjoin%
\definecolor{currentfill}{rgb}{0.000000,0.000000,0.000000}%
\pgfsetfillcolor{currentfill}%
\pgfsetlinewidth{0.803000pt}%
\definecolor{currentstroke}{rgb}{0.000000,0.000000,0.000000}%
\pgfsetstrokecolor{currentstroke}%
\pgfsetdash{}{0pt}%
\pgfsys@defobject{currentmarker}{\pgfqpoint{0.000000in}{0.000000in}}{\pgfqpoint{0.000000in}{0.138889in}}{%
\pgfpathmoveto{\pgfqpoint{0.000000in}{0.000000in}}%
\pgfpathlineto{\pgfqpoint{0.000000in}{0.138889in}}%
\pgfusepath{stroke,fill}%
}%
\begin{pgfscope}%
\pgfsys@transformshift{5.039394in}{0.660000in}%
\pgfsys@useobject{currentmarker}{}%
\end{pgfscope}%
\end{pgfscope}%
\begin{pgfscope}%
\pgfsetbuttcap%
\pgfsetroundjoin%
\definecolor{currentfill}{rgb}{0.000000,0.000000,0.000000}%
\pgfsetfillcolor{currentfill}%
\pgfsetlinewidth{0.803000pt}%
\definecolor{currentstroke}{rgb}{0.000000,0.000000,0.000000}%
\pgfsetstrokecolor{currentstroke}%
\pgfsetdash{}{0pt}%
\pgfsys@defobject{currentmarker}{\pgfqpoint{0.000000in}{-0.138889in}}{\pgfqpoint{0.000000in}{0.000000in}}{%
\pgfpathmoveto{\pgfqpoint{0.000000in}{0.000000in}}%
\pgfpathlineto{\pgfqpoint{0.000000in}{-0.138889in}}%
\pgfusepath{stroke,fill}%
}%
\begin{pgfscope}%
\pgfsys@transformshift{5.039394in}{5.280000in}%
\pgfsys@useobject{currentmarker}{}%
\end{pgfscope}%
\end{pgfscope}%
\begin{pgfscope}%
\definecolor{textcolor}{rgb}{0.000000,0.000000,0.000000}%
\pgfsetstrokecolor{textcolor}%
\pgfsetfillcolor{textcolor}%
\pgftext[x=5.039394in,y=0.611389in,,top]{\color{textcolor}{\rmfamily\fontsize{16.000000}{19.200000}\selectfont\catcode`\^=\active\def^{\ifmmode\sp\else\^{}\fi}\catcode`\%=\active\def%{\%}8.62}}%
\end{pgfscope}%
\begin{pgfscope}%
\pgfsetbuttcap%
\pgfsetroundjoin%
\definecolor{currentfill}{rgb}{0.000000,0.000000,0.000000}%
\pgfsetfillcolor{currentfill}%
\pgfsetlinewidth{0.803000pt}%
\definecolor{currentstroke}{rgb}{0.000000,0.000000,0.000000}%
\pgfsetstrokecolor{currentstroke}%
\pgfsetdash{}{0pt}%
\pgfsys@defobject{currentmarker}{\pgfqpoint{0.000000in}{0.000000in}}{\pgfqpoint{0.000000in}{0.138889in}}{%
\pgfpathmoveto{\pgfqpoint{0.000000in}{0.000000in}}%
\pgfpathlineto{\pgfqpoint{0.000000in}{0.138889in}}%
\pgfusepath{stroke,fill}%
}%
\begin{pgfscope}%
\pgfsys@transformshift{6.291919in}{0.660000in}%
\pgfsys@useobject{currentmarker}{}%
\end{pgfscope}%
\end{pgfscope}%
\begin{pgfscope}%
\pgfsetbuttcap%
\pgfsetroundjoin%
\definecolor{currentfill}{rgb}{0.000000,0.000000,0.000000}%
\pgfsetfillcolor{currentfill}%
\pgfsetlinewidth{0.803000pt}%
\definecolor{currentstroke}{rgb}{0.000000,0.000000,0.000000}%
\pgfsetstrokecolor{currentstroke}%
\pgfsetdash{}{0pt}%
\pgfsys@defobject{currentmarker}{\pgfqpoint{0.000000in}{-0.138889in}}{\pgfqpoint{0.000000in}{0.000000in}}{%
\pgfpathmoveto{\pgfqpoint{0.000000in}{0.000000in}}%
\pgfpathlineto{\pgfqpoint{0.000000in}{-0.138889in}}%
\pgfusepath{stroke,fill}%
}%
\begin{pgfscope}%
\pgfsys@transformshift{6.291919in}{5.280000in}%
\pgfsys@useobject{currentmarker}{}%
\end{pgfscope}%
\end{pgfscope}%
\begin{pgfscope}%
\definecolor{textcolor}{rgb}{0.000000,0.000000,0.000000}%
\pgfsetstrokecolor{textcolor}%
\pgfsetfillcolor{textcolor}%
\pgftext[x=6.291919in,y=0.611389in,,top]{\color{textcolor}{\rmfamily\fontsize{16.000000}{19.200000}\selectfont\catcode`\^=\active\def^{\ifmmode\sp\else\^{}\fi}\catcode`\%=\active\def%{\%}8.64}}%
\end{pgfscope}%
\begin{pgfscope}%
\definecolor{textcolor}{rgb}{0.000000,0.000000,0.000000}%
\pgfsetstrokecolor{textcolor}%
\pgfsetfillcolor{textcolor}%
\pgftext[x=4.100000in,y=0.355096in,,top]{\color{textcolor}{\rmfamily\fontsize{16.000000}{19.200000}\selectfont\catcode`\^=\active\def^{\ifmmode\sp\else\^{}\fi}\catcode`\%=\active\def%{\%}$\log \epsilon_{\mathrm{O}}$}}%
\end{pgfscope}%
\begin{pgfscope}%
\pgfsetbuttcap%
\pgfsetroundjoin%
\definecolor{currentfill}{rgb}{0.000000,0.000000,0.000000}%
\pgfsetfillcolor{currentfill}%
\pgfsetlinewidth{0.803000pt}%
\definecolor{currentstroke}{rgb}{0.000000,0.000000,0.000000}%
\pgfsetstrokecolor{currentstroke}%
\pgfsetdash{}{0pt}%
\pgfsys@defobject{currentmarker}{\pgfqpoint{0.000000in}{0.000000in}}{\pgfqpoint{0.138889in}{0.000000in}}{%
\pgfpathmoveto{\pgfqpoint{0.000000in}{0.000000in}}%
\pgfpathlineto{\pgfqpoint{0.138889in}{0.000000in}}%
\pgfusepath{stroke,fill}%
}%
\begin{pgfscope}%
\pgfsys@transformshift{1.000000in}{1.442351in}%
\pgfsys@useobject{currentmarker}{}%
\end{pgfscope}%
\end{pgfscope}%
\begin{pgfscope}%
\pgfsetbuttcap%
\pgfsetroundjoin%
\definecolor{currentfill}{rgb}{0.000000,0.000000,0.000000}%
\pgfsetfillcolor{currentfill}%
\pgfsetlinewidth{0.803000pt}%
\definecolor{currentstroke}{rgb}{0.000000,0.000000,0.000000}%
\pgfsetstrokecolor{currentstroke}%
\pgfsetdash{}{0pt}%
\pgfsys@defobject{currentmarker}{\pgfqpoint{-0.138889in}{0.000000in}}{\pgfqpoint{-0.000000in}{0.000000in}}{%
\pgfpathmoveto{\pgfqpoint{-0.000000in}{0.000000in}}%
\pgfpathlineto{\pgfqpoint{-0.138889in}{0.000000in}}%
\pgfusepath{stroke,fill}%
}%
\begin{pgfscope}%
\pgfsys@transformshift{7.200000in}{1.442351in}%
\pgfsys@useobject{currentmarker}{}%
\end{pgfscope}%
\end{pgfscope}%
\begin{pgfscope}%
\definecolor{textcolor}{rgb}{0.000000,0.000000,0.000000}%
\pgfsetstrokecolor{textcolor}%
\pgfsetfillcolor{textcolor}%
\pgftext[x=0.562500in, y=1.366179in, left, base]{\color{textcolor}{\rmfamily\fontsize{16.000000}{19.200000}\selectfont\catcode`\^=\active\def^{\ifmmode\sp\else\^{}\fi}\catcode`\%=\active\def%{\%}0.02}}%
\end{pgfscope}%
\begin{pgfscope}%
\pgfsetbuttcap%
\pgfsetroundjoin%
\definecolor{currentfill}{rgb}{0.000000,0.000000,0.000000}%
\pgfsetfillcolor{currentfill}%
\pgfsetlinewidth{0.803000pt}%
\definecolor{currentstroke}{rgb}{0.000000,0.000000,0.000000}%
\pgfsetstrokecolor{currentstroke}%
\pgfsetdash{}{0pt}%
\pgfsys@defobject{currentmarker}{\pgfqpoint{0.000000in}{0.000000in}}{\pgfqpoint{0.138889in}{0.000000in}}{%
\pgfpathmoveto{\pgfqpoint{0.000000in}{0.000000in}}%
\pgfpathlineto{\pgfqpoint{0.138889in}{0.000000in}}%
\pgfusepath{stroke,fill}%
}%
\begin{pgfscope}%
\pgfsys@transformshift{1.000000in}{2.444869in}%
\pgfsys@useobject{currentmarker}{}%
\end{pgfscope}%
\end{pgfscope}%
\begin{pgfscope}%
\pgfsetbuttcap%
\pgfsetroundjoin%
\definecolor{currentfill}{rgb}{0.000000,0.000000,0.000000}%
\pgfsetfillcolor{currentfill}%
\pgfsetlinewidth{0.803000pt}%
\definecolor{currentstroke}{rgb}{0.000000,0.000000,0.000000}%
\pgfsetstrokecolor{currentstroke}%
\pgfsetdash{}{0pt}%
\pgfsys@defobject{currentmarker}{\pgfqpoint{-0.138889in}{0.000000in}}{\pgfqpoint{-0.000000in}{0.000000in}}{%
\pgfpathmoveto{\pgfqpoint{-0.000000in}{0.000000in}}%
\pgfpathlineto{\pgfqpoint{-0.138889in}{0.000000in}}%
\pgfusepath{stroke,fill}%
}%
\begin{pgfscope}%
\pgfsys@transformshift{7.200000in}{2.444869in}%
\pgfsys@useobject{currentmarker}{}%
\end{pgfscope}%
\end{pgfscope}%
\begin{pgfscope}%
\definecolor{textcolor}{rgb}{0.000000,0.000000,0.000000}%
\pgfsetstrokecolor{textcolor}%
\pgfsetfillcolor{textcolor}%
\pgftext[x=0.562500in, y=2.368697in, left, base]{\color{textcolor}{\rmfamily\fontsize{16.000000}{19.200000}\selectfont\catcode`\^=\active\def^{\ifmmode\sp\else\^{}\fi}\catcode`\%=\active\def%{\%}0.03}}%
\end{pgfscope}%
\begin{pgfscope}%
\pgfsetbuttcap%
\pgfsetroundjoin%
\definecolor{currentfill}{rgb}{0.000000,0.000000,0.000000}%
\pgfsetfillcolor{currentfill}%
\pgfsetlinewidth{0.803000pt}%
\definecolor{currentstroke}{rgb}{0.000000,0.000000,0.000000}%
\pgfsetstrokecolor{currentstroke}%
\pgfsetdash{}{0pt}%
\pgfsys@defobject{currentmarker}{\pgfqpoint{0.000000in}{0.000000in}}{\pgfqpoint{0.138889in}{0.000000in}}{%
\pgfpathmoveto{\pgfqpoint{0.000000in}{0.000000in}}%
\pgfpathlineto{\pgfqpoint{0.138889in}{0.000000in}}%
\pgfusepath{stroke,fill}%
}%
\begin{pgfscope}%
\pgfsys@transformshift{1.000000in}{3.447386in}%
\pgfsys@useobject{currentmarker}{}%
\end{pgfscope}%
\end{pgfscope}%
\begin{pgfscope}%
\pgfsetbuttcap%
\pgfsetroundjoin%
\definecolor{currentfill}{rgb}{0.000000,0.000000,0.000000}%
\pgfsetfillcolor{currentfill}%
\pgfsetlinewidth{0.803000pt}%
\definecolor{currentstroke}{rgb}{0.000000,0.000000,0.000000}%
\pgfsetstrokecolor{currentstroke}%
\pgfsetdash{}{0pt}%
\pgfsys@defobject{currentmarker}{\pgfqpoint{-0.138889in}{0.000000in}}{\pgfqpoint{-0.000000in}{0.000000in}}{%
\pgfpathmoveto{\pgfqpoint{-0.000000in}{0.000000in}}%
\pgfpathlineto{\pgfqpoint{-0.138889in}{0.000000in}}%
\pgfusepath{stroke,fill}%
}%
\begin{pgfscope}%
\pgfsys@transformshift{7.200000in}{3.447386in}%
\pgfsys@useobject{currentmarker}{}%
\end{pgfscope}%
\end{pgfscope}%
\begin{pgfscope}%
\definecolor{textcolor}{rgb}{0.000000,0.000000,0.000000}%
\pgfsetstrokecolor{textcolor}%
\pgfsetfillcolor{textcolor}%
\pgftext[x=0.562500in, y=3.371215in, left, base]{\color{textcolor}{\rmfamily\fontsize{16.000000}{19.200000}\selectfont\catcode`\^=\active\def^{\ifmmode\sp\else\^{}\fi}\catcode`\%=\active\def%{\%}0.04}}%
\end{pgfscope}%
\begin{pgfscope}%
\pgfsetbuttcap%
\pgfsetroundjoin%
\definecolor{currentfill}{rgb}{0.000000,0.000000,0.000000}%
\pgfsetfillcolor{currentfill}%
\pgfsetlinewidth{0.803000pt}%
\definecolor{currentstroke}{rgb}{0.000000,0.000000,0.000000}%
\pgfsetstrokecolor{currentstroke}%
\pgfsetdash{}{0pt}%
\pgfsys@defobject{currentmarker}{\pgfqpoint{0.000000in}{0.000000in}}{\pgfqpoint{0.138889in}{0.000000in}}{%
\pgfpathmoveto{\pgfqpoint{0.000000in}{0.000000in}}%
\pgfpathlineto{\pgfqpoint{0.138889in}{0.000000in}}%
\pgfusepath{stroke,fill}%
}%
\begin{pgfscope}%
\pgfsys@transformshift{1.000000in}{4.449904in}%
\pgfsys@useobject{currentmarker}{}%
\end{pgfscope}%
\end{pgfscope}%
\begin{pgfscope}%
\pgfsetbuttcap%
\pgfsetroundjoin%
\definecolor{currentfill}{rgb}{0.000000,0.000000,0.000000}%
\pgfsetfillcolor{currentfill}%
\pgfsetlinewidth{0.803000pt}%
\definecolor{currentstroke}{rgb}{0.000000,0.000000,0.000000}%
\pgfsetstrokecolor{currentstroke}%
\pgfsetdash{}{0pt}%
\pgfsys@defobject{currentmarker}{\pgfqpoint{-0.138889in}{0.000000in}}{\pgfqpoint{-0.000000in}{0.000000in}}{%
\pgfpathmoveto{\pgfqpoint{-0.000000in}{0.000000in}}%
\pgfpathlineto{\pgfqpoint{-0.138889in}{0.000000in}}%
\pgfusepath{stroke,fill}%
}%
\begin{pgfscope}%
\pgfsys@transformshift{7.200000in}{4.449904in}%
\pgfsys@useobject{currentmarker}{}%
\end{pgfscope}%
\end{pgfscope}%
\begin{pgfscope}%
\definecolor{textcolor}{rgb}{0.000000,0.000000,0.000000}%
\pgfsetstrokecolor{textcolor}%
\pgfsetfillcolor{textcolor}%
\pgftext[x=0.562500in, y=4.373732in, left, base]{\color{textcolor}{\rmfamily\fontsize{16.000000}{19.200000}\selectfont\catcode`\^=\active\def^{\ifmmode\sp\else\^{}\fi}\catcode`\%=\active\def%{\%}0.05}}%
\end{pgfscope}%
\begin{pgfscope}%
\definecolor{textcolor}{rgb}{0.000000,0.000000,0.000000}%
\pgfsetstrokecolor{textcolor}%
\pgfsetfillcolor{textcolor}%
\pgftext[x=0.506944in,y=2.970000in,,bottom,rotate=90.000000]{\color{textcolor}{\rmfamily\fontsize{16.000000}{19.200000}\selectfont\catcode`\^=\active\def^{\ifmmode\sp\else\^{}\fi}\catcode`\%=\active\def%{\%}$\chi^2$}}%
\end{pgfscope}%
\begin{pgfscope}%
\pgfpathrectangle{\pgfqpoint{1.000000in}{0.660000in}}{\pgfqpoint{6.200000in}{4.620000in}}%
\pgfusepath{clip}%
\pgfsetrectcap%
\pgfsetroundjoin%
\pgfsetlinewidth{1.505625pt}%
\definecolor{currentstroke}{rgb}{0.827451,0.827451,0.827451}%
\pgfsetstrokecolor{currentstroke}%
\pgfsetdash{}{0pt}%
\pgfpathmoveto{\pgfqpoint{1.281818in}{2.993539in}}%
\pgfpathlineto{\pgfqpoint{1.338751in}{2.891562in}}%
\pgfpathlineto{\pgfqpoint{1.395684in}{2.792094in}}%
\pgfpathlineto{\pgfqpoint{1.452617in}{2.695135in}}%
\pgfpathlineto{\pgfqpoint{1.509550in}{2.600685in}}%
\pgfpathlineto{\pgfqpoint{1.566483in}{2.508744in}}%
\pgfpathlineto{\pgfqpoint{1.623416in}{2.419313in}}%
\pgfpathlineto{\pgfqpoint{1.680349in}{2.332391in}}%
\pgfpathlineto{\pgfqpoint{1.737282in}{2.247978in}}%
\pgfpathlineto{\pgfqpoint{1.794215in}{2.166074in}}%
\pgfpathlineto{\pgfqpoint{1.851148in}{2.086680in}}%
\pgfpathlineto{\pgfqpoint{1.908081in}{2.009795in}}%
\pgfpathlineto{\pgfqpoint{1.965014in}{1.935419in}}%
\pgfpathlineto{\pgfqpoint{2.021947in}{1.863552in}}%
\pgfpathlineto{\pgfqpoint{2.078880in}{1.794194in}}%
\pgfpathlineto{\pgfqpoint{2.135813in}{1.727346in}}%
\pgfpathlineto{\pgfqpoint{2.192746in}{1.663007in}}%
\pgfpathlineto{\pgfqpoint{2.249679in}{1.601177in}}%
\pgfpathlineto{\pgfqpoint{2.306612in}{1.541856in}}%
\pgfpathlineto{\pgfqpoint{2.363545in}{1.485045in}}%
\pgfpathlineto{\pgfqpoint{2.420478in}{1.430743in}}%
\pgfpathlineto{\pgfqpoint{2.477410in}{1.378949in}}%
\pgfpathlineto{\pgfqpoint{2.534343in}{1.329666in}}%
\pgfpathlineto{\pgfqpoint{2.591276in}{1.282891in}}%
\pgfpathlineto{\pgfqpoint{2.648209in}{1.238626in}}%
\pgfpathlineto{\pgfqpoint{2.705142in}{1.196870in}}%
\pgfpathlineto{\pgfqpoint{2.762075in}{1.157623in}}%
\pgfpathlineto{\pgfqpoint{2.819008in}{1.120885in}}%
\pgfpathlineto{\pgfqpoint{2.875941in}{1.086656in}}%
\pgfpathlineto{\pgfqpoint{2.932874in}{1.054937in}}%
\pgfpathlineto{\pgfqpoint{2.989807in}{1.025727in}}%
\pgfpathlineto{\pgfqpoint{3.046740in}{0.999026in}}%
\pgfpathlineto{\pgfqpoint{3.103673in}{0.974835in}}%
\pgfpathlineto{\pgfqpoint{3.160606in}{0.953152in}}%
\pgfpathlineto{\pgfqpoint{3.217539in}{0.933979in}}%
\pgfpathlineto{\pgfqpoint{3.274472in}{0.917315in}}%
\pgfpathlineto{\pgfqpoint{3.331405in}{0.903160in}}%
\pgfpathlineto{\pgfqpoint{3.388338in}{0.891515in}}%
\pgfpathlineto{\pgfqpoint{3.445271in}{0.882379in}}%
\pgfpathlineto{\pgfqpoint{3.502204in}{0.875751in}}%
\pgfpathlineto{\pgfqpoint{3.559137in}{0.871634in}}%
\pgfpathlineto{\pgfqpoint{3.616070in}{0.870025in}}%
\pgfpathlineto{\pgfqpoint{3.673003in}{0.870926in}}%
\pgfpathlineto{\pgfqpoint{3.729936in}{0.874335in}}%
\pgfpathlineto{\pgfqpoint{3.786869in}{0.880254in}}%
\pgfpathlineto{\pgfqpoint{3.843802in}{0.888683in}}%
\pgfpathlineto{\pgfqpoint{3.900735in}{0.899620in}}%
\pgfpathlineto{\pgfqpoint{3.957668in}{0.913067in}}%
\pgfpathlineto{\pgfqpoint{4.014601in}{0.929023in}}%
\pgfpathlineto{\pgfqpoint{4.071534in}{0.947488in}}%
\pgfpathlineto{\pgfqpoint{4.128466in}{0.968462in}}%
\pgfpathlineto{\pgfqpoint{4.185399in}{0.991946in}}%
\pgfpathlineto{\pgfqpoint{4.242332in}{1.017939in}}%
\pgfpathlineto{\pgfqpoint{4.299265in}{1.046441in}}%
\pgfpathlineto{\pgfqpoint{4.356198in}{1.077452in}}%
\pgfpathlineto{\pgfqpoint{4.413131in}{1.110972in}}%
\pgfpathlineto{\pgfqpoint{4.470064in}{1.147002in}}%
\pgfpathlineto{\pgfqpoint{4.526997in}{1.185541in}}%
\pgfpathlineto{\pgfqpoint{4.583930in}{1.226589in}}%
\pgfpathlineto{\pgfqpoint{4.640863in}{1.270146in}}%
\pgfpathlineto{\pgfqpoint{4.697796in}{1.316213in}}%
\pgfpathlineto{\pgfqpoint{4.754729in}{1.364789in}}%
\pgfpathlineto{\pgfqpoint{4.811662in}{1.415874in}}%
\pgfpathlineto{\pgfqpoint{4.868595in}{1.469468in}}%
\pgfpathlineto{\pgfqpoint{4.925528in}{1.525571in}}%
\pgfpathlineto{\pgfqpoint{4.982461in}{1.584184in}}%
\pgfpathlineto{\pgfqpoint{5.039394in}{1.645306in}}%
\pgfpathlineto{\pgfqpoint{5.096327in}{1.708937in}}%
\pgfpathlineto{\pgfqpoint{5.153260in}{1.775077in}}%
\pgfpathlineto{\pgfqpoint{5.210193in}{1.843727in}}%
\pgfpathlineto{\pgfqpoint{5.267126in}{1.914885in}}%
\pgfpathlineto{\pgfqpoint{5.324059in}{1.988553in}}%
\pgfpathlineto{\pgfqpoint{5.380992in}{2.064731in}}%
\pgfpathlineto{\pgfqpoint{5.437925in}{2.143417in}}%
\pgfpathlineto{\pgfqpoint{5.494858in}{2.224613in}}%
\pgfpathlineto{\pgfqpoint{5.551791in}{2.308317in}}%
\pgfpathlineto{\pgfqpoint{5.608724in}{2.394532in}}%
\pgfpathlineto{\pgfqpoint{5.665657in}{2.483255in}}%
\pgfpathlineto{\pgfqpoint{5.722590in}{2.574487in}}%
\pgfpathlineto{\pgfqpoint{5.779522in}{2.668229in}}%
\pgfpathlineto{\pgfqpoint{5.836455in}{2.764480in}}%
\pgfpathlineto{\pgfqpoint{5.893388in}{2.863240in}}%
\pgfpathlineto{\pgfqpoint{5.950321in}{2.964510in}}%
\pgfpathlineto{\pgfqpoint{6.007254in}{3.068288in}}%
\pgfpathlineto{\pgfqpoint{6.064187in}{3.174576in}}%
\pgfpathlineto{\pgfqpoint{6.121120in}{3.283373in}}%
\pgfpathlineto{\pgfqpoint{6.178053in}{3.394679in}}%
\pgfpathlineto{\pgfqpoint{6.234986in}{3.508495in}}%
\pgfpathlineto{\pgfqpoint{6.291919in}{3.624820in}}%
\pgfpathlineto{\pgfqpoint{6.348852in}{3.743654in}}%
\pgfpathlineto{\pgfqpoint{6.405785in}{3.864997in}}%
\pgfpathlineto{\pgfqpoint{6.462718in}{3.988849in}}%
\pgfpathlineto{\pgfqpoint{6.519651in}{4.115211in}}%
\pgfpathlineto{\pgfqpoint{6.576584in}{4.244081in}}%
\pgfpathlineto{\pgfqpoint{6.633517in}{4.375462in}}%
\pgfpathlineto{\pgfqpoint{6.690450in}{4.509351in}}%
\pgfpathlineto{\pgfqpoint{6.747383in}{4.645749in}}%
\pgfpathlineto{\pgfqpoint{6.804316in}{4.784657in}}%
\pgfpathlineto{\pgfqpoint{6.861249in}{4.926074in}}%
\pgfpathlineto{\pgfqpoint{6.918182in}{5.070000in}}%
\pgfusepath{stroke}%
\end{pgfscope}%
\begin{pgfscope}%
\pgfsetrectcap%
\pgfsetmiterjoin%
\pgfsetlinewidth{0.803000pt}%
\definecolor{currentstroke}{rgb}{0.000000,0.000000,0.000000}%
\pgfsetstrokecolor{currentstroke}%
\pgfsetdash{}{0pt}%
\pgfpathmoveto{\pgfqpoint{1.000000in}{0.660000in}}%
\pgfpathlineto{\pgfqpoint{1.000000in}{5.280000in}}%
\pgfusepath{stroke}%
\end{pgfscope}%
\begin{pgfscope}%
\pgfsetrectcap%
\pgfsetmiterjoin%
\pgfsetlinewidth{0.803000pt}%
\definecolor{currentstroke}{rgb}{0.000000,0.000000,0.000000}%
\pgfsetstrokecolor{currentstroke}%
\pgfsetdash{}{0pt}%
\pgfpathmoveto{\pgfqpoint{7.200000in}{0.660000in}}%
\pgfpathlineto{\pgfqpoint{7.200000in}{5.280000in}}%
\pgfusepath{stroke}%
\end{pgfscope}%
\begin{pgfscope}%
\pgfsetrectcap%
\pgfsetmiterjoin%
\pgfsetlinewidth{0.803000pt}%
\definecolor{currentstroke}{rgb}{0.000000,0.000000,0.000000}%
\pgfsetstrokecolor{currentstroke}%
\pgfsetdash{}{0pt}%
\pgfpathmoveto{\pgfqpoint{1.000000in}{0.660000in}}%
\pgfpathlineto{\pgfqpoint{7.200000in}{0.660000in}}%
\pgfusepath{stroke}%
\end{pgfscope}%
\begin{pgfscope}%
\pgfsetrectcap%
\pgfsetmiterjoin%
\pgfsetlinewidth{0.803000pt}%
\definecolor{currentstroke}{rgb}{0.000000,0.000000,0.000000}%
\pgfsetstrokecolor{currentstroke}%
\pgfsetdash{}{0pt}%
\pgfpathmoveto{\pgfqpoint{1.000000in}{5.280000in}}%
\pgfpathlineto{\pgfqpoint{7.200000in}{5.280000in}}%
\pgfusepath{stroke}%
\end{pgfscope}%
\begin{pgfscope}%
\pgfsetbuttcap%
\pgfsetmiterjoin%
\definecolor{currentfill}{rgb}{1.000000,1.000000,1.000000}%
\pgfsetfillcolor{currentfill}%
\pgfsetfillopacity{0.800000}%
\pgfsetlinewidth{1.003750pt}%
\definecolor{currentstroke}{rgb}{0.800000,0.800000,0.800000}%
\pgfsetstrokecolor{currentstroke}%
\pgfsetstrokeopacity{0.800000}%
\pgfsetdash{}{0pt}%
\pgfpathmoveto{\pgfqpoint{1.155556in}{4.156563in}}%
\pgfpathlineto{\pgfqpoint{4.397049in}{4.156563in}}%
\pgfpathquadraticcurveto{\pgfqpoint{4.441493in}{4.156563in}}{\pgfqpoint{4.441493in}{4.201008in}}%
\pgfpathlineto{\pgfqpoint{4.441493in}{5.124444in}}%
\pgfpathquadraticcurveto{\pgfqpoint{4.441493in}{5.168889in}}{\pgfqpoint{4.397049in}{5.168889in}}%
\pgfpathlineto{\pgfqpoint{1.155556in}{5.168889in}}%
\pgfpathquadraticcurveto{\pgfqpoint{1.111111in}{5.168889in}}{\pgfqpoint{1.111111in}{5.124444in}}%
\pgfpathlineto{\pgfqpoint{1.111111in}{4.201008in}}%
\pgfpathquadraticcurveto{\pgfqpoint{1.111111in}{4.156563in}}{\pgfqpoint{1.155556in}{4.156563in}}%
\pgfpathlineto{\pgfqpoint{1.155556in}{4.156563in}}%
\pgfpathclose%
\pgfusepath{stroke,fill}%
\end{pgfscope}%
\begin{pgfscope}%
\pgfsetbuttcap%
\pgfsetroundjoin%
\definecolor{currentfill}{rgb}{0.000000,0.000000,0.545098}%
\pgfsetfillcolor{currentfill}%
\pgfsetlinewidth{1.505625pt}%
\definecolor{currentstroke}{rgb}{0.000000,0.000000,0.545098}%
\pgfsetstrokecolor{currentstroke}%
\pgfsetdash{}{0pt}%
\pgfsys@defobject{currentmarker}{\pgfqpoint{-0.041667in}{-0.041667in}}{\pgfqpoint{0.041667in}{0.041667in}}{%
\pgfpathmoveto{\pgfqpoint{-0.041667in}{-0.041667in}}%
\pgfpathlineto{\pgfqpoint{0.041667in}{0.041667in}}%
\pgfpathmoveto{\pgfqpoint{-0.041667in}{0.041667in}}%
\pgfpathlineto{\pgfqpoint{0.041667in}{-0.041667in}}%
\pgfusepath{stroke,fill}%
}%
\begin{pgfscope}%
\pgfsys@transformshift{1.422222in}{4.982517in}%
\pgfsys@useobject{currentmarker}{}%
\end{pgfscope}%
\end{pgfscope}%
\begin{pgfscope}%
\definecolor{textcolor}{rgb}{0.000000,0.000000,0.000000}%
\pgfsetstrokecolor{textcolor}%
\pgfsetfillcolor{textcolor}%
\pgftext[x=1.822222in,y=4.924184in,left,base]{\color{textcolor}{\rmfamily\fontsize{16.000000}{19.200000}\selectfont\catcode`\^=\active\def^{\ifmmode\sp\else\^{}\fi}\catcode`\%=\active\def%{\%}Raie de OH en 15391.057 Å}}%
\end{pgfscope}%
\begin{pgfscope}%
\pgfsetrectcap%
\pgfsetroundjoin%
\pgfsetlinewidth{1.505625pt}%
\definecolor{currentstroke}{rgb}{0.827451,0.827451,0.827451}%
\pgfsetstrokecolor{currentstroke}%
\pgfsetdash{}{0pt}%
\pgfpathmoveto{\pgfqpoint{1.200000in}{4.686901in}}%
\pgfpathlineto{\pgfqpoint{1.422222in}{4.686901in}}%
\pgfpathlineto{\pgfqpoint{1.644444in}{4.686901in}}%
\pgfusepath{stroke}%
\end{pgfscope}%
\begin{pgfscope}%
\definecolor{textcolor}{rgb}{0.000000,0.000000,0.000000}%
\pgfsetstrokecolor{textcolor}%
\pgfsetfillcolor{textcolor}%
\pgftext[x=1.822222in,y=4.609123in,left,base]{\color{textcolor}{\rmfamily\fontsize{16.000000}{19.200000}\selectfont\catcode`\^=\active\def^{\ifmmode\sp\else\^{}\fi}\catcode`\%=\active\def%{\%}Ajustement quadratique}}%
\end{pgfscope}%
\begin{pgfscope}%
\pgfsetbuttcap%
\pgfsetroundjoin%
\definecolor{currentfill}{rgb}{1.000000,0.000000,0.000000}%
\pgfsetfillcolor{currentfill}%
\pgfsetlinewidth{1.505625pt}%
\definecolor{currentstroke}{rgb}{1.000000,0.000000,0.000000}%
\pgfsetstrokecolor{currentstroke}%
\pgfsetdash{}{0pt}%
\pgfsys@defobject{currentmarker}{\pgfqpoint{-0.041667in}{-0.041667in}}{\pgfqpoint{0.041667in}{0.041667in}}{%
\pgfpathmoveto{\pgfqpoint{-0.041667in}{-0.041667in}}%
\pgfpathlineto{\pgfqpoint{0.041667in}{0.041667in}}%
\pgfpathmoveto{\pgfqpoint{-0.041667in}{0.041667in}}%
\pgfpathlineto{\pgfqpoint{0.041667in}{-0.041667in}}%
\pgfusepath{stroke,fill}%
}%
\begin{pgfscope}%
\pgfsys@transformshift{1.422222in}{4.352179in}%
\pgfsys@useobject{currentmarker}{}%
\end{pgfscope}%
\end{pgfscope}%
\begin{pgfscope}%
\definecolor{textcolor}{rgb}{0.000000,0.000000,0.000000}%
\pgfsetstrokecolor{textcolor}%
\pgfsetfillcolor{textcolor}%
\pgftext[x=1.822222in,y=4.293846in,left,base]{\color{textcolor}{\rmfamily\fontsize{16.000000}{19.200000}\selectfont\catcode`\^=\active\def^{\ifmmode\sp\else\^{}\fi}\catcode`\%=\active\def%{\%}Minimum en 8.60}}%
\end{pgfscope}%
\end{pgfpicture}%
\makeatother%
\endgroup%
}  % Insérer le fichier PGF du premier graphique
%   \end{subfigure}
%   \hfill
%   % Deuxième image (graphique 2)
%   \begin{subfigure}[b]{0.45\textwidth}
%       \centering
%       \scalebox{0.45}{%% Creator: Matplotlib, PGF backend
%%
%% To include the figure in your LaTeX document, write
%%   \input{<filename>.pgf}
%%
%% Make sure the required packages are loaded in your preamble
%%   \usepackage{pgf}
%%
%% Also ensure that all the required font packages are loaded; for instance,
%% the lmodern package is sometimes necessary when using math font.
%%   \usepackage{lmodern}
%%
%% Figures using additional raster images can only be included by \input if
%% they are in the same directory as the main LaTeX file. For loading figures
%% from other directories you can use the `import` package
%%   \usepackage{import}
%%
%% and then include the figures with
%%   \import{<path to file>}{<filename>.pgf}
%%
%% Matplotlib used the following preamble
%%   \def\mathdefault#1{#1}
%%   \everymath=\expandafter{\the\everymath\displaystyle}
%%   
%%   \ifdefined\pdftexversion\else  % non-pdftex case.
%%     \usepackage{fontspec}
%%     \setmainfont{Times.ttc}[Path=\detokenize{/System/Library/Fonts/}]
%%     \setsansfont{DejaVuSans.ttf}[Path=\detokenize{/Users/margauxvandererven/anaconda3/lib/python3.11/site-packages/matplotlib/mpl-data/fonts/ttf/}]
%%     \setmonofont{DejaVuSansMono.ttf}[Path=\detokenize{/Users/margauxvandererven/anaconda3/lib/python3.11/site-packages/matplotlib/mpl-data/fonts/ttf/}]
%%   \fi
%%   \makeatletter\@ifpackageloaded{underscore}{}{\usepackage[strings]{underscore}}\makeatother
%%
\begingroup%
\makeatletter%
\begin{pgfpicture}%
\pgfpathrectangle{\pgfpointorigin}{\pgfqpoint{8.000000in}{6.000000in}}%
\pgfusepath{use as bounding box, clip}%
\begin{pgfscope}%
\pgfsetbuttcap%
\pgfsetmiterjoin%
\definecolor{currentfill}{rgb}{1.000000,1.000000,1.000000}%
\pgfsetfillcolor{currentfill}%
\pgfsetlinewidth{0.000000pt}%
\definecolor{currentstroke}{rgb}{1.000000,1.000000,1.000000}%
\pgfsetstrokecolor{currentstroke}%
\pgfsetdash{}{0pt}%
\pgfpathmoveto{\pgfqpoint{0.000000in}{0.000000in}}%
\pgfpathlineto{\pgfqpoint{8.000000in}{0.000000in}}%
\pgfpathlineto{\pgfqpoint{8.000000in}{6.000000in}}%
\pgfpathlineto{\pgfqpoint{0.000000in}{6.000000in}}%
\pgfpathlineto{\pgfqpoint{0.000000in}{0.000000in}}%
\pgfpathclose%
\pgfusepath{fill}%
\end{pgfscope}%
\begin{pgfscope}%
\pgfsetbuttcap%
\pgfsetmiterjoin%
\definecolor{currentfill}{rgb}{1.000000,1.000000,1.000000}%
\pgfsetfillcolor{currentfill}%
\pgfsetlinewidth{0.000000pt}%
\definecolor{currentstroke}{rgb}{0.000000,0.000000,0.000000}%
\pgfsetstrokecolor{currentstroke}%
\pgfsetstrokeopacity{0.000000}%
\pgfsetdash{}{0pt}%
\pgfpathmoveto{\pgfqpoint{1.000000in}{0.660000in}}%
\pgfpathlineto{\pgfqpoint{7.200000in}{0.660000in}}%
\pgfpathlineto{\pgfqpoint{7.200000in}{5.280000in}}%
\pgfpathlineto{\pgfqpoint{1.000000in}{5.280000in}}%
\pgfpathlineto{\pgfqpoint{1.000000in}{0.660000in}}%
\pgfpathclose%
\pgfusepath{fill}%
\end{pgfscope}%
\begin{pgfscope}%
\pgfpathrectangle{\pgfqpoint{1.000000in}{0.660000in}}{\pgfqpoint{6.200000in}{4.620000in}}%
\pgfusepath{clip}%
\pgfsetbuttcap%
\pgfsetroundjoin%
\definecolor{currentfill}{rgb}{0.000000,0.000000,0.545098}%
\pgfsetfillcolor{currentfill}%
\pgfsetlinewidth{1.505625pt}%
\definecolor{currentstroke}{rgb}{0.000000,0.000000,0.545098}%
\pgfsetstrokecolor{currentstroke}%
\pgfsetdash{}{0pt}%
\pgfsys@defobject{currentmarker}{\pgfqpoint{-0.041667in}{-0.041667in}}{\pgfqpoint{0.041667in}{0.041667in}}{%
\pgfpathmoveto{\pgfqpoint{-0.041667in}{-0.041667in}}%
\pgfpathlineto{\pgfqpoint{0.041667in}{0.041667in}}%
\pgfpathmoveto{\pgfqpoint{-0.041667in}{0.041667in}}%
\pgfpathlineto{\pgfqpoint{0.041667in}{-0.041667in}}%
\pgfusepath{stroke,fill}%
}%
\begin{pgfscope}%
\pgfsys@transformshift{1.281818in}{3.010460in}%
\pgfsys@useobject{currentmarker}{}%
\end{pgfscope}%
\begin{pgfscope}%
\pgfsys@transformshift{2.534343in}{1.300169in}%
\pgfsys@useobject{currentmarker}{}%
\end{pgfscope}%
\begin{pgfscope}%
\pgfsys@transformshift{3.786869in}{0.871443in}%
\pgfsys@useobject{currentmarker}{}%
\end{pgfscope}%
\begin{pgfscope}%
\pgfsys@transformshift{5.039394in}{1.678055in}%
\pgfsys@useobject{currentmarker}{}%
\end{pgfscope}%
\begin{pgfscope}%
\pgfsys@transformshift{6.918182in}{5.058638in}%
\pgfsys@useobject{currentmarker}{}%
\end{pgfscope}%
\end{pgfscope}%
\begin{pgfscope}%
\pgfpathrectangle{\pgfqpoint{1.000000in}{0.660000in}}{\pgfqpoint{6.200000in}{4.620000in}}%
\pgfusepath{clip}%
\pgfsetbuttcap%
\pgfsetroundjoin%
\definecolor{currentfill}{rgb}{1.000000,0.000000,0.000000}%
\pgfsetfillcolor{currentfill}%
\pgfsetlinewidth{1.505625pt}%
\definecolor{currentstroke}{rgb}{1.000000,0.000000,0.000000}%
\pgfsetstrokecolor{currentstroke}%
\pgfsetdash{}{0pt}%
\pgfsys@defobject{currentmarker}{\pgfqpoint{-0.041667in}{-0.041667in}}{\pgfqpoint{0.041667in}{0.041667in}}{%
\pgfpathmoveto{\pgfqpoint{-0.041667in}{-0.041667in}}%
\pgfpathlineto{\pgfqpoint{0.041667in}{0.041667in}}%
\pgfpathmoveto{\pgfqpoint{-0.041667in}{0.041667in}}%
\pgfpathlineto{\pgfqpoint{0.041667in}{-0.041667in}}%
\pgfusepath{stroke,fill}%
}%
\begin{pgfscope}%
\pgfsys@transformshift{3.624102in}{0.870000in}%
\pgfsys@useobject{currentmarker}{}%
\end{pgfscope}%
\end{pgfscope}%
\begin{pgfscope}%
\pgfsetbuttcap%
\pgfsetroundjoin%
\definecolor{currentfill}{rgb}{0.000000,0.000000,0.000000}%
\pgfsetfillcolor{currentfill}%
\pgfsetlinewidth{0.803000pt}%
\definecolor{currentstroke}{rgb}{0.000000,0.000000,0.000000}%
\pgfsetstrokecolor{currentstroke}%
\pgfsetdash{}{0pt}%
\pgfsys@defobject{currentmarker}{\pgfqpoint{0.000000in}{0.000000in}}{\pgfqpoint{0.000000in}{0.138889in}}{%
\pgfpathmoveto{\pgfqpoint{0.000000in}{0.000000in}}%
\pgfpathlineto{\pgfqpoint{0.000000in}{0.138889in}}%
\pgfusepath{stroke,fill}%
}%
\begin{pgfscope}%
\pgfsys@transformshift{1.281818in}{0.660000in}%
\pgfsys@useobject{currentmarker}{}%
\end{pgfscope}%
\end{pgfscope}%
\begin{pgfscope}%
\pgfsetbuttcap%
\pgfsetroundjoin%
\definecolor{currentfill}{rgb}{0.000000,0.000000,0.000000}%
\pgfsetfillcolor{currentfill}%
\pgfsetlinewidth{0.803000pt}%
\definecolor{currentstroke}{rgb}{0.000000,0.000000,0.000000}%
\pgfsetstrokecolor{currentstroke}%
\pgfsetdash{}{0pt}%
\pgfsys@defobject{currentmarker}{\pgfqpoint{0.000000in}{-0.138889in}}{\pgfqpoint{0.000000in}{0.000000in}}{%
\pgfpathmoveto{\pgfqpoint{0.000000in}{0.000000in}}%
\pgfpathlineto{\pgfqpoint{0.000000in}{-0.138889in}}%
\pgfusepath{stroke,fill}%
}%
\begin{pgfscope}%
\pgfsys@transformshift{1.281818in}{5.280000in}%
\pgfsys@useobject{currentmarker}{}%
\end{pgfscope}%
\end{pgfscope}%
\begin{pgfscope}%
\definecolor{textcolor}{rgb}{0.000000,0.000000,0.000000}%
\pgfsetstrokecolor{textcolor}%
\pgfsetfillcolor{textcolor}%
\pgftext[x=1.281818in,y=0.611389in,,top]{\color{textcolor}{\rmfamily\fontsize{16.000000}{19.200000}\selectfont\catcode`\^=\active\def^{\ifmmode\sp\else\^{}\fi}\catcode`\%=\active\def%{\%}8.56}}%
\end{pgfscope}%
\begin{pgfscope}%
\pgfsetbuttcap%
\pgfsetroundjoin%
\definecolor{currentfill}{rgb}{0.000000,0.000000,0.000000}%
\pgfsetfillcolor{currentfill}%
\pgfsetlinewidth{0.803000pt}%
\definecolor{currentstroke}{rgb}{0.000000,0.000000,0.000000}%
\pgfsetstrokecolor{currentstroke}%
\pgfsetdash{}{0pt}%
\pgfsys@defobject{currentmarker}{\pgfqpoint{0.000000in}{0.000000in}}{\pgfqpoint{0.000000in}{0.138889in}}{%
\pgfpathmoveto{\pgfqpoint{0.000000in}{0.000000in}}%
\pgfpathlineto{\pgfqpoint{0.000000in}{0.138889in}}%
\pgfusepath{stroke,fill}%
}%
\begin{pgfscope}%
\pgfsys@transformshift{2.534343in}{0.660000in}%
\pgfsys@useobject{currentmarker}{}%
\end{pgfscope}%
\end{pgfscope}%
\begin{pgfscope}%
\pgfsetbuttcap%
\pgfsetroundjoin%
\definecolor{currentfill}{rgb}{0.000000,0.000000,0.000000}%
\pgfsetfillcolor{currentfill}%
\pgfsetlinewidth{0.803000pt}%
\definecolor{currentstroke}{rgb}{0.000000,0.000000,0.000000}%
\pgfsetstrokecolor{currentstroke}%
\pgfsetdash{}{0pt}%
\pgfsys@defobject{currentmarker}{\pgfqpoint{0.000000in}{-0.138889in}}{\pgfqpoint{0.000000in}{0.000000in}}{%
\pgfpathmoveto{\pgfqpoint{0.000000in}{0.000000in}}%
\pgfpathlineto{\pgfqpoint{0.000000in}{-0.138889in}}%
\pgfusepath{stroke,fill}%
}%
\begin{pgfscope}%
\pgfsys@transformshift{2.534343in}{5.280000in}%
\pgfsys@useobject{currentmarker}{}%
\end{pgfscope}%
\end{pgfscope}%
\begin{pgfscope}%
\definecolor{textcolor}{rgb}{0.000000,0.000000,0.000000}%
\pgfsetstrokecolor{textcolor}%
\pgfsetfillcolor{textcolor}%
\pgftext[x=2.534343in,y=0.611389in,,top]{\color{textcolor}{\rmfamily\fontsize{16.000000}{19.200000}\selectfont\catcode`\^=\active\def^{\ifmmode\sp\else\^{}\fi}\catcode`\%=\active\def%{\%}8.58}}%
\end{pgfscope}%
\begin{pgfscope}%
\pgfsetbuttcap%
\pgfsetroundjoin%
\definecolor{currentfill}{rgb}{0.000000,0.000000,0.000000}%
\pgfsetfillcolor{currentfill}%
\pgfsetlinewidth{0.803000pt}%
\definecolor{currentstroke}{rgb}{0.000000,0.000000,0.000000}%
\pgfsetstrokecolor{currentstroke}%
\pgfsetdash{}{0pt}%
\pgfsys@defobject{currentmarker}{\pgfqpoint{0.000000in}{0.000000in}}{\pgfqpoint{0.000000in}{0.138889in}}{%
\pgfpathmoveto{\pgfqpoint{0.000000in}{0.000000in}}%
\pgfpathlineto{\pgfqpoint{0.000000in}{0.138889in}}%
\pgfusepath{stroke,fill}%
}%
\begin{pgfscope}%
\pgfsys@transformshift{3.786869in}{0.660000in}%
\pgfsys@useobject{currentmarker}{}%
\end{pgfscope}%
\end{pgfscope}%
\begin{pgfscope}%
\pgfsetbuttcap%
\pgfsetroundjoin%
\definecolor{currentfill}{rgb}{0.000000,0.000000,0.000000}%
\pgfsetfillcolor{currentfill}%
\pgfsetlinewidth{0.803000pt}%
\definecolor{currentstroke}{rgb}{0.000000,0.000000,0.000000}%
\pgfsetstrokecolor{currentstroke}%
\pgfsetdash{}{0pt}%
\pgfsys@defobject{currentmarker}{\pgfqpoint{0.000000in}{-0.138889in}}{\pgfqpoint{0.000000in}{0.000000in}}{%
\pgfpathmoveto{\pgfqpoint{0.000000in}{0.000000in}}%
\pgfpathlineto{\pgfqpoint{0.000000in}{-0.138889in}}%
\pgfusepath{stroke,fill}%
}%
\begin{pgfscope}%
\pgfsys@transformshift{3.786869in}{5.280000in}%
\pgfsys@useobject{currentmarker}{}%
\end{pgfscope}%
\end{pgfscope}%
\begin{pgfscope}%
\definecolor{textcolor}{rgb}{0.000000,0.000000,0.000000}%
\pgfsetstrokecolor{textcolor}%
\pgfsetfillcolor{textcolor}%
\pgftext[x=3.786869in,y=0.611389in,,top]{\color{textcolor}{\rmfamily\fontsize{16.000000}{19.200000}\selectfont\catcode`\^=\active\def^{\ifmmode\sp\else\^{}\fi}\catcode`\%=\active\def%{\%}8.60}}%
\end{pgfscope}%
\begin{pgfscope}%
\pgfsetbuttcap%
\pgfsetroundjoin%
\definecolor{currentfill}{rgb}{0.000000,0.000000,0.000000}%
\pgfsetfillcolor{currentfill}%
\pgfsetlinewidth{0.803000pt}%
\definecolor{currentstroke}{rgb}{0.000000,0.000000,0.000000}%
\pgfsetstrokecolor{currentstroke}%
\pgfsetdash{}{0pt}%
\pgfsys@defobject{currentmarker}{\pgfqpoint{0.000000in}{0.000000in}}{\pgfqpoint{0.000000in}{0.138889in}}{%
\pgfpathmoveto{\pgfqpoint{0.000000in}{0.000000in}}%
\pgfpathlineto{\pgfqpoint{0.000000in}{0.138889in}}%
\pgfusepath{stroke,fill}%
}%
\begin{pgfscope}%
\pgfsys@transformshift{5.039394in}{0.660000in}%
\pgfsys@useobject{currentmarker}{}%
\end{pgfscope}%
\end{pgfscope}%
\begin{pgfscope}%
\pgfsetbuttcap%
\pgfsetroundjoin%
\definecolor{currentfill}{rgb}{0.000000,0.000000,0.000000}%
\pgfsetfillcolor{currentfill}%
\pgfsetlinewidth{0.803000pt}%
\definecolor{currentstroke}{rgb}{0.000000,0.000000,0.000000}%
\pgfsetstrokecolor{currentstroke}%
\pgfsetdash{}{0pt}%
\pgfsys@defobject{currentmarker}{\pgfqpoint{0.000000in}{-0.138889in}}{\pgfqpoint{0.000000in}{0.000000in}}{%
\pgfpathmoveto{\pgfqpoint{0.000000in}{0.000000in}}%
\pgfpathlineto{\pgfqpoint{0.000000in}{-0.138889in}}%
\pgfusepath{stroke,fill}%
}%
\begin{pgfscope}%
\pgfsys@transformshift{5.039394in}{5.280000in}%
\pgfsys@useobject{currentmarker}{}%
\end{pgfscope}%
\end{pgfscope}%
\begin{pgfscope}%
\definecolor{textcolor}{rgb}{0.000000,0.000000,0.000000}%
\pgfsetstrokecolor{textcolor}%
\pgfsetfillcolor{textcolor}%
\pgftext[x=5.039394in,y=0.611389in,,top]{\color{textcolor}{\rmfamily\fontsize{16.000000}{19.200000}\selectfont\catcode`\^=\active\def^{\ifmmode\sp\else\^{}\fi}\catcode`\%=\active\def%{\%}8.62}}%
\end{pgfscope}%
\begin{pgfscope}%
\pgfsetbuttcap%
\pgfsetroundjoin%
\definecolor{currentfill}{rgb}{0.000000,0.000000,0.000000}%
\pgfsetfillcolor{currentfill}%
\pgfsetlinewidth{0.803000pt}%
\definecolor{currentstroke}{rgb}{0.000000,0.000000,0.000000}%
\pgfsetstrokecolor{currentstroke}%
\pgfsetdash{}{0pt}%
\pgfsys@defobject{currentmarker}{\pgfqpoint{0.000000in}{0.000000in}}{\pgfqpoint{0.000000in}{0.138889in}}{%
\pgfpathmoveto{\pgfqpoint{0.000000in}{0.000000in}}%
\pgfpathlineto{\pgfqpoint{0.000000in}{0.138889in}}%
\pgfusepath{stroke,fill}%
}%
\begin{pgfscope}%
\pgfsys@transformshift{6.291919in}{0.660000in}%
\pgfsys@useobject{currentmarker}{}%
\end{pgfscope}%
\end{pgfscope}%
\begin{pgfscope}%
\pgfsetbuttcap%
\pgfsetroundjoin%
\definecolor{currentfill}{rgb}{0.000000,0.000000,0.000000}%
\pgfsetfillcolor{currentfill}%
\pgfsetlinewidth{0.803000pt}%
\definecolor{currentstroke}{rgb}{0.000000,0.000000,0.000000}%
\pgfsetstrokecolor{currentstroke}%
\pgfsetdash{}{0pt}%
\pgfsys@defobject{currentmarker}{\pgfqpoint{0.000000in}{-0.138889in}}{\pgfqpoint{0.000000in}{0.000000in}}{%
\pgfpathmoveto{\pgfqpoint{0.000000in}{0.000000in}}%
\pgfpathlineto{\pgfqpoint{0.000000in}{-0.138889in}}%
\pgfusepath{stroke,fill}%
}%
\begin{pgfscope}%
\pgfsys@transformshift{6.291919in}{5.280000in}%
\pgfsys@useobject{currentmarker}{}%
\end{pgfscope}%
\end{pgfscope}%
\begin{pgfscope}%
\definecolor{textcolor}{rgb}{0.000000,0.000000,0.000000}%
\pgfsetstrokecolor{textcolor}%
\pgfsetfillcolor{textcolor}%
\pgftext[x=6.291919in,y=0.611389in,,top]{\color{textcolor}{\rmfamily\fontsize{16.000000}{19.200000}\selectfont\catcode`\^=\active\def^{\ifmmode\sp\else\^{}\fi}\catcode`\%=\active\def%{\%}8.64}}%
\end{pgfscope}%
\begin{pgfscope}%
\definecolor{textcolor}{rgb}{0.000000,0.000000,0.000000}%
\pgfsetstrokecolor{textcolor}%
\pgfsetfillcolor{textcolor}%
\pgftext[x=4.100000in,y=0.355096in,,top]{\color{textcolor}{\rmfamily\fontsize{16.000000}{19.200000}\selectfont\catcode`\^=\active\def^{\ifmmode\sp\else\^{}\fi}\catcode`\%=\active\def%{\%}$\log \epsilon_{\mathrm{O}}$}}%
\end{pgfscope}%
\begin{pgfscope}%
\pgfsetbuttcap%
\pgfsetroundjoin%
\definecolor{currentfill}{rgb}{0.000000,0.000000,0.000000}%
\pgfsetfillcolor{currentfill}%
\pgfsetlinewidth{0.803000pt}%
\definecolor{currentstroke}{rgb}{0.000000,0.000000,0.000000}%
\pgfsetstrokecolor{currentstroke}%
\pgfsetdash{}{0pt}%
\pgfsys@defobject{currentmarker}{\pgfqpoint{0.000000in}{0.000000in}}{\pgfqpoint{0.138889in}{0.000000in}}{%
\pgfpathmoveto{\pgfqpoint{0.000000in}{0.000000in}}%
\pgfpathlineto{\pgfqpoint{0.138889in}{0.000000in}}%
\pgfusepath{stroke,fill}%
}%
\begin{pgfscope}%
\pgfsys@transformshift{1.000000in}{1.442351in}%
\pgfsys@useobject{currentmarker}{}%
\end{pgfscope}%
\end{pgfscope}%
\begin{pgfscope}%
\pgfsetbuttcap%
\pgfsetroundjoin%
\definecolor{currentfill}{rgb}{0.000000,0.000000,0.000000}%
\pgfsetfillcolor{currentfill}%
\pgfsetlinewidth{0.803000pt}%
\definecolor{currentstroke}{rgb}{0.000000,0.000000,0.000000}%
\pgfsetstrokecolor{currentstroke}%
\pgfsetdash{}{0pt}%
\pgfsys@defobject{currentmarker}{\pgfqpoint{-0.138889in}{0.000000in}}{\pgfqpoint{-0.000000in}{0.000000in}}{%
\pgfpathmoveto{\pgfqpoint{-0.000000in}{0.000000in}}%
\pgfpathlineto{\pgfqpoint{-0.138889in}{0.000000in}}%
\pgfusepath{stroke,fill}%
}%
\begin{pgfscope}%
\pgfsys@transformshift{7.200000in}{1.442351in}%
\pgfsys@useobject{currentmarker}{}%
\end{pgfscope}%
\end{pgfscope}%
\begin{pgfscope}%
\definecolor{textcolor}{rgb}{0.000000,0.000000,0.000000}%
\pgfsetstrokecolor{textcolor}%
\pgfsetfillcolor{textcolor}%
\pgftext[x=0.562500in, y=1.366179in, left, base]{\color{textcolor}{\rmfamily\fontsize{16.000000}{19.200000}\selectfont\catcode`\^=\active\def^{\ifmmode\sp\else\^{}\fi}\catcode`\%=\active\def%{\%}0.02}}%
\end{pgfscope}%
\begin{pgfscope}%
\pgfsetbuttcap%
\pgfsetroundjoin%
\definecolor{currentfill}{rgb}{0.000000,0.000000,0.000000}%
\pgfsetfillcolor{currentfill}%
\pgfsetlinewidth{0.803000pt}%
\definecolor{currentstroke}{rgb}{0.000000,0.000000,0.000000}%
\pgfsetstrokecolor{currentstroke}%
\pgfsetdash{}{0pt}%
\pgfsys@defobject{currentmarker}{\pgfqpoint{0.000000in}{0.000000in}}{\pgfqpoint{0.138889in}{0.000000in}}{%
\pgfpathmoveto{\pgfqpoint{0.000000in}{0.000000in}}%
\pgfpathlineto{\pgfqpoint{0.138889in}{0.000000in}}%
\pgfusepath{stroke,fill}%
}%
\begin{pgfscope}%
\pgfsys@transformshift{1.000000in}{2.444869in}%
\pgfsys@useobject{currentmarker}{}%
\end{pgfscope}%
\end{pgfscope}%
\begin{pgfscope}%
\pgfsetbuttcap%
\pgfsetroundjoin%
\definecolor{currentfill}{rgb}{0.000000,0.000000,0.000000}%
\pgfsetfillcolor{currentfill}%
\pgfsetlinewidth{0.803000pt}%
\definecolor{currentstroke}{rgb}{0.000000,0.000000,0.000000}%
\pgfsetstrokecolor{currentstroke}%
\pgfsetdash{}{0pt}%
\pgfsys@defobject{currentmarker}{\pgfqpoint{-0.138889in}{0.000000in}}{\pgfqpoint{-0.000000in}{0.000000in}}{%
\pgfpathmoveto{\pgfqpoint{-0.000000in}{0.000000in}}%
\pgfpathlineto{\pgfqpoint{-0.138889in}{0.000000in}}%
\pgfusepath{stroke,fill}%
}%
\begin{pgfscope}%
\pgfsys@transformshift{7.200000in}{2.444869in}%
\pgfsys@useobject{currentmarker}{}%
\end{pgfscope}%
\end{pgfscope}%
\begin{pgfscope}%
\definecolor{textcolor}{rgb}{0.000000,0.000000,0.000000}%
\pgfsetstrokecolor{textcolor}%
\pgfsetfillcolor{textcolor}%
\pgftext[x=0.562500in, y=2.368697in, left, base]{\color{textcolor}{\rmfamily\fontsize{16.000000}{19.200000}\selectfont\catcode`\^=\active\def^{\ifmmode\sp\else\^{}\fi}\catcode`\%=\active\def%{\%}0.03}}%
\end{pgfscope}%
\begin{pgfscope}%
\pgfsetbuttcap%
\pgfsetroundjoin%
\definecolor{currentfill}{rgb}{0.000000,0.000000,0.000000}%
\pgfsetfillcolor{currentfill}%
\pgfsetlinewidth{0.803000pt}%
\definecolor{currentstroke}{rgb}{0.000000,0.000000,0.000000}%
\pgfsetstrokecolor{currentstroke}%
\pgfsetdash{}{0pt}%
\pgfsys@defobject{currentmarker}{\pgfqpoint{0.000000in}{0.000000in}}{\pgfqpoint{0.138889in}{0.000000in}}{%
\pgfpathmoveto{\pgfqpoint{0.000000in}{0.000000in}}%
\pgfpathlineto{\pgfqpoint{0.138889in}{0.000000in}}%
\pgfusepath{stroke,fill}%
}%
\begin{pgfscope}%
\pgfsys@transformshift{1.000000in}{3.447386in}%
\pgfsys@useobject{currentmarker}{}%
\end{pgfscope}%
\end{pgfscope}%
\begin{pgfscope}%
\pgfsetbuttcap%
\pgfsetroundjoin%
\definecolor{currentfill}{rgb}{0.000000,0.000000,0.000000}%
\pgfsetfillcolor{currentfill}%
\pgfsetlinewidth{0.803000pt}%
\definecolor{currentstroke}{rgb}{0.000000,0.000000,0.000000}%
\pgfsetstrokecolor{currentstroke}%
\pgfsetdash{}{0pt}%
\pgfsys@defobject{currentmarker}{\pgfqpoint{-0.138889in}{0.000000in}}{\pgfqpoint{-0.000000in}{0.000000in}}{%
\pgfpathmoveto{\pgfqpoint{-0.000000in}{0.000000in}}%
\pgfpathlineto{\pgfqpoint{-0.138889in}{0.000000in}}%
\pgfusepath{stroke,fill}%
}%
\begin{pgfscope}%
\pgfsys@transformshift{7.200000in}{3.447386in}%
\pgfsys@useobject{currentmarker}{}%
\end{pgfscope}%
\end{pgfscope}%
\begin{pgfscope}%
\definecolor{textcolor}{rgb}{0.000000,0.000000,0.000000}%
\pgfsetstrokecolor{textcolor}%
\pgfsetfillcolor{textcolor}%
\pgftext[x=0.562500in, y=3.371215in, left, base]{\color{textcolor}{\rmfamily\fontsize{16.000000}{19.200000}\selectfont\catcode`\^=\active\def^{\ifmmode\sp\else\^{}\fi}\catcode`\%=\active\def%{\%}0.04}}%
\end{pgfscope}%
\begin{pgfscope}%
\pgfsetbuttcap%
\pgfsetroundjoin%
\definecolor{currentfill}{rgb}{0.000000,0.000000,0.000000}%
\pgfsetfillcolor{currentfill}%
\pgfsetlinewidth{0.803000pt}%
\definecolor{currentstroke}{rgb}{0.000000,0.000000,0.000000}%
\pgfsetstrokecolor{currentstroke}%
\pgfsetdash{}{0pt}%
\pgfsys@defobject{currentmarker}{\pgfqpoint{0.000000in}{0.000000in}}{\pgfqpoint{0.138889in}{0.000000in}}{%
\pgfpathmoveto{\pgfqpoint{0.000000in}{0.000000in}}%
\pgfpathlineto{\pgfqpoint{0.138889in}{0.000000in}}%
\pgfusepath{stroke,fill}%
}%
\begin{pgfscope}%
\pgfsys@transformshift{1.000000in}{4.449904in}%
\pgfsys@useobject{currentmarker}{}%
\end{pgfscope}%
\end{pgfscope}%
\begin{pgfscope}%
\pgfsetbuttcap%
\pgfsetroundjoin%
\definecolor{currentfill}{rgb}{0.000000,0.000000,0.000000}%
\pgfsetfillcolor{currentfill}%
\pgfsetlinewidth{0.803000pt}%
\definecolor{currentstroke}{rgb}{0.000000,0.000000,0.000000}%
\pgfsetstrokecolor{currentstroke}%
\pgfsetdash{}{0pt}%
\pgfsys@defobject{currentmarker}{\pgfqpoint{-0.138889in}{0.000000in}}{\pgfqpoint{-0.000000in}{0.000000in}}{%
\pgfpathmoveto{\pgfqpoint{-0.000000in}{0.000000in}}%
\pgfpathlineto{\pgfqpoint{-0.138889in}{0.000000in}}%
\pgfusepath{stroke,fill}%
}%
\begin{pgfscope}%
\pgfsys@transformshift{7.200000in}{4.449904in}%
\pgfsys@useobject{currentmarker}{}%
\end{pgfscope}%
\end{pgfscope}%
\begin{pgfscope}%
\definecolor{textcolor}{rgb}{0.000000,0.000000,0.000000}%
\pgfsetstrokecolor{textcolor}%
\pgfsetfillcolor{textcolor}%
\pgftext[x=0.562500in, y=4.373732in, left, base]{\color{textcolor}{\rmfamily\fontsize{16.000000}{19.200000}\selectfont\catcode`\^=\active\def^{\ifmmode\sp\else\^{}\fi}\catcode`\%=\active\def%{\%}0.05}}%
\end{pgfscope}%
\begin{pgfscope}%
\definecolor{textcolor}{rgb}{0.000000,0.000000,0.000000}%
\pgfsetstrokecolor{textcolor}%
\pgfsetfillcolor{textcolor}%
\pgftext[x=0.506944in,y=2.970000in,,bottom,rotate=90.000000]{\color{textcolor}{\rmfamily\fontsize{16.000000}{19.200000}\selectfont\catcode`\^=\active\def^{\ifmmode\sp\else\^{}\fi}\catcode`\%=\active\def%{\%}$\chi^2$}}%
\end{pgfscope}%
\begin{pgfscope}%
\pgfpathrectangle{\pgfqpoint{1.000000in}{0.660000in}}{\pgfqpoint{6.200000in}{4.620000in}}%
\pgfusepath{clip}%
\pgfsetrectcap%
\pgfsetroundjoin%
\pgfsetlinewidth{1.505625pt}%
\definecolor{currentstroke}{rgb}{0.827451,0.827451,0.827451}%
\pgfsetstrokecolor{currentstroke}%
\pgfsetdash{}{0pt}%
\pgfpathmoveto{\pgfqpoint{1.281818in}{2.993539in}}%
\pgfpathlineto{\pgfqpoint{1.338751in}{2.891562in}}%
\pgfpathlineto{\pgfqpoint{1.395684in}{2.792094in}}%
\pgfpathlineto{\pgfqpoint{1.452617in}{2.695135in}}%
\pgfpathlineto{\pgfqpoint{1.509550in}{2.600685in}}%
\pgfpathlineto{\pgfqpoint{1.566483in}{2.508744in}}%
\pgfpathlineto{\pgfqpoint{1.623416in}{2.419313in}}%
\pgfpathlineto{\pgfqpoint{1.680349in}{2.332391in}}%
\pgfpathlineto{\pgfqpoint{1.737282in}{2.247978in}}%
\pgfpathlineto{\pgfqpoint{1.794215in}{2.166074in}}%
\pgfpathlineto{\pgfqpoint{1.851148in}{2.086680in}}%
\pgfpathlineto{\pgfqpoint{1.908081in}{2.009795in}}%
\pgfpathlineto{\pgfqpoint{1.965014in}{1.935419in}}%
\pgfpathlineto{\pgfqpoint{2.021947in}{1.863552in}}%
\pgfpathlineto{\pgfqpoint{2.078880in}{1.794194in}}%
\pgfpathlineto{\pgfqpoint{2.135813in}{1.727346in}}%
\pgfpathlineto{\pgfqpoint{2.192746in}{1.663007in}}%
\pgfpathlineto{\pgfqpoint{2.249679in}{1.601177in}}%
\pgfpathlineto{\pgfqpoint{2.306612in}{1.541856in}}%
\pgfpathlineto{\pgfqpoint{2.363545in}{1.485045in}}%
\pgfpathlineto{\pgfqpoint{2.420478in}{1.430743in}}%
\pgfpathlineto{\pgfqpoint{2.477410in}{1.378949in}}%
\pgfpathlineto{\pgfqpoint{2.534343in}{1.329666in}}%
\pgfpathlineto{\pgfqpoint{2.591276in}{1.282891in}}%
\pgfpathlineto{\pgfqpoint{2.648209in}{1.238626in}}%
\pgfpathlineto{\pgfqpoint{2.705142in}{1.196870in}}%
\pgfpathlineto{\pgfqpoint{2.762075in}{1.157623in}}%
\pgfpathlineto{\pgfqpoint{2.819008in}{1.120885in}}%
\pgfpathlineto{\pgfqpoint{2.875941in}{1.086656in}}%
\pgfpathlineto{\pgfqpoint{2.932874in}{1.054937in}}%
\pgfpathlineto{\pgfqpoint{2.989807in}{1.025727in}}%
\pgfpathlineto{\pgfqpoint{3.046740in}{0.999026in}}%
\pgfpathlineto{\pgfqpoint{3.103673in}{0.974835in}}%
\pgfpathlineto{\pgfqpoint{3.160606in}{0.953152in}}%
\pgfpathlineto{\pgfqpoint{3.217539in}{0.933979in}}%
\pgfpathlineto{\pgfqpoint{3.274472in}{0.917315in}}%
\pgfpathlineto{\pgfqpoint{3.331405in}{0.903160in}}%
\pgfpathlineto{\pgfqpoint{3.388338in}{0.891515in}}%
\pgfpathlineto{\pgfqpoint{3.445271in}{0.882379in}}%
\pgfpathlineto{\pgfqpoint{3.502204in}{0.875751in}}%
\pgfpathlineto{\pgfqpoint{3.559137in}{0.871634in}}%
\pgfpathlineto{\pgfqpoint{3.616070in}{0.870025in}}%
\pgfpathlineto{\pgfqpoint{3.673003in}{0.870926in}}%
\pgfpathlineto{\pgfqpoint{3.729936in}{0.874335in}}%
\pgfpathlineto{\pgfqpoint{3.786869in}{0.880254in}}%
\pgfpathlineto{\pgfqpoint{3.843802in}{0.888683in}}%
\pgfpathlineto{\pgfqpoint{3.900735in}{0.899620in}}%
\pgfpathlineto{\pgfqpoint{3.957668in}{0.913067in}}%
\pgfpathlineto{\pgfqpoint{4.014601in}{0.929023in}}%
\pgfpathlineto{\pgfqpoint{4.071534in}{0.947488in}}%
\pgfpathlineto{\pgfqpoint{4.128466in}{0.968462in}}%
\pgfpathlineto{\pgfqpoint{4.185399in}{0.991946in}}%
\pgfpathlineto{\pgfqpoint{4.242332in}{1.017939in}}%
\pgfpathlineto{\pgfqpoint{4.299265in}{1.046441in}}%
\pgfpathlineto{\pgfqpoint{4.356198in}{1.077452in}}%
\pgfpathlineto{\pgfqpoint{4.413131in}{1.110972in}}%
\pgfpathlineto{\pgfqpoint{4.470064in}{1.147002in}}%
\pgfpathlineto{\pgfqpoint{4.526997in}{1.185541in}}%
\pgfpathlineto{\pgfqpoint{4.583930in}{1.226589in}}%
\pgfpathlineto{\pgfqpoint{4.640863in}{1.270146in}}%
\pgfpathlineto{\pgfqpoint{4.697796in}{1.316213in}}%
\pgfpathlineto{\pgfqpoint{4.754729in}{1.364789in}}%
\pgfpathlineto{\pgfqpoint{4.811662in}{1.415874in}}%
\pgfpathlineto{\pgfqpoint{4.868595in}{1.469468in}}%
\pgfpathlineto{\pgfqpoint{4.925528in}{1.525571in}}%
\pgfpathlineto{\pgfqpoint{4.982461in}{1.584184in}}%
\pgfpathlineto{\pgfqpoint{5.039394in}{1.645306in}}%
\pgfpathlineto{\pgfqpoint{5.096327in}{1.708937in}}%
\pgfpathlineto{\pgfqpoint{5.153260in}{1.775077in}}%
\pgfpathlineto{\pgfqpoint{5.210193in}{1.843727in}}%
\pgfpathlineto{\pgfqpoint{5.267126in}{1.914885in}}%
\pgfpathlineto{\pgfqpoint{5.324059in}{1.988553in}}%
\pgfpathlineto{\pgfqpoint{5.380992in}{2.064731in}}%
\pgfpathlineto{\pgfqpoint{5.437925in}{2.143417in}}%
\pgfpathlineto{\pgfqpoint{5.494858in}{2.224613in}}%
\pgfpathlineto{\pgfqpoint{5.551791in}{2.308317in}}%
\pgfpathlineto{\pgfqpoint{5.608724in}{2.394532in}}%
\pgfpathlineto{\pgfqpoint{5.665657in}{2.483255in}}%
\pgfpathlineto{\pgfqpoint{5.722590in}{2.574487in}}%
\pgfpathlineto{\pgfqpoint{5.779522in}{2.668229in}}%
\pgfpathlineto{\pgfqpoint{5.836455in}{2.764480in}}%
\pgfpathlineto{\pgfqpoint{5.893388in}{2.863240in}}%
\pgfpathlineto{\pgfqpoint{5.950321in}{2.964510in}}%
\pgfpathlineto{\pgfqpoint{6.007254in}{3.068288in}}%
\pgfpathlineto{\pgfqpoint{6.064187in}{3.174576in}}%
\pgfpathlineto{\pgfqpoint{6.121120in}{3.283373in}}%
\pgfpathlineto{\pgfqpoint{6.178053in}{3.394679in}}%
\pgfpathlineto{\pgfqpoint{6.234986in}{3.508495in}}%
\pgfpathlineto{\pgfqpoint{6.291919in}{3.624820in}}%
\pgfpathlineto{\pgfqpoint{6.348852in}{3.743654in}}%
\pgfpathlineto{\pgfqpoint{6.405785in}{3.864997in}}%
\pgfpathlineto{\pgfqpoint{6.462718in}{3.988849in}}%
\pgfpathlineto{\pgfqpoint{6.519651in}{4.115211in}}%
\pgfpathlineto{\pgfqpoint{6.576584in}{4.244081in}}%
\pgfpathlineto{\pgfqpoint{6.633517in}{4.375462in}}%
\pgfpathlineto{\pgfqpoint{6.690450in}{4.509351in}}%
\pgfpathlineto{\pgfqpoint{6.747383in}{4.645749in}}%
\pgfpathlineto{\pgfqpoint{6.804316in}{4.784657in}}%
\pgfpathlineto{\pgfqpoint{6.861249in}{4.926074in}}%
\pgfpathlineto{\pgfqpoint{6.918182in}{5.070000in}}%
\pgfusepath{stroke}%
\end{pgfscope}%
\begin{pgfscope}%
\pgfsetrectcap%
\pgfsetmiterjoin%
\pgfsetlinewidth{0.803000pt}%
\definecolor{currentstroke}{rgb}{0.000000,0.000000,0.000000}%
\pgfsetstrokecolor{currentstroke}%
\pgfsetdash{}{0pt}%
\pgfpathmoveto{\pgfqpoint{1.000000in}{0.660000in}}%
\pgfpathlineto{\pgfqpoint{1.000000in}{5.280000in}}%
\pgfusepath{stroke}%
\end{pgfscope}%
\begin{pgfscope}%
\pgfsetrectcap%
\pgfsetmiterjoin%
\pgfsetlinewidth{0.803000pt}%
\definecolor{currentstroke}{rgb}{0.000000,0.000000,0.000000}%
\pgfsetstrokecolor{currentstroke}%
\pgfsetdash{}{0pt}%
\pgfpathmoveto{\pgfqpoint{7.200000in}{0.660000in}}%
\pgfpathlineto{\pgfqpoint{7.200000in}{5.280000in}}%
\pgfusepath{stroke}%
\end{pgfscope}%
\begin{pgfscope}%
\pgfsetrectcap%
\pgfsetmiterjoin%
\pgfsetlinewidth{0.803000pt}%
\definecolor{currentstroke}{rgb}{0.000000,0.000000,0.000000}%
\pgfsetstrokecolor{currentstroke}%
\pgfsetdash{}{0pt}%
\pgfpathmoveto{\pgfqpoint{1.000000in}{0.660000in}}%
\pgfpathlineto{\pgfqpoint{7.200000in}{0.660000in}}%
\pgfusepath{stroke}%
\end{pgfscope}%
\begin{pgfscope}%
\pgfsetrectcap%
\pgfsetmiterjoin%
\pgfsetlinewidth{0.803000pt}%
\definecolor{currentstroke}{rgb}{0.000000,0.000000,0.000000}%
\pgfsetstrokecolor{currentstroke}%
\pgfsetdash{}{0pt}%
\pgfpathmoveto{\pgfqpoint{1.000000in}{5.280000in}}%
\pgfpathlineto{\pgfqpoint{7.200000in}{5.280000in}}%
\pgfusepath{stroke}%
\end{pgfscope}%
\begin{pgfscope}%
\pgfsetbuttcap%
\pgfsetmiterjoin%
\definecolor{currentfill}{rgb}{1.000000,1.000000,1.000000}%
\pgfsetfillcolor{currentfill}%
\pgfsetfillopacity{0.800000}%
\pgfsetlinewidth{1.003750pt}%
\definecolor{currentstroke}{rgb}{0.800000,0.800000,0.800000}%
\pgfsetstrokecolor{currentstroke}%
\pgfsetstrokeopacity{0.800000}%
\pgfsetdash{}{0pt}%
\pgfpathmoveto{\pgfqpoint{1.155556in}{4.156563in}}%
\pgfpathlineto{\pgfqpoint{4.397049in}{4.156563in}}%
\pgfpathquadraticcurveto{\pgfqpoint{4.441493in}{4.156563in}}{\pgfqpoint{4.441493in}{4.201008in}}%
\pgfpathlineto{\pgfqpoint{4.441493in}{5.124444in}}%
\pgfpathquadraticcurveto{\pgfqpoint{4.441493in}{5.168889in}}{\pgfqpoint{4.397049in}{5.168889in}}%
\pgfpathlineto{\pgfqpoint{1.155556in}{5.168889in}}%
\pgfpathquadraticcurveto{\pgfqpoint{1.111111in}{5.168889in}}{\pgfqpoint{1.111111in}{5.124444in}}%
\pgfpathlineto{\pgfqpoint{1.111111in}{4.201008in}}%
\pgfpathquadraticcurveto{\pgfqpoint{1.111111in}{4.156563in}}{\pgfqpoint{1.155556in}{4.156563in}}%
\pgfpathlineto{\pgfqpoint{1.155556in}{4.156563in}}%
\pgfpathclose%
\pgfusepath{stroke,fill}%
\end{pgfscope}%
\begin{pgfscope}%
\pgfsetbuttcap%
\pgfsetroundjoin%
\definecolor{currentfill}{rgb}{0.000000,0.000000,0.545098}%
\pgfsetfillcolor{currentfill}%
\pgfsetlinewidth{1.505625pt}%
\definecolor{currentstroke}{rgb}{0.000000,0.000000,0.545098}%
\pgfsetstrokecolor{currentstroke}%
\pgfsetdash{}{0pt}%
\pgfsys@defobject{currentmarker}{\pgfqpoint{-0.041667in}{-0.041667in}}{\pgfqpoint{0.041667in}{0.041667in}}{%
\pgfpathmoveto{\pgfqpoint{-0.041667in}{-0.041667in}}%
\pgfpathlineto{\pgfqpoint{0.041667in}{0.041667in}}%
\pgfpathmoveto{\pgfqpoint{-0.041667in}{0.041667in}}%
\pgfpathlineto{\pgfqpoint{0.041667in}{-0.041667in}}%
\pgfusepath{stroke,fill}%
}%
\begin{pgfscope}%
\pgfsys@transformshift{1.422222in}{4.982517in}%
\pgfsys@useobject{currentmarker}{}%
\end{pgfscope}%
\end{pgfscope}%
\begin{pgfscope}%
\definecolor{textcolor}{rgb}{0.000000,0.000000,0.000000}%
\pgfsetstrokecolor{textcolor}%
\pgfsetfillcolor{textcolor}%
\pgftext[x=1.822222in,y=4.924184in,left,base]{\color{textcolor}{\rmfamily\fontsize{16.000000}{19.200000}\selectfont\catcode`\^=\active\def^{\ifmmode\sp\else\^{}\fi}\catcode`\%=\active\def%{\%}Raie de OH en 15391.057 Å}}%
\end{pgfscope}%
\begin{pgfscope}%
\pgfsetrectcap%
\pgfsetroundjoin%
\pgfsetlinewidth{1.505625pt}%
\definecolor{currentstroke}{rgb}{0.827451,0.827451,0.827451}%
\pgfsetstrokecolor{currentstroke}%
\pgfsetdash{}{0pt}%
\pgfpathmoveto{\pgfqpoint{1.200000in}{4.686901in}}%
\pgfpathlineto{\pgfqpoint{1.422222in}{4.686901in}}%
\pgfpathlineto{\pgfqpoint{1.644444in}{4.686901in}}%
\pgfusepath{stroke}%
\end{pgfscope}%
\begin{pgfscope}%
\definecolor{textcolor}{rgb}{0.000000,0.000000,0.000000}%
\pgfsetstrokecolor{textcolor}%
\pgfsetfillcolor{textcolor}%
\pgftext[x=1.822222in,y=4.609123in,left,base]{\color{textcolor}{\rmfamily\fontsize{16.000000}{19.200000}\selectfont\catcode`\^=\active\def^{\ifmmode\sp\else\^{}\fi}\catcode`\%=\active\def%{\%}Ajustement quadratique}}%
\end{pgfscope}%
\begin{pgfscope}%
\pgfsetbuttcap%
\pgfsetroundjoin%
\definecolor{currentfill}{rgb}{1.000000,0.000000,0.000000}%
\pgfsetfillcolor{currentfill}%
\pgfsetlinewidth{1.505625pt}%
\definecolor{currentstroke}{rgb}{1.000000,0.000000,0.000000}%
\pgfsetstrokecolor{currentstroke}%
\pgfsetdash{}{0pt}%
\pgfsys@defobject{currentmarker}{\pgfqpoint{-0.041667in}{-0.041667in}}{\pgfqpoint{0.041667in}{0.041667in}}{%
\pgfpathmoveto{\pgfqpoint{-0.041667in}{-0.041667in}}%
\pgfpathlineto{\pgfqpoint{0.041667in}{0.041667in}}%
\pgfpathmoveto{\pgfqpoint{-0.041667in}{0.041667in}}%
\pgfpathlineto{\pgfqpoint{0.041667in}{-0.041667in}}%
\pgfusepath{stroke,fill}%
}%
\begin{pgfscope}%
\pgfsys@transformshift{1.422222in}{4.352179in}%
\pgfsys@useobject{currentmarker}{}%
\end{pgfscope}%
\end{pgfscope}%
\begin{pgfscope}%
\definecolor{textcolor}{rgb}{0.000000,0.000000,0.000000}%
\pgfsetstrokecolor{textcolor}%
\pgfsetfillcolor{textcolor}%
\pgftext[x=1.822222in,y=4.293846in,left,base]{\color{textcolor}{\rmfamily\fontsize{16.000000}{19.200000}\selectfont\catcode`\^=\active\def^{\ifmmode\sp\else\^{}\fi}\catcode`\%=\active\def%{\%}Minimum en 8.60}}%
\end{pgfscope}%
\end{pgfpicture}%
\makeatother%
\endgroup%
} % Insérer le fichier PGF du deuxième graphique
%   \end{subfigure}
  
%   \vskip\baselineskip  % Espace vertical entre les lignes de graphiques

%   % Troisième image (graphique 3)
%   \begin{subfigure}[b]{0.45\textwidth}
%       \centering
%       \scalebox{0.45}{%% Creator: Matplotlib, PGF backend
%%
%% To include the figure in your LaTeX document, write
%%   \input{<filename>.pgf}
%%
%% Make sure the required packages are loaded in your preamble
%%   \usepackage{pgf}
%%
%% Also ensure that all the required font packages are loaded; for instance,
%% the lmodern package is sometimes necessary when using math font.
%%   \usepackage{lmodern}
%%
%% Figures using additional raster images can only be included by \input if
%% they are in the same directory as the main LaTeX file. For loading figures
%% from other directories you can use the `import` package
%%   \usepackage{import}
%%
%% and then include the figures with
%%   \import{<path to file>}{<filename>.pgf}
%%
%% Matplotlib used the following preamble
%%   \def\mathdefault#1{#1}
%%   \everymath=\expandafter{\the\everymath\displaystyle}
%%   
%%   \ifdefined\pdftexversion\else  % non-pdftex case.
%%     \usepackage{fontspec}
%%     \setmainfont{Times.ttc}[Path=\detokenize{/System/Library/Fonts/}]
%%     \setsansfont{DejaVuSans.ttf}[Path=\detokenize{/Users/margauxvandererven/anaconda3/lib/python3.11/site-packages/matplotlib/mpl-data/fonts/ttf/}]
%%     \setmonofont{DejaVuSansMono.ttf}[Path=\detokenize{/Users/margauxvandererven/anaconda3/lib/python3.11/site-packages/matplotlib/mpl-data/fonts/ttf/}]
%%   \fi
%%   \makeatletter\@ifpackageloaded{underscore}{}{\usepackage[strings]{underscore}}\makeatother
%%
\begingroup%
\makeatletter%
\begin{pgfpicture}%
\pgfpathrectangle{\pgfpointorigin}{\pgfqpoint{8.000000in}{6.000000in}}%
\pgfusepath{use as bounding box, clip}%
\begin{pgfscope}%
\pgfsetbuttcap%
\pgfsetmiterjoin%
\definecolor{currentfill}{rgb}{1.000000,1.000000,1.000000}%
\pgfsetfillcolor{currentfill}%
\pgfsetlinewidth{0.000000pt}%
\definecolor{currentstroke}{rgb}{1.000000,1.000000,1.000000}%
\pgfsetstrokecolor{currentstroke}%
\pgfsetdash{}{0pt}%
\pgfpathmoveto{\pgfqpoint{0.000000in}{0.000000in}}%
\pgfpathlineto{\pgfqpoint{8.000000in}{0.000000in}}%
\pgfpathlineto{\pgfqpoint{8.000000in}{6.000000in}}%
\pgfpathlineto{\pgfqpoint{0.000000in}{6.000000in}}%
\pgfpathlineto{\pgfqpoint{0.000000in}{0.000000in}}%
\pgfpathclose%
\pgfusepath{fill}%
\end{pgfscope}%
\begin{pgfscope}%
\pgfsetbuttcap%
\pgfsetmiterjoin%
\definecolor{currentfill}{rgb}{1.000000,1.000000,1.000000}%
\pgfsetfillcolor{currentfill}%
\pgfsetlinewidth{0.000000pt}%
\definecolor{currentstroke}{rgb}{0.000000,0.000000,0.000000}%
\pgfsetstrokecolor{currentstroke}%
\pgfsetstrokeopacity{0.000000}%
\pgfsetdash{}{0pt}%
\pgfpathmoveto{\pgfqpoint{1.000000in}{0.660000in}}%
\pgfpathlineto{\pgfqpoint{7.200000in}{0.660000in}}%
\pgfpathlineto{\pgfqpoint{7.200000in}{5.280000in}}%
\pgfpathlineto{\pgfqpoint{1.000000in}{5.280000in}}%
\pgfpathlineto{\pgfqpoint{1.000000in}{0.660000in}}%
\pgfpathclose%
\pgfusepath{fill}%
\end{pgfscope}%
\begin{pgfscope}%
\pgfpathrectangle{\pgfqpoint{1.000000in}{0.660000in}}{\pgfqpoint{6.200000in}{4.620000in}}%
\pgfusepath{clip}%
\pgfsetbuttcap%
\pgfsetroundjoin%
\definecolor{currentfill}{rgb}{0.000000,0.000000,0.545098}%
\pgfsetfillcolor{currentfill}%
\pgfsetlinewidth{1.505625pt}%
\definecolor{currentstroke}{rgb}{0.000000,0.000000,0.545098}%
\pgfsetstrokecolor{currentstroke}%
\pgfsetdash{}{0pt}%
\pgfsys@defobject{currentmarker}{\pgfqpoint{-0.041667in}{-0.041667in}}{\pgfqpoint{0.041667in}{0.041667in}}{%
\pgfpathmoveto{\pgfqpoint{-0.041667in}{-0.041667in}}%
\pgfpathlineto{\pgfqpoint{0.041667in}{0.041667in}}%
\pgfpathmoveto{\pgfqpoint{-0.041667in}{0.041667in}}%
\pgfpathlineto{\pgfqpoint{0.041667in}{-0.041667in}}%
\pgfusepath{stroke,fill}%
}%
\begin{pgfscope}%
\pgfsys@transformshift{1.281818in}{3.010460in}%
\pgfsys@useobject{currentmarker}{}%
\end{pgfscope}%
\begin{pgfscope}%
\pgfsys@transformshift{2.534343in}{1.300169in}%
\pgfsys@useobject{currentmarker}{}%
\end{pgfscope}%
\begin{pgfscope}%
\pgfsys@transformshift{3.786869in}{0.871443in}%
\pgfsys@useobject{currentmarker}{}%
\end{pgfscope}%
\begin{pgfscope}%
\pgfsys@transformshift{5.039394in}{1.678055in}%
\pgfsys@useobject{currentmarker}{}%
\end{pgfscope}%
\begin{pgfscope}%
\pgfsys@transformshift{6.918182in}{5.058638in}%
\pgfsys@useobject{currentmarker}{}%
\end{pgfscope}%
\end{pgfscope}%
\begin{pgfscope}%
\pgfpathrectangle{\pgfqpoint{1.000000in}{0.660000in}}{\pgfqpoint{6.200000in}{4.620000in}}%
\pgfusepath{clip}%
\pgfsetbuttcap%
\pgfsetroundjoin%
\definecolor{currentfill}{rgb}{1.000000,0.000000,0.000000}%
\pgfsetfillcolor{currentfill}%
\pgfsetlinewidth{1.505625pt}%
\definecolor{currentstroke}{rgb}{1.000000,0.000000,0.000000}%
\pgfsetstrokecolor{currentstroke}%
\pgfsetdash{}{0pt}%
\pgfsys@defobject{currentmarker}{\pgfqpoint{-0.041667in}{-0.041667in}}{\pgfqpoint{0.041667in}{0.041667in}}{%
\pgfpathmoveto{\pgfqpoint{-0.041667in}{-0.041667in}}%
\pgfpathlineto{\pgfqpoint{0.041667in}{0.041667in}}%
\pgfpathmoveto{\pgfqpoint{-0.041667in}{0.041667in}}%
\pgfpathlineto{\pgfqpoint{0.041667in}{-0.041667in}}%
\pgfusepath{stroke,fill}%
}%
\begin{pgfscope}%
\pgfsys@transformshift{3.624102in}{0.870000in}%
\pgfsys@useobject{currentmarker}{}%
\end{pgfscope}%
\end{pgfscope}%
\begin{pgfscope}%
\pgfsetbuttcap%
\pgfsetroundjoin%
\definecolor{currentfill}{rgb}{0.000000,0.000000,0.000000}%
\pgfsetfillcolor{currentfill}%
\pgfsetlinewidth{0.803000pt}%
\definecolor{currentstroke}{rgb}{0.000000,0.000000,0.000000}%
\pgfsetstrokecolor{currentstroke}%
\pgfsetdash{}{0pt}%
\pgfsys@defobject{currentmarker}{\pgfqpoint{0.000000in}{0.000000in}}{\pgfqpoint{0.000000in}{0.138889in}}{%
\pgfpathmoveto{\pgfqpoint{0.000000in}{0.000000in}}%
\pgfpathlineto{\pgfqpoint{0.000000in}{0.138889in}}%
\pgfusepath{stroke,fill}%
}%
\begin{pgfscope}%
\pgfsys@transformshift{1.281818in}{0.660000in}%
\pgfsys@useobject{currentmarker}{}%
\end{pgfscope}%
\end{pgfscope}%
\begin{pgfscope}%
\pgfsetbuttcap%
\pgfsetroundjoin%
\definecolor{currentfill}{rgb}{0.000000,0.000000,0.000000}%
\pgfsetfillcolor{currentfill}%
\pgfsetlinewidth{0.803000pt}%
\definecolor{currentstroke}{rgb}{0.000000,0.000000,0.000000}%
\pgfsetstrokecolor{currentstroke}%
\pgfsetdash{}{0pt}%
\pgfsys@defobject{currentmarker}{\pgfqpoint{0.000000in}{-0.138889in}}{\pgfqpoint{0.000000in}{0.000000in}}{%
\pgfpathmoveto{\pgfqpoint{0.000000in}{0.000000in}}%
\pgfpathlineto{\pgfqpoint{0.000000in}{-0.138889in}}%
\pgfusepath{stroke,fill}%
}%
\begin{pgfscope}%
\pgfsys@transformshift{1.281818in}{5.280000in}%
\pgfsys@useobject{currentmarker}{}%
\end{pgfscope}%
\end{pgfscope}%
\begin{pgfscope}%
\definecolor{textcolor}{rgb}{0.000000,0.000000,0.000000}%
\pgfsetstrokecolor{textcolor}%
\pgfsetfillcolor{textcolor}%
\pgftext[x=1.281818in,y=0.611389in,,top]{\color{textcolor}{\rmfamily\fontsize{16.000000}{19.200000}\selectfont\catcode`\^=\active\def^{\ifmmode\sp\else\^{}\fi}\catcode`\%=\active\def%{\%}8.56}}%
\end{pgfscope}%
\begin{pgfscope}%
\pgfsetbuttcap%
\pgfsetroundjoin%
\definecolor{currentfill}{rgb}{0.000000,0.000000,0.000000}%
\pgfsetfillcolor{currentfill}%
\pgfsetlinewidth{0.803000pt}%
\definecolor{currentstroke}{rgb}{0.000000,0.000000,0.000000}%
\pgfsetstrokecolor{currentstroke}%
\pgfsetdash{}{0pt}%
\pgfsys@defobject{currentmarker}{\pgfqpoint{0.000000in}{0.000000in}}{\pgfqpoint{0.000000in}{0.138889in}}{%
\pgfpathmoveto{\pgfqpoint{0.000000in}{0.000000in}}%
\pgfpathlineto{\pgfqpoint{0.000000in}{0.138889in}}%
\pgfusepath{stroke,fill}%
}%
\begin{pgfscope}%
\pgfsys@transformshift{2.534343in}{0.660000in}%
\pgfsys@useobject{currentmarker}{}%
\end{pgfscope}%
\end{pgfscope}%
\begin{pgfscope}%
\pgfsetbuttcap%
\pgfsetroundjoin%
\definecolor{currentfill}{rgb}{0.000000,0.000000,0.000000}%
\pgfsetfillcolor{currentfill}%
\pgfsetlinewidth{0.803000pt}%
\definecolor{currentstroke}{rgb}{0.000000,0.000000,0.000000}%
\pgfsetstrokecolor{currentstroke}%
\pgfsetdash{}{0pt}%
\pgfsys@defobject{currentmarker}{\pgfqpoint{0.000000in}{-0.138889in}}{\pgfqpoint{0.000000in}{0.000000in}}{%
\pgfpathmoveto{\pgfqpoint{0.000000in}{0.000000in}}%
\pgfpathlineto{\pgfqpoint{0.000000in}{-0.138889in}}%
\pgfusepath{stroke,fill}%
}%
\begin{pgfscope}%
\pgfsys@transformshift{2.534343in}{5.280000in}%
\pgfsys@useobject{currentmarker}{}%
\end{pgfscope}%
\end{pgfscope}%
\begin{pgfscope}%
\definecolor{textcolor}{rgb}{0.000000,0.000000,0.000000}%
\pgfsetstrokecolor{textcolor}%
\pgfsetfillcolor{textcolor}%
\pgftext[x=2.534343in,y=0.611389in,,top]{\color{textcolor}{\rmfamily\fontsize{16.000000}{19.200000}\selectfont\catcode`\^=\active\def^{\ifmmode\sp\else\^{}\fi}\catcode`\%=\active\def%{\%}8.58}}%
\end{pgfscope}%
\begin{pgfscope}%
\pgfsetbuttcap%
\pgfsetroundjoin%
\definecolor{currentfill}{rgb}{0.000000,0.000000,0.000000}%
\pgfsetfillcolor{currentfill}%
\pgfsetlinewidth{0.803000pt}%
\definecolor{currentstroke}{rgb}{0.000000,0.000000,0.000000}%
\pgfsetstrokecolor{currentstroke}%
\pgfsetdash{}{0pt}%
\pgfsys@defobject{currentmarker}{\pgfqpoint{0.000000in}{0.000000in}}{\pgfqpoint{0.000000in}{0.138889in}}{%
\pgfpathmoveto{\pgfqpoint{0.000000in}{0.000000in}}%
\pgfpathlineto{\pgfqpoint{0.000000in}{0.138889in}}%
\pgfusepath{stroke,fill}%
}%
\begin{pgfscope}%
\pgfsys@transformshift{3.786869in}{0.660000in}%
\pgfsys@useobject{currentmarker}{}%
\end{pgfscope}%
\end{pgfscope}%
\begin{pgfscope}%
\pgfsetbuttcap%
\pgfsetroundjoin%
\definecolor{currentfill}{rgb}{0.000000,0.000000,0.000000}%
\pgfsetfillcolor{currentfill}%
\pgfsetlinewidth{0.803000pt}%
\definecolor{currentstroke}{rgb}{0.000000,0.000000,0.000000}%
\pgfsetstrokecolor{currentstroke}%
\pgfsetdash{}{0pt}%
\pgfsys@defobject{currentmarker}{\pgfqpoint{0.000000in}{-0.138889in}}{\pgfqpoint{0.000000in}{0.000000in}}{%
\pgfpathmoveto{\pgfqpoint{0.000000in}{0.000000in}}%
\pgfpathlineto{\pgfqpoint{0.000000in}{-0.138889in}}%
\pgfusepath{stroke,fill}%
}%
\begin{pgfscope}%
\pgfsys@transformshift{3.786869in}{5.280000in}%
\pgfsys@useobject{currentmarker}{}%
\end{pgfscope}%
\end{pgfscope}%
\begin{pgfscope}%
\definecolor{textcolor}{rgb}{0.000000,0.000000,0.000000}%
\pgfsetstrokecolor{textcolor}%
\pgfsetfillcolor{textcolor}%
\pgftext[x=3.786869in,y=0.611389in,,top]{\color{textcolor}{\rmfamily\fontsize{16.000000}{19.200000}\selectfont\catcode`\^=\active\def^{\ifmmode\sp\else\^{}\fi}\catcode`\%=\active\def%{\%}8.60}}%
\end{pgfscope}%
\begin{pgfscope}%
\pgfsetbuttcap%
\pgfsetroundjoin%
\definecolor{currentfill}{rgb}{0.000000,0.000000,0.000000}%
\pgfsetfillcolor{currentfill}%
\pgfsetlinewidth{0.803000pt}%
\definecolor{currentstroke}{rgb}{0.000000,0.000000,0.000000}%
\pgfsetstrokecolor{currentstroke}%
\pgfsetdash{}{0pt}%
\pgfsys@defobject{currentmarker}{\pgfqpoint{0.000000in}{0.000000in}}{\pgfqpoint{0.000000in}{0.138889in}}{%
\pgfpathmoveto{\pgfqpoint{0.000000in}{0.000000in}}%
\pgfpathlineto{\pgfqpoint{0.000000in}{0.138889in}}%
\pgfusepath{stroke,fill}%
}%
\begin{pgfscope}%
\pgfsys@transformshift{5.039394in}{0.660000in}%
\pgfsys@useobject{currentmarker}{}%
\end{pgfscope}%
\end{pgfscope}%
\begin{pgfscope}%
\pgfsetbuttcap%
\pgfsetroundjoin%
\definecolor{currentfill}{rgb}{0.000000,0.000000,0.000000}%
\pgfsetfillcolor{currentfill}%
\pgfsetlinewidth{0.803000pt}%
\definecolor{currentstroke}{rgb}{0.000000,0.000000,0.000000}%
\pgfsetstrokecolor{currentstroke}%
\pgfsetdash{}{0pt}%
\pgfsys@defobject{currentmarker}{\pgfqpoint{0.000000in}{-0.138889in}}{\pgfqpoint{0.000000in}{0.000000in}}{%
\pgfpathmoveto{\pgfqpoint{0.000000in}{0.000000in}}%
\pgfpathlineto{\pgfqpoint{0.000000in}{-0.138889in}}%
\pgfusepath{stroke,fill}%
}%
\begin{pgfscope}%
\pgfsys@transformshift{5.039394in}{5.280000in}%
\pgfsys@useobject{currentmarker}{}%
\end{pgfscope}%
\end{pgfscope}%
\begin{pgfscope}%
\definecolor{textcolor}{rgb}{0.000000,0.000000,0.000000}%
\pgfsetstrokecolor{textcolor}%
\pgfsetfillcolor{textcolor}%
\pgftext[x=5.039394in,y=0.611389in,,top]{\color{textcolor}{\rmfamily\fontsize{16.000000}{19.200000}\selectfont\catcode`\^=\active\def^{\ifmmode\sp\else\^{}\fi}\catcode`\%=\active\def%{\%}8.62}}%
\end{pgfscope}%
\begin{pgfscope}%
\pgfsetbuttcap%
\pgfsetroundjoin%
\definecolor{currentfill}{rgb}{0.000000,0.000000,0.000000}%
\pgfsetfillcolor{currentfill}%
\pgfsetlinewidth{0.803000pt}%
\definecolor{currentstroke}{rgb}{0.000000,0.000000,0.000000}%
\pgfsetstrokecolor{currentstroke}%
\pgfsetdash{}{0pt}%
\pgfsys@defobject{currentmarker}{\pgfqpoint{0.000000in}{0.000000in}}{\pgfqpoint{0.000000in}{0.138889in}}{%
\pgfpathmoveto{\pgfqpoint{0.000000in}{0.000000in}}%
\pgfpathlineto{\pgfqpoint{0.000000in}{0.138889in}}%
\pgfusepath{stroke,fill}%
}%
\begin{pgfscope}%
\pgfsys@transformshift{6.291919in}{0.660000in}%
\pgfsys@useobject{currentmarker}{}%
\end{pgfscope}%
\end{pgfscope}%
\begin{pgfscope}%
\pgfsetbuttcap%
\pgfsetroundjoin%
\definecolor{currentfill}{rgb}{0.000000,0.000000,0.000000}%
\pgfsetfillcolor{currentfill}%
\pgfsetlinewidth{0.803000pt}%
\definecolor{currentstroke}{rgb}{0.000000,0.000000,0.000000}%
\pgfsetstrokecolor{currentstroke}%
\pgfsetdash{}{0pt}%
\pgfsys@defobject{currentmarker}{\pgfqpoint{0.000000in}{-0.138889in}}{\pgfqpoint{0.000000in}{0.000000in}}{%
\pgfpathmoveto{\pgfqpoint{0.000000in}{0.000000in}}%
\pgfpathlineto{\pgfqpoint{0.000000in}{-0.138889in}}%
\pgfusepath{stroke,fill}%
}%
\begin{pgfscope}%
\pgfsys@transformshift{6.291919in}{5.280000in}%
\pgfsys@useobject{currentmarker}{}%
\end{pgfscope}%
\end{pgfscope}%
\begin{pgfscope}%
\definecolor{textcolor}{rgb}{0.000000,0.000000,0.000000}%
\pgfsetstrokecolor{textcolor}%
\pgfsetfillcolor{textcolor}%
\pgftext[x=6.291919in,y=0.611389in,,top]{\color{textcolor}{\rmfamily\fontsize{16.000000}{19.200000}\selectfont\catcode`\^=\active\def^{\ifmmode\sp\else\^{}\fi}\catcode`\%=\active\def%{\%}8.64}}%
\end{pgfscope}%
\begin{pgfscope}%
\definecolor{textcolor}{rgb}{0.000000,0.000000,0.000000}%
\pgfsetstrokecolor{textcolor}%
\pgfsetfillcolor{textcolor}%
\pgftext[x=4.100000in,y=0.355096in,,top]{\color{textcolor}{\rmfamily\fontsize{16.000000}{19.200000}\selectfont\catcode`\^=\active\def^{\ifmmode\sp\else\^{}\fi}\catcode`\%=\active\def%{\%}$\log \epsilon_{\mathrm{O}}$}}%
\end{pgfscope}%
\begin{pgfscope}%
\pgfsetbuttcap%
\pgfsetroundjoin%
\definecolor{currentfill}{rgb}{0.000000,0.000000,0.000000}%
\pgfsetfillcolor{currentfill}%
\pgfsetlinewidth{0.803000pt}%
\definecolor{currentstroke}{rgb}{0.000000,0.000000,0.000000}%
\pgfsetstrokecolor{currentstroke}%
\pgfsetdash{}{0pt}%
\pgfsys@defobject{currentmarker}{\pgfqpoint{0.000000in}{0.000000in}}{\pgfqpoint{0.138889in}{0.000000in}}{%
\pgfpathmoveto{\pgfqpoint{0.000000in}{0.000000in}}%
\pgfpathlineto{\pgfqpoint{0.138889in}{0.000000in}}%
\pgfusepath{stroke,fill}%
}%
\begin{pgfscope}%
\pgfsys@transformshift{1.000000in}{1.442351in}%
\pgfsys@useobject{currentmarker}{}%
\end{pgfscope}%
\end{pgfscope}%
\begin{pgfscope}%
\pgfsetbuttcap%
\pgfsetroundjoin%
\definecolor{currentfill}{rgb}{0.000000,0.000000,0.000000}%
\pgfsetfillcolor{currentfill}%
\pgfsetlinewidth{0.803000pt}%
\definecolor{currentstroke}{rgb}{0.000000,0.000000,0.000000}%
\pgfsetstrokecolor{currentstroke}%
\pgfsetdash{}{0pt}%
\pgfsys@defobject{currentmarker}{\pgfqpoint{-0.138889in}{0.000000in}}{\pgfqpoint{-0.000000in}{0.000000in}}{%
\pgfpathmoveto{\pgfqpoint{-0.000000in}{0.000000in}}%
\pgfpathlineto{\pgfqpoint{-0.138889in}{0.000000in}}%
\pgfusepath{stroke,fill}%
}%
\begin{pgfscope}%
\pgfsys@transformshift{7.200000in}{1.442351in}%
\pgfsys@useobject{currentmarker}{}%
\end{pgfscope}%
\end{pgfscope}%
\begin{pgfscope}%
\definecolor{textcolor}{rgb}{0.000000,0.000000,0.000000}%
\pgfsetstrokecolor{textcolor}%
\pgfsetfillcolor{textcolor}%
\pgftext[x=0.562500in, y=1.366179in, left, base]{\color{textcolor}{\rmfamily\fontsize{16.000000}{19.200000}\selectfont\catcode`\^=\active\def^{\ifmmode\sp\else\^{}\fi}\catcode`\%=\active\def%{\%}0.02}}%
\end{pgfscope}%
\begin{pgfscope}%
\pgfsetbuttcap%
\pgfsetroundjoin%
\definecolor{currentfill}{rgb}{0.000000,0.000000,0.000000}%
\pgfsetfillcolor{currentfill}%
\pgfsetlinewidth{0.803000pt}%
\definecolor{currentstroke}{rgb}{0.000000,0.000000,0.000000}%
\pgfsetstrokecolor{currentstroke}%
\pgfsetdash{}{0pt}%
\pgfsys@defobject{currentmarker}{\pgfqpoint{0.000000in}{0.000000in}}{\pgfqpoint{0.138889in}{0.000000in}}{%
\pgfpathmoveto{\pgfqpoint{0.000000in}{0.000000in}}%
\pgfpathlineto{\pgfqpoint{0.138889in}{0.000000in}}%
\pgfusepath{stroke,fill}%
}%
\begin{pgfscope}%
\pgfsys@transformshift{1.000000in}{2.444869in}%
\pgfsys@useobject{currentmarker}{}%
\end{pgfscope}%
\end{pgfscope}%
\begin{pgfscope}%
\pgfsetbuttcap%
\pgfsetroundjoin%
\definecolor{currentfill}{rgb}{0.000000,0.000000,0.000000}%
\pgfsetfillcolor{currentfill}%
\pgfsetlinewidth{0.803000pt}%
\definecolor{currentstroke}{rgb}{0.000000,0.000000,0.000000}%
\pgfsetstrokecolor{currentstroke}%
\pgfsetdash{}{0pt}%
\pgfsys@defobject{currentmarker}{\pgfqpoint{-0.138889in}{0.000000in}}{\pgfqpoint{-0.000000in}{0.000000in}}{%
\pgfpathmoveto{\pgfqpoint{-0.000000in}{0.000000in}}%
\pgfpathlineto{\pgfqpoint{-0.138889in}{0.000000in}}%
\pgfusepath{stroke,fill}%
}%
\begin{pgfscope}%
\pgfsys@transformshift{7.200000in}{2.444869in}%
\pgfsys@useobject{currentmarker}{}%
\end{pgfscope}%
\end{pgfscope}%
\begin{pgfscope}%
\definecolor{textcolor}{rgb}{0.000000,0.000000,0.000000}%
\pgfsetstrokecolor{textcolor}%
\pgfsetfillcolor{textcolor}%
\pgftext[x=0.562500in, y=2.368697in, left, base]{\color{textcolor}{\rmfamily\fontsize{16.000000}{19.200000}\selectfont\catcode`\^=\active\def^{\ifmmode\sp\else\^{}\fi}\catcode`\%=\active\def%{\%}0.03}}%
\end{pgfscope}%
\begin{pgfscope}%
\pgfsetbuttcap%
\pgfsetroundjoin%
\definecolor{currentfill}{rgb}{0.000000,0.000000,0.000000}%
\pgfsetfillcolor{currentfill}%
\pgfsetlinewidth{0.803000pt}%
\definecolor{currentstroke}{rgb}{0.000000,0.000000,0.000000}%
\pgfsetstrokecolor{currentstroke}%
\pgfsetdash{}{0pt}%
\pgfsys@defobject{currentmarker}{\pgfqpoint{0.000000in}{0.000000in}}{\pgfqpoint{0.138889in}{0.000000in}}{%
\pgfpathmoveto{\pgfqpoint{0.000000in}{0.000000in}}%
\pgfpathlineto{\pgfqpoint{0.138889in}{0.000000in}}%
\pgfusepath{stroke,fill}%
}%
\begin{pgfscope}%
\pgfsys@transformshift{1.000000in}{3.447386in}%
\pgfsys@useobject{currentmarker}{}%
\end{pgfscope}%
\end{pgfscope}%
\begin{pgfscope}%
\pgfsetbuttcap%
\pgfsetroundjoin%
\definecolor{currentfill}{rgb}{0.000000,0.000000,0.000000}%
\pgfsetfillcolor{currentfill}%
\pgfsetlinewidth{0.803000pt}%
\definecolor{currentstroke}{rgb}{0.000000,0.000000,0.000000}%
\pgfsetstrokecolor{currentstroke}%
\pgfsetdash{}{0pt}%
\pgfsys@defobject{currentmarker}{\pgfqpoint{-0.138889in}{0.000000in}}{\pgfqpoint{-0.000000in}{0.000000in}}{%
\pgfpathmoveto{\pgfqpoint{-0.000000in}{0.000000in}}%
\pgfpathlineto{\pgfqpoint{-0.138889in}{0.000000in}}%
\pgfusepath{stroke,fill}%
}%
\begin{pgfscope}%
\pgfsys@transformshift{7.200000in}{3.447386in}%
\pgfsys@useobject{currentmarker}{}%
\end{pgfscope}%
\end{pgfscope}%
\begin{pgfscope}%
\definecolor{textcolor}{rgb}{0.000000,0.000000,0.000000}%
\pgfsetstrokecolor{textcolor}%
\pgfsetfillcolor{textcolor}%
\pgftext[x=0.562500in, y=3.371215in, left, base]{\color{textcolor}{\rmfamily\fontsize{16.000000}{19.200000}\selectfont\catcode`\^=\active\def^{\ifmmode\sp\else\^{}\fi}\catcode`\%=\active\def%{\%}0.04}}%
\end{pgfscope}%
\begin{pgfscope}%
\pgfsetbuttcap%
\pgfsetroundjoin%
\definecolor{currentfill}{rgb}{0.000000,0.000000,0.000000}%
\pgfsetfillcolor{currentfill}%
\pgfsetlinewidth{0.803000pt}%
\definecolor{currentstroke}{rgb}{0.000000,0.000000,0.000000}%
\pgfsetstrokecolor{currentstroke}%
\pgfsetdash{}{0pt}%
\pgfsys@defobject{currentmarker}{\pgfqpoint{0.000000in}{0.000000in}}{\pgfqpoint{0.138889in}{0.000000in}}{%
\pgfpathmoveto{\pgfqpoint{0.000000in}{0.000000in}}%
\pgfpathlineto{\pgfqpoint{0.138889in}{0.000000in}}%
\pgfusepath{stroke,fill}%
}%
\begin{pgfscope}%
\pgfsys@transformshift{1.000000in}{4.449904in}%
\pgfsys@useobject{currentmarker}{}%
\end{pgfscope}%
\end{pgfscope}%
\begin{pgfscope}%
\pgfsetbuttcap%
\pgfsetroundjoin%
\definecolor{currentfill}{rgb}{0.000000,0.000000,0.000000}%
\pgfsetfillcolor{currentfill}%
\pgfsetlinewidth{0.803000pt}%
\definecolor{currentstroke}{rgb}{0.000000,0.000000,0.000000}%
\pgfsetstrokecolor{currentstroke}%
\pgfsetdash{}{0pt}%
\pgfsys@defobject{currentmarker}{\pgfqpoint{-0.138889in}{0.000000in}}{\pgfqpoint{-0.000000in}{0.000000in}}{%
\pgfpathmoveto{\pgfqpoint{-0.000000in}{0.000000in}}%
\pgfpathlineto{\pgfqpoint{-0.138889in}{0.000000in}}%
\pgfusepath{stroke,fill}%
}%
\begin{pgfscope}%
\pgfsys@transformshift{7.200000in}{4.449904in}%
\pgfsys@useobject{currentmarker}{}%
\end{pgfscope}%
\end{pgfscope}%
\begin{pgfscope}%
\definecolor{textcolor}{rgb}{0.000000,0.000000,0.000000}%
\pgfsetstrokecolor{textcolor}%
\pgfsetfillcolor{textcolor}%
\pgftext[x=0.562500in, y=4.373732in, left, base]{\color{textcolor}{\rmfamily\fontsize{16.000000}{19.200000}\selectfont\catcode`\^=\active\def^{\ifmmode\sp\else\^{}\fi}\catcode`\%=\active\def%{\%}0.05}}%
\end{pgfscope}%
\begin{pgfscope}%
\definecolor{textcolor}{rgb}{0.000000,0.000000,0.000000}%
\pgfsetstrokecolor{textcolor}%
\pgfsetfillcolor{textcolor}%
\pgftext[x=0.506944in,y=2.970000in,,bottom,rotate=90.000000]{\color{textcolor}{\rmfamily\fontsize{16.000000}{19.200000}\selectfont\catcode`\^=\active\def^{\ifmmode\sp\else\^{}\fi}\catcode`\%=\active\def%{\%}$\chi^2$}}%
\end{pgfscope}%
\begin{pgfscope}%
\pgfpathrectangle{\pgfqpoint{1.000000in}{0.660000in}}{\pgfqpoint{6.200000in}{4.620000in}}%
\pgfusepath{clip}%
\pgfsetrectcap%
\pgfsetroundjoin%
\pgfsetlinewidth{1.505625pt}%
\definecolor{currentstroke}{rgb}{0.827451,0.827451,0.827451}%
\pgfsetstrokecolor{currentstroke}%
\pgfsetdash{}{0pt}%
\pgfpathmoveto{\pgfqpoint{1.281818in}{2.993539in}}%
\pgfpathlineto{\pgfqpoint{1.338751in}{2.891562in}}%
\pgfpathlineto{\pgfqpoint{1.395684in}{2.792094in}}%
\pgfpathlineto{\pgfqpoint{1.452617in}{2.695135in}}%
\pgfpathlineto{\pgfqpoint{1.509550in}{2.600685in}}%
\pgfpathlineto{\pgfqpoint{1.566483in}{2.508744in}}%
\pgfpathlineto{\pgfqpoint{1.623416in}{2.419313in}}%
\pgfpathlineto{\pgfqpoint{1.680349in}{2.332391in}}%
\pgfpathlineto{\pgfqpoint{1.737282in}{2.247978in}}%
\pgfpathlineto{\pgfqpoint{1.794215in}{2.166074in}}%
\pgfpathlineto{\pgfqpoint{1.851148in}{2.086680in}}%
\pgfpathlineto{\pgfqpoint{1.908081in}{2.009795in}}%
\pgfpathlineto{\pgfqpoint{1.965014in}{1.935419in}}%
\pgfpathlineto{\pgfqpoint{2.021947in}{1.863552in}}%
\pgfpathlineto{\pgfqpoint{2.078880in}{1.794194in}}%
\pgfpathlineto{\pgfqpoint{2.135813in}{1.727346in}}%
\pgfpathlineto{\pgfqpoint{2.192746in}{1.663007in}}%
\pgfpathlineto{\pgfqpoint{2.249679in}{1.601177in}}%
\pgfpathlineto{\pgfqpoint{2.306612in}{1.541856in}}%
\pgfpathlineto{\pgfqpoint{2.363545in}{1.485045in}}%
\pgfpathlineto{\pgfqpoint{2.420478in}{1.430743in}}%
\pgfpathlineto{\pgfqpoint{2.477410in}{1.378949in}}%
\pgfpathlineto{\pgfqpoint{2.534343in}{1.329666in}}%
\pgfpathlineto{\pgfqpoint{2.591276in}{1.282891in}}%
\pgfpathlineto{\pgfqpoint{2.648209in}{1.238626in}}%
\pgfpathlineto{\pgfqpoint{2.705142in}{1.196870in}}%
\pgfpathlineto{\pgfqpoint{2.762075in}{1.157623in}}%
\pgfpathlineto{\pgfqpoint{2.819008in}{1.120885in}}%
\pgfpathlineto{\pgfqpoint{2.875941in}{1.086656in}}%
\pgfpathlineto{\pgfqpoint{2.932874in}{1.054937in}}%
\pgfpathlineto{\pgfqpoint{2.989807in}{1.025727in}}%
\pgfpathlineto{\pgfqpoint{3.046740in}{0.999026in}}%
\pgfpathlineto{\pgfqpoint{3.103673in}{0.974835in}}%
\pgfpathlineto{\pgfqpoint{3.160606in}{0.953152in}}%
\pgfpathlineto{\pgfqpoint{3.217539in}{0.933979in}}%
\pgfpathlineto{\pgfqpoint{3.274472in}{0.917315in}}%
\pgfpathlineto{\pgfqpoint{3.331405in}{0.903160in}}%
\pgfpathlineto{\pgfqpoint{3.388338in}{0.891515in}}%
\pgfpathlineto{\pgfqpoint{3.445271in}{0.882379in}}%
\pgfpathlineto{\pgfqpoint{3.502204in}{0.875751in}}%
\pgfpathlineto{\pgfqpoint{3.559137in}{0.871634in}}%
\pgfpathlineto{\pgfqpoint{3.616070in}{0.870025in}}%
\pgfpathlineto{\pgfqpoint{3.673003in}{0.870926in}}%
\pgfpathlineto{\pgfqpoint{3.729936in}{0.874335in}}%
\pgfpathlineto{\pgfqpoint{3.786869in}{0.880254in}}%
\pgfpathlineto{\pgfqpoint{3.843802in}{0.888683in}}%
\pgfpathlineto{\pgfqpoint{3.900735in}{0.899620in}}%
\pgfpathlineto{\pgfqpoint{3.957668in}{0.913067in}}%
\pgfpathlineto{\pgfqpoint{4.014601in}{0.929023in}}%
\pgfpathlineto{\pgfqpoint{4.071534in}{0.947488in}}%
\pgfpathlineto{\pgfqpoint{4.128466in}{0.968462in}}%
\pgfpathlineto{\pgfqpoint{4.185399in}{0.991946in}}%
\pgfpathlineto{\pgfqpoint{4.242332in}{1.017939in}}%
\pgfpathlineto{\pgfqpoint{4.299265in}{1.046441in}}%
\pgfpathlineto{\pgfqpoint{4.356198in}{1.077452in}}%
\pgfpathlineto{\pgfqpoint{4.413131in}{1.110972in}}%
\pgfpathlineto{\pgfqpoint{4.470064in}{1.147002in}}%
\pgfpathlineto{\pgfqpoint{4.526997in}{1.185541in}}%
\pgfpathlineto{\pgfqpoint{4.583930in}{1.226589in}}%
\pgfpathlineto{\pgfqpoint{4.640863in}{1.270146in}}%
\pgfpathlineto{\pgfqpoint{4.697796in}{1.316213in}}%
\pgfpathlineto{\pgfqpoint{4.754729in}{1.364789in}}%
\pgfpathlineto{\pgfqpoint{4.811662in}{1.415874in}}%
\pgfpathlineto{\pgfqpoint{4.868595in}{1.469468in}}%
\pgfpathlineto{\pgfqpoint{4.925528in}{1.525571in}}%
\pgfpathlineto{\pgfqpoint{4.982461in}{1.584184in}}%
\pgfpathlineto{\pgfqpoint{5.039394in}{1.645306in}}%
\pgfpathlineto{\pgfqpoint{5.096327in}{1.708937in}}%
\pgfpathlineto{\pgfqpoint{5.153260in}{1.775077in}}%
\pgfpathlineto{\pgfqpoint{5.210193in}{1.843727in}}%
\pgfpathlineto{\pgfqpoint{5.267126in}{1.914885in}}%
\pgfpathlineto{\pgfqpoint{5.324059in}{1.988553in}}%
\pgfpathlineto{\pgfqpoint{5.380992in}{2.064731in}}%
\pgfpathlineto{\pgfqpoint{5.437925in}{2.143417in}}%
\pgfpathlineto{\pgfqpoint{5.494858in}{2.224613in}}%
\pgfpathlineto{\pgfqpoint{5.551791in}{2.308317in}}%
\pgfpathlineto{\pgfqpoint{5.608724in}{2.394532in}}%
\pgfpathlineto{\pgfqpoint{5.665657in}{2.483255in}}%
\pgfpathlineto{\pgfqpoint{5.722590in}{2.574487in}}%
\pgfpathlineto{\pgfqpoint{5.779522in}{2.668229in}}%
\pgfpathlineto{\pgfqpoint{5.836455in}{2.764480in}}%
\pgfpathlineto{\pgfqpoint{5.893388in}{2.863240in}}%
\pgfpathlineto{\pgfqpoint{5.950321in}{2.964510in}}%
\pgfpathlineto{\pgfqpoint{6.007254in}{3.068288in}}%
\pgfpathlineto{\pgfqpoint{6.064187in}{3.174576in}}%
\pgfpathlineto{\pgfqpoint{6.121120in}{3.283373in}}%
\pgfpathlineto{\pgfqpoint{6.178053in}{3.394679in}}%
\pgfpathlineto{\pgfqpoint{6.234986in}{3.508495in}}%
\pgfpathlineto{\pgfqpoint{6.291919in}{3.624820in}}%
\pgfpathlineto{\pgfqpoint{6.348852in}{3.743654in}}%
\pgfpathlineto{\pgfqpoint{6.405785in}{3.864997in}}%
\pgfpathlineto{\pgfqpoint{6.462718in}{3.988849in}}%
\pgfpathlineto{\pgfqpoint{6.519651in}{4.115211in}}%
\pgfpathlineto{\pgfqpoint{6.576584in}{4.244081in}}%
\pgfpathlineto{\pgfqpoint{6.633517in}{4.375462in}}%
\pgfpathlineto{\pgfqpoint{6.690450in}{4.509351in}}%
\pgfpathlineto{\pgfqpoint{6.747383in}{4.645749in}}%
\pgfpathlineto{\pgfqpoint{6.804316in}{4.784657in}}%
\pgfpathlineto{\pgfqpoint{6.861249in}{4.926074in}}%
\pgfpathlineto{\pgfqpoint{6.918182in}{5.070000in}}%
\pgfusepath{stroke}%
\end{pgfscope}%
\begin{pgfscope}%
\pgfsetrectcap%
\pgfsetmiterjoin%
\pgfsetlinewidth{0.803000pt}%
\definecolor{currentstroke}{rgb}{0.000000,0.000000,0.000000}%
\pgfsetstrokecolor{currentstroke}%
\pgfsetdash{}{0pt}%
\pgfpathmoveto{\pgfqpoint{1.000000in}{0.660000in}}%
\pgfpathlineto{\pgfqpoint{1.000000in}{5.280000in}}%
\pgfusepath{stroke}%
\end{pgfscope}%
\begin{pgfscope}%
\pgfsetrectcap%
\pgfsetmiterjoin%
\pgfsetlinewidth{0.803000pt}%
\definecolor{currentstroke}{rgb}{0.000000,0.000000,0.000000}%
\pgfsetstrokecolor{currentstroke}%
\pgfsetdash{}{0pt}%
\pgfpathmoveto{\pgfqpoint{7.200000in}{0.660000in}}%
\pgfpathlineto{\pgfqpoint{7.200000in}{5.280000in}}%
\pgfusepath{stroke}%
\end{pgfscope}%
\begin{pgfscope}%
\pgfsetrectcap%
\pgfsetmiterjoin%
\pgfsetlinewidth{0.803000pt}%
\definecolor{currentstroke}{rgb}{0.000000,0.000000,0.000000}%
\pgfsetstrokecolor{currentstroke}%
\pgfsetdash{}{0pt}%
\pgfpathmoveto{\pgfqpoint{1.000000in}{0.660000in}}%
\pgfpathlineto{\pgfqpoint{7.200000in}{0.660000in}}%
\pgfusepath{stroke}%
\end{pgfscope}%
\begin{pgfscope}%
\pgfsetrectcap%
\pgfsetmiterjoin%
\pgfsetlinewidth{0.803000pt}%
\definecolor{currentstroke}{rgb}{0.000000,0.000000,0.000000}%
\pgfsetstrokecolor{currentstroke}%
\pgfsetdash{}{0pt}%
\pgfpathmoveto{\pgfqpoint{1.000000in}{5.280000in}}%
\pgfpathlineto{\pgfqpoint{7.200000in}{5.280000in}}%
\pgfusepath{stroke}%
\end{pgfscope}%
\begin{pgfscope}%
\pgfsetbuttcap%
\pgfsetmiterjoin%
\definecolor{currentfill}{rgb}{1.000000,1.000000,1.000000}%
\pgfsetfillcolor{currentfill}%
\pgfsetfillopacity{0.800000}%
\pgfsetlinewidth{1.003750pt}%
\definecolor{currentstroke}{rgb}{0.800000,0.800000,0.800000}%
\pgfsetstrokecolor{currentstroke}%
\pgfsetstrokeopacity{0.800000}%
\pgfsetdash{}{0pt}%
\pgfpathmoveto{\pgfqpoint{1.155556in}{4.156563in}}%
\pgfpathlineto{\pgfqpoint{4.397049in}{4.156563in}}%
\pgfpathquadraticcurveto{\pgfqpoint{4.441493in}{4.156563in}}{\pgfqpoint{4.441493in}{4.201008in}}%
\pgfpathlineto{\pgfqpoint{4.441493in}{5.124444in}}%
\pgfpathquadraticcurveto{\pgfqpoint{4.441493in}{5.168889in}}{\pgfqpoint{4.397049in}{5.168889in}}%
\pgfpathlineto{\pgfqpoint{1.155556in}{5.168889in}}%
\pgfpathquadraticcurveto{\pgfqpoint{1.111111in}{5.168889in}}{\pgfqpoint{1.111111in}{5.124444in}}%
\pgfpathlineto{\pgfqpoint{1.111111in}{4.201008in}}%
\pgfpathquadraticcurveto{\pgfqpoint{1.111111in}{4.156563in}}{\pgfqpoint{1.155556in}{4.156563in}}%
\pgfpathlineto{\pgfqpoint{1.155556in}{4.156563in}}%
\pgfpathclose%
\pgfusepath{stroke,fill}%
\end{pgfscope}%
\begin{pgfscope}%
\pgfsetbuttcap%
\pgfsetroundjoin%
\definecolor{currentfill}{rgb}{0.000000,0.000000,0.545098}%
\pgfsetfillcolor{currentfill}%
\pgfsetlinewidth{1.505625pt}%
\definecolor{currentstroke}{rgb}{0.000000,0.000000,0.545098}%
\pgfsetstrokecolor{currentstroke}%
\pgfsetdash{}{0pt}%
\pgfsys@defobject{currentmarker}{\pgfqpoint{-0.041667in}{-0.041667in}}{\pgfqpoint{0.041667in}{0.041667in}}{%
\pgfpathmoveto{\pgfqpoint{-0.041667in}{-0.041667in}}%
\pgfpathlineto{\pgfqpoint{0.041667in}{0.041667in}}%
\pgfpathmoveto{\pgfqpoint{-0.041667in}{0.041667in}}%
\pgfpathlineto{\pgfqpoint{0.041667in}{-0.041667in}}%
\pgfusepath{stroke,fill}%
}%
\begin{pgfscope}%
\pgfsys@transformshift{1.422222in}{4.982517in}%
\pgfsys@useobject{currentmarker}{}%
\end{pgfscope}%
\end{pgfscope}%
\begin{pgfscope}%
\definecolor{textcolor}{rgb}{0.000000,0.000000,0.000000}%
\pgfsetstrokecolor{textcolor}%
\pgfsetfillcolor{textcolor}%
\pgftext[x=1.822222in,y=4.924184in,left,base]{\color{textcolor}{\rmfamily\fontsize{16.000000}{19.200000}\selectfont\catcode`\^=\active\def^{\ifmmode\sp\else\^{}\fi}\catcode`\%=\active\def%{\%}Raie de OH en 15391.057 Å}}%
\end{pgfscope}%
\begin{pgfscope}%
\pgfsetrectcap%
\pgfsetroundjoin%
\pgfsetlinewidth{1.505625pt}%
\definecolor{currentstroke}{rgb}{0.827451,0.827451,0.827451}%
\pgfsetstrokecolor{currentstroke}%
\pgfsetdash{}{0pt}%
\pgfpathmoveto{\pgfqpoint{1.200000in}{4.686901in}}%
\pgfpathlineto{\pgfqpoint{1.422222in}{4.686901in}}%
\pgfpathlineto{\pgfqpoint{1.644444in}{4.686901in}}%
\pgfusepath{stroke}%
\end{pgfscope}%
\begin{pgfscope}%
\definecolor{textcolor}{rgb}{0.000000,0.000000,0.000000}%
\pgfsetstrokecolor{textcolor}%
\pgfsetfillcolor{textcolor}%
\pgftext[x=1.822222in,y=4.609123in,left,base]{\color{textcolor}{\rmfamily\fontsize{16.000000}{19.200000}\selectfont\catcode`\^=\active\def^{\ifmmode\sp\else\^{}\fi}\catcode`\%=\active\def%{\%}Ajustement quadratique}}%
\end{pgfscope}%
\begin{pgfscope}%
\pgfsetbuttcap%
\pgfsetroundjoin%
\definecolor{currentfill}{rgb}{1.000000,0.000000,0.000000}%
\pgfsetfillcolor{currentfill}%
\pgfsetlinewidth{1.505625pt}%
\definecolor{currentstroke}{rgb}{1.000000,0.000000,0.000000}%
\pgfsetstrokecolor{currentstroke}%
\pgfsetdash{}{0pt}%
\pgfsys@defobject{currentmarker}{\pgfqpoint{-0.041667in}{-0.041667in}}{\pgfqpoint{0.041667in}{0.041667in}}{%
\pgfpathmoveto{\pgfqpoint{-0.041667in}{-0.041667in}}%
\pgfpathlineto{\pgfqpoint{0.041667in}{0.041667in}}%
\pgfpathmoveto{\pgfqpoint{-0.041667in}{0.041667in}}%
\pgfpathlineto{\pgfqpoint{0.041667in}{-0.041667in}}%
\pgfusepath{stroke,fill}%
}%
\begin{pgfscope}%
\pgfsys@transformshift{1.422222in}{4.352179in}%
\pgfsys@useobject{currentmarker}{}%
\end{pgfscope}%
\end{pgfscope}%
\begin{pgfscope}%
\definecolor{textcolor}{rgb}{0.000000,0.000000,0.000000}%
\pgfsetstrokecolor{textcolor}%
\pgfsetfillcolor{textcolor}%
\pgftext[x=1.822222in,y=4.293846in,left,base]{\color{textcolor}{\rmfamily\fontsize{16.000000}{19.200000}\selectfont\catcode`\^=\active\def^{\ifmmode\sp\else\^{}\fi}\catcode`\%=\active\def%{\%}Minimum en 8.60}}%
\end{pgfscope}%
\end{pgfpicture}%
\makeatother%
\endgroup%
} % Insérer le fichier PGF du troisième graphique
%   \end{subfigure}
%   \hfill
%   % Quatrième image (graphique 4)
%   \begin{subfigure}[b]{0.45\textwidth}
%       \centering
%       \scalebox{0.45}{%% Creator: Matplotlib, PGF backend
%%
%% To include the figure in your LaTeX document, write
%%   \input{<filename>.pgf}
%%
%% Make sure the required packages are loaded in your preamble
%%   \usepackage{pgf}
%%
%% Also ensure that all the required font packages are loaded; for instance,
%% the lmodern package is sometimes necessary when using math font.
%%   \usepackage{lmodern}
%%
%% Figures using additional raster images can only be included by \input if
%% they are in the same directory as the main LaTeX file. For loading figures
%% from other directories you can use the `import` package
%%   \usepackage{import}
%%
%% and then include the figures with
%%   \import{<path to file>}{<filename>.pgf}
%%
%% Matplotlib used the following preamble
%%   \def\mathdefault#1{#1}
%%   \everymath=\expandafter{\the\everymath\displaystyle}
%%   
%%   \ifdefined\pdftexversion\else  % non-pdftex case.
%%     \usepackage{fontspec}
%%     \setmainfont{Times.ttc}[Path=\detokenize{/System/Library/Fonts/}]
%%     \setsansfont{DejaVuSans.ttf}[Path=\detokenize{/Users/margauxvandererven/anaconda3/lib/python3.11/site-packages/matplotlib/mpl-data/fonts/ttf/}]
%%     \setmonofont{DejaVuSansMono.ttf}[Path=\detokenize{/Users/margauxvandererven/anaconda3/lib/python3.11/site-packages/matplotlib/mpl-data/fonts/ttf/}]
%%   \fi
%%   \makeatletter\@ifpackageloaded{underscore}{}{\usepackage[strings]{underscore}}\makeatother
%%
\begingroup%
\makeatletter%
\begin{pgfpicture}%
\pgfpathrectangle{\pgfpointorigin}{\pgfqpoint{8.000000in}{6.000000in}}%
\pgfusepath{use as bounding box, clip}%
\begin{pgfscope}%
\pgfsetbuttcap%
\pgfsetmiterjoin%
\definecolor{currentfill}{rgb}{1.000000,1.000000,1.000000}%
\pgfsetfillcolor{currentfill}%
\pgfsetlinewidth{0.000000pt}%
\definecolor{currentstroke}{rgb}{1.000000,1.000000,1.000000}%
\pgfsetstrokecolor{currentstroke}%
\pgfsetdash{}{0pt}%
\pgfpathmoveto{\pgfqpoint{0.000000in}{0.000000in}}%
\pgfpathlineto{\pgfqpoint{8.000000in}{0.000000in}}%
\pgfpathlineto{\pgfqpoint{8.000000in}{6.000000in}}%
\pgfpathlineto{\pgfqpoint{0.000000in}{6.000000in}}%
\pgfpathlineto{\pgfqpoint{0.000000in}{0.000000in}}%
\pgfpathclose%
\pgfusepath{fill}%
\end{pgfscope}%
\begin{pgfscope}%
\pgfsetbuttcap%
\pgfsetmiterjoin%
\definecolor{currentfill}{rgb}{1.000000,1.000000,1.000000}%
\pgfsetfillcolor{currentfill}%
\pgfsetlinewidth{0.000000pt}%
\definecolor{currentstroke}{rgb}{0.000000,0.000000,0.000000}%
\pgfsetstrokecolor{currentstroke}%
\pgfsetstrokeopacity{0.000000}%
\pgfsetdash{}{0pt}%
\pgfpathmoveto{\pgfqpoint{1.000000in}{0.660000in}}%
\pgfpathlineto{\pgfqpoint{7.200000in}{0.660000in}}%
\pgfpathlineto{\pgfqpoint{7.200000in}{5.280000in}}%
\pgfpathlineto{\pgfqpoint{1.000000in}{5.280000in}}%
\pgfpathlineto{\pgfqpoint{1.000000in}{0.660000in}}%
\pgfpathclose%
\pgfusepath{fill}%
\end{pgfscope}%
\begin{pgfscope}%
\pgfpathrectangle{\pgfqpoint{1.000000in}{0.660000in}}{\pgfqpoint{6.200000in}{4.620000in}}%
\pgfusepath{clip}%
\pgfsetbuttcap%
\pgfsetroundjoin%
\definecolor{currentfill}{rgb}{0.000000,0.000000,0.545098}%
\pgfsetfillcolor{currentfill}%
\pgfsetlinewidth{1.505625pt}%
\definecolor{currentstroke}{rgb}{0.000000,0.000000,0.545098}%
\pgfsetstrokecolor{currentstroke}%
\pgfsetdash{}{0pt}%
\pgfsys@defobject{currentmarker}{\pgfqpoint{-0.041667in}{-0.041667in}}{\pgfqpoint{0.041667in}{0.041667in}}{%
\pgfpathmoveto{\pgfqpoint{-0.041667in}{-0.041667in}}%
\pgfpathlineto{\pgfqpoint{0.041667in}{0.041667in}}%
\pgfpathmoveto{\pgfqpoint{-0.041667in}{0.041667in}}%
\pgfpathlineto{\pgfqpoint{0.041667in}{-0.041667in}}%
\pgfusepath{stroke,fill}%
}%
\begin{pgfscope}%
\pgfsys@transformshift{1.281818in}{3.010460in}%
\pgfsys@useobject{currentmarker}{}%
\end{pgfscope}%
\begin{pgfscope}%
\pgfsys@transformshift{2.534343in}{1.300169in}%
\pgfsys@useobject{currentmarker}{}%
\end{pgfscope}%
\begin{pgfscope}%
\pgfsys@transformshift{3.786869in}{0.871443in}%
\pgfsys@useobject{currentmarker}{}%
\end{pgfscope}%
\begin{pgfscope}%
\pgfsys@transformshift{5.039394in}{1.678055in}%
\pgfsys@useobject{currentmarker}{}%
\end{pgfscope}%
\begin{pgfscope}%
\pgfsys@transformshift{6.918182in}{5.058638in}%
\pgfsys@useobject{currentmarker}{}%
\end{pgfscope}%
\end{pgfscope}%
\begin{pgfscope}%
\pgfpathrectangle{\pgfqpoint{1.000000in}{0.660000in}}{\pgfqpoint{6.200000in}{4.620000in}}%
\pgfusepath{clip}%
\pgfsetbuttcap%
\pgfsetroundjoin%
\definecolor{currentfill}{rgb}{1.000000,0.000000,0.000000}%
\pgfsetfillcolor{currentfill}%
\pgfsetlinewidth{1.505625pt}%
\definecolor{currentstroke}{rgb}{1.000000,0.000000,0.000000}%
\pgfsetstrokecolor{currentstroke}%
\pgfsetdash{}{0pt}%
\pgfsys@defobject{currentmarker}{\pgfqpoint{-0.041667in}{-0.041667in}}{\pgfqpoint{0.041667in}{0.041667in}}{%
\pgfpathmoveto{\pgfqpoint{-0.041667in}{-0.041667in}}%
\pgfpathlineto{\pgfqpoint{0.041667in}{0.041667in}}%
\pgfpathmoveto{\pgfqpoint{-0.041667in}{0.041667in}}%
\pgfpathlineto{\pgfqpoint{0.041667in}{-0.041667in}}%
\pgfusepath{stroke,fill}%
}%
\begin{pgfscope}%
\pgfsys@transformshift{3.624102in}{0.870000in}%
\pgfsys@useobject{currentmarker}{}%
\end{pgfscope}%
\end{pgfscope}%
\begin{pgfscope}%
\pgfsetbuttcap%
\pgfsetroundjoin%
\definecolor{currentfill}{rgb}{0.000000,0.000000,0.000000}%
\pgfsetfillcolor{currentfill}%
\pgfsetlinewidth{0.803000pt}%
\definecolor{currentstroke}{rgb}{0.000000,0.000000,0.000000}%
\pgfsetstrokecolor{currentstroke}%
\pgfsetdash{}{0pt}%
\pgfsys@defobject{currentmarker}{\pgfqpoint{0.000000in}{0.000000in}}{\pgfqpoint{0.000000in}{0.138889in}}{%
\pgfpathmoveto{\pgfqpoint{0.000000in}{0.000000in}}%
\pgfpathlineto{\pgfqpoint{0.000000in}{0.138889in}}%
\pgfusepath{stroke,fill}%
}%
\begin{pgfscope}%
\pgfsys@transformshift{1.281818in}{0.660000in}%
\pgfsys@useobject{currentmarker}{}%
\end{pgfscope}%
\end{pgfscope}%
\begin{pgfscope}%
\pgfsetbuttcap%
\pgfsetroundjoin%
\definecolor{currentfill}{rgb}{0.000000,0.000000,0.000000}%
\pgfsetfillcolor{currentfill}%
\pgfsetlinewidth{0.803000pt}%
\definecolor{currentstroke}{rgb}{0.000000,0.000000,0.000000}%
\pgfsetstrokecolor{currentstroke}%
\pgfsetdash{}{0pt}%
\pgfsys@defobject{currentmarker}{\pgfqpoint{0.000000in}{-0.138889in}}{\pgfqpoint{0.000000in}{0.000000in}}{%
\pgfpathmoveto{\pgfqpoint{0.000000in}{0.000000in}}%
\pgfpathlineto{\pgfqpoint{0.000000in}{-0.138889in}}%
\pgfusepath{stroke,fill}%
}%
\begin{pgfscope}%
\pgfsys@transformshift{1.281818in}{5.280000in}%
\pgfsys@useobject{currentmarker}{}%
\end{pgfscope}%
\end{pgfscope}%
\begin{pgfscope}%
\definecolor{textcolor}{rgb}{0.000000,0.000000,0.000000}%
\pgfsetstrokecolor{textcolor}%
\pgfsetfillcolor{textcolor}%
\pgftext[x=1.281818in,y=0.611389in,,top]{\color{textcolor}{\rmfamily\fontsize{16.000000}{19.200000}\selectfont\catcode`\^=\active\def^{\ifmmode\sp\else\^{}\fi}\catcode`\%=\active\def%{\%}8.56}}%
\end{pgfscope}%
\begin{pgfscope}%
\pgfsetbuttcap%
\pgfsetroundjoin%
\definecolor{currentfill}{rgb}{0.000000,0.000000,0.000000}%
\pgfsetfillcolor{currentfill}%
\pgfsetlinewidth{0.803000pt}%
\definecolor{currentstroke}{rgb}{0.000000,0.000000,0.000000}%
\pgfsetstrokecolor{currentstroke}%
\pgfsetdash{}{0pt}%
\pgfsys@defobject{currentmarker}{\pgfqpoint{0.000000in}{0.000000in}}{\pgfqpoint{0.000000in}{0.138889in}}{%
\pgfpathmoveto{\pgfqpoint{0.000000in}{0.000000in}}%
\pgfpathlineto{\pgfqpoint{0.000000in}{0.138889in}}%
\pgfusepath{stroke,fill}%
}%
\begin{pgfscope}%
\pgfsys@transformshift{2.534343in}{0.660000in}%
\pgfsys@useobject{currentmarker}{}%
\end{pgfscope}%
\end{pgfscope}%
\begin{pgfscope}%
\pgfsetbuttcap%
\pgfsetroundjoin%
\definecolor{currentfill}{rgb}{0.000000,0.000000,0.000000}%
\pgfsetfillcolor{currentfill}%
\pgfsetlinewidth{0.803000pt}%
\definecolor{currentstroke}{rgb}{0.000000,0.000000,0.000000}%
\pgfsetstrokecolor{currentstroke}%
\pgfsetdash{}{0pt}%
\pgfsys@defobject{currentmarker}{\pgfqpoint{0.000000in}{-0.138889in}}{\pgfqpoint{0.000000in}{0.000000in}}{%
\pgfpathmoveto{\pgfqpoint{0.000000in}{0.000000in}}%
\pgfpathlineto{\pgfqpoint{0.000000in}{-0.138889in}}%
\pgfusepath{stroke,fill}%
}%
\begin{pgfscope}%
\pgfsys@transformshift{2.534343in}{5.280000in}%
\pgfsys@useobject{currentmarker}{}%
\end{pgfscope}%
\end{pgfscope}%
\begin{pgfscope}%
\definecolor{textcolor}{rgb}{0.000000,0.000000,0.000000}%
\pgfsetstrokecolor{textcolor}%
\pgfsetfillcolor{textcolor}%
\pgftext[x=2.534343in,y=0.611389in,,top]{\color{textcolor}{\rmfamily\fontsize{16.000000}{19.200000}\selectfont\catcode`\^=\active\def^{\ifmmode\sp\else\^{}\fi}\catcode`\%=\active\def%{\%}8.58}}%
\end{pgfscope}%
\begin{pgfscope}%
\pgfsetbuttcap%
\pgfsetroundjoin%
\definecolor{currentfill}{rgb}{0.000000,0.000000,0.000000}%
\pgfsetfillcolor{currentfill}%
\pgfsetlinewidth{0.803000pt}%
\definecolor{currentstroke}{rgb}{0.000000,0.000000,0.000000}%
\pgfsetstrokecolor{currentstroke}%
\pgfsetdash{}{0pt}%
\pgfsys@defobject{currentmarker}{\pgfqpoint{0.000000in}{0.000000in}}{\pgfqpoint{0.000000in}{0.138889in}}{%
\pgfpathmoveto{\pgfqpoint{0.000000in}{0.000000in}}%
\pgfpathlineto{\pgfqpoint{0.000000in}{0.138889in}}%
\pgfusepath{stroke,fill}%
}%
\begin{pgfscope}%
\pgfsys@transformshift{3.786869in}{0.660000in}%
\pgfsys@useobject{currentmarker}{}%
\end{pgfscope}%
\end{pgfscope}%
\begin{pgfscope}%
\pgfsetbuttcap%
\pgfsetroundjoin%
\definecolor{currentfill}{rgb}{0.000000,0.000000,0.000000}%
\pgfsetfillcolor{currentfill}%
\pgfsetlinewidth{0.803000pt}%
\definecolor{currentstroke}{rgb}{0.000000,0.000000,0.000000}%
\pgfsetstrokecolor{currentstroke}%
\pgfsetdash{}{0pt}%
\pgfsys@defobject{currentmarker}{\pgfqpoint{0.000000in}{-0.138889in}}{\pgfqpoint{0.000000in}{0.000000in}}{%
\pgfpathmoveto{\pgfqpoint{0.000000in}{0.000000in}}%
\pgfpathlineto{\pgfqpoint{0.000000in}{-0.138889in}}%
\pgfusepath{stroke,fill}%
}%
\begin{pgfscope}%
\pgfsys@transformshift{3.786869in}{5.280000in}%
\pgfsys@useobject{currentmarker}{}%
\end{pgfscope}%
\end{pgfscope}%
\begin{pgfscope}%
\definecolor{textcolor}{rgb}{0.000000,0.000000,0.000000}%
\pgfsetstrokecolor{textcolor}%
\pgfsetfillcolor{textcolor}%
\pgftext[x=3.786869in,y=0.611389in,,top]{\color{textcolor}{\rmfamily\fontsize{16.000000}{19.200000}\selectfont\catcode`\^=\active\def^{\ifmmode\sp\else\^{}\fi}\catcode`\%=\active\def%{\%}8.60}}%
\end{pgfscope}%
\begin{pgfscope}%
\pgfsetbuttcap%
\pgfsetroundjoin%
\definecolor{currentfill}{rgb}{0.000000,0.000000,0.000000}%
\pgfsetfillcolor{currentfill}%
\pgfsetlinewidth{0.803000pt}%
\definecolor{currentstroke}{rgb}{0.000000,0.000000,0.000000}%
\pgfsetstrokecolor{currentstroke}%
\pgfsetdash{}{0pt}%
\pgfsys@defobject{currentmarker}{\pgfqpoint{0.000000in}{0.000000in}}{\pgfqpoint{0.000000in}{0.138889in}}{%
\pgfpathmoveto{\pgfqpoint{0.000000in}{0.000000in}}%
\pgfpathlineto{\pgfqpoint{0.000000in}{0.138889in}}%
\pgfusepath{stroke,fill}%
}%
\begin{pgfscope}%
\pgfsys@transformshift{5.039394in}{0.660000in}%
\pgfsys@useobject{currentmarker}{}%
\end{pgfscope}%
\end{pgfscope}%
\begin{pgfscope}%
\pgfsetbuttcap%
\pgfsetroundjoin%
\definecolor{currentfill}{rgb}{0.000000,0.000000,0.000000}%
\pgfsetfillcolor{currentfill}%
\pgfsetlinewidth{0.803000pt}%
\definecolor{currentstroke}{rgb}{0.000000,0.000000,0.000000}%
\pgfsetstrokecolor{currentstroke}%
\pgfsetdash{}{0pt}%
\pgfsys@defobject{currentmarker}{\pgfqpoint{0.000000in}{-0.138889in}}{\pgfqpoint{0.000000in}{0.000000in}}{%
\pgfpathmoveto{\pgfqpoint{0.000000in}{0.000000in}}%
\pgfpathlineto{\pgfqpoint{0.000000in}{-0.138889in}}%
\pgfusepath{stroke,fill}%
}%
\begin{pgfscope}%
\pgfsys@transformshift{5.039394in}{5.280000in}%
\pgfsys@useobject{currentmarker}{}%
\end{pgfscope}%
\end{pgfscope}%
\begin{pgfscope}%
\definecolor{textcolor}{rgb}{0.000000,0.000000,0.000000}%
\pgfsetstrokecolor{textcolor}%
\pgfsetfillcolor{textcolor}%
\pgftext[x=5.039394in,y=0.611389in,,top]{\color{textcolor}{\rmfamily\fontsize{16.000000}{19.200000}\selectfont\catcode`\^=\active\def^{\ifmmode\sp\else\^{}\fi}\catcode`\%=\active\def%{\%}8.62}}%
\end{pgfscope}%
\begin{pgfscope}%
\pgfsetbuttcap%
\pgfsetroundjoin%
\definecolor{currentfill}{rgb}{0.000000,0.000000,0.000000}%
\pgfsetfillcolor{currentfill}%
\pgfsetlinewidth{0.803000pt}%
\definecolor{currentstroke}{rgb}{0.000000,0.000000,0.000000}%
\pgfsetstrokecolor{currentstroke}%
\pgfsetdash{}{0pt}%
\pgfsys@defobject{currentmarker}{\pgfqpoint{0.000000in}{0.000000in}}{\pgfqpoint{0.000000in}{0.138889in}}{%
\pgfpathmoveto{\pgfqpoint{0.000000in}{0.000000in}}%
\pgfpathlineto{\pgfqpoint{0.000000in}{0.138889in}}%
\pgfusepath{stroke,fill}%
}%
\begin{pgfscope}%
\pgfsys@transformshift{6.291919in}{0.660000in}%
\pgfsys@useobject{currentmarker}{}%
\end{pgfscope}%
\end{pgfscope}%
\begin{pgfscope}%
\pgfsetbuttcap%
\pgfsetroundjoin%
\definecolor{currentfill}{rgb}{0.000000,0.000000,0.000000}%
\pgfsetfillcolor{currentfill}%
\pgfsetlinewidth{0.803000pt}%
\definecolor{currentstroke}{rgb}{0.000000,0.000000,0.000000}%
\pgfsetstrokecolor{currentstroke}%
\pgfsetdash{}{0pt}%
\pgfsys@defobject{currentmarker}{\pgfqpoint{0.000000in}{-0.138889in}}{\pgfqpoint{0.000000in}{0.000000in}}{%
\pgfpathmoveto{\pgfqpoint{0.000000in}{0.000000in}}%
\pgfpathlineto{\pgfqpoint{0.000000in}{-0.138889in}}%
\pgfusepath{stroke,fill}%
}%
\begin{pgfscope}%
\pgfsys@transformshift{6.291919in}{5.280000in}%
\pgfsys@useobject{currentmarker}{}%
\end{pgfscope}%
\end{pgfscope}%
\begin{pgfscope}%
\definecolor{textcolor}{rgb}{0.000000,0.000000,0.000000}%
\pgfsetstrokecolor{textcolor}%
\pgfsetfillcolor{textcolor}%
\pgftext[x=6.291919in,y=0.611389in,,top]{\color{textcolor}{\rmfamily\fontsize{16.000000}{19.200000}\selectfont\catcode`\^=\active\def^{\ifmmode\sp\else\^{}\fi}\catcode`\%=\active\def%{\%}8.64}}%
\end{pgfscope}%
\begin{pgfscope}%
\definecolor{textcolor}{rgb}{0.000000,0.000000,0.000000}%
\pgfsetstrokecolor{textcolor}%
\pgfsetfillcolor{textcolor}%
\pgftext[x=4.100000in,y=0.355096in,,top]{\color{textcolor}{\rmfamily\fontsize{16.000000}{19.200000}\selectfont\catcode`\^=\active\def^{\ifmmode\sp\else\^{}\fi}\catcode`\%=\active\def%{\%}$\log \epsilon_{\mathrm{O}}$}}%
\end{pgfscope}%
\begin{pgfscope}%
\pgfsetbuttcap%
\pgfsetroundjoin%
\definecolor{currentfill}{rgb}{0.000000,0.000000,0.000000}%
\pgfsetfillcolor{currentfill}%
\pgfsetlinewidth{0.803000pt}%
\definecolor{currentstroke}{rgb}{0.000000,0.000000,0.000000}%
\pgfsetstrokecolor{currentstroke}%
\pgfsetdash{}{0pt}%
\pgfsys@defobject{currentmarker}{\pgfqpoint{0.000000in}{0.000000in}}{\pgfqpoint{0.138889in}{0.000000in}}{%
\pgfpathmoveto{\pgfqpoint{0.000000in}{0.000000in}}%
\pgfpathlineto{\pgfqpoint{0.138889in}{0.000000in}}%
\pgfusepath{stroke,fill}%
}%
\begin{pgfscope}%
\pgfsys@transformshift{1.000000in}{1.442351in}%
\pgfsys@useobject{currentmarker}{}%
\end{pgfscope}%
\end{pgfscope}%
\begin{pgfscope}%
\pgfsetbuttcap%
\pgfsetroundjoin%
\definecolor{currentfill}{rgb}{0.000000,0.000000,0.000000}%
\pgfsetfillcolor{currentfill}%
\pgfsetlinewidth{0.803000pt}%
\definecolor{currentstroke}{rgb}{0.000000,0.000000,0.000000}%
\pgfsetstrokecolor{currentstroke}%
\pgfsetdash{}{0pt}%
\pgfsys@defobject{currentmarker}{\pgfqpoint{-0.138889in}{0.000000in}}{\pgfqpoint{-0.000000in}{0.000000in}}{%
\pgfpathmoveto{\pgfqpoint{-0.000000in}{0.000000in}}%
\pgfpathlineto{\pgfqpoint{-0.138889in}{0.000000in}}%
\pgfusepath{stroke,fill}%
}%
\begin{pgfscope}%
\pgfsys@transformshift{7.200000in}{1.442351in}%
\pgfsys@useobject{currentmarker}{}%
\end{pgfscope}%
\end{pgfscope}%
\begin{pgfscope}%
\definecolor{textcolor}{rgb}{0.000000,0.000000,0.000000}%
\pgfsetstrokecolor{textcolor}%
\pgfsetfillcolor{textcolor}%
\pgftext[x=0.562500in, y=1.366179in, left, base]{\color{textcolor}{\rmfamily\fontsize{16.000000}{19.200000}\selectfont\catcode`\^=\active\def^{\ifmmode\sp\else\^{}\fi}\catcode`\%=\active\def%{\%}0.02}}%
\end{pgfscope}%
\begin{pgfscope}%
\pgfsetbuttcap%
\pgfsetroundjoin%
\definecolor{currentfill}{rgb}{0.000000,0.000000,0.000000}%
\pgfsetfillcolor{currentfill}%
\pgfsetlinewidth{0.803000pt}%
\definecolor{currentstroke}{rgb}{0.000000,0.000000,0.000000}%
\pgfsetstrokecolor{currentstroke}%
\pgfsetdash{}{0pt}%
\pgfsys@defobject{currentmarker}{\pgfqpoint{0.000000in}{0.000000in}}{\pgfqpoint{0.138889in}{0.000000in}}{%
\pgfpathmoveto{\pgfqpoint{0.000000in}{0.000000in}}%
\pgfpathlineto{\pgfqpoint{0.138889in}{0.000000in}}%
\pgfusepath{stroke,fill}%
}%
\begin{pgfscope}%
\pgfsys@transformshift{1.000000in}{2.444869in}%
\pgfsys@useobject{currentmarker}{}%
\end{pgfscope}%
\end{pgfscope}%
\begin{pgfscope}%
\pgfsetbuttcap%
\pgfsetroundjoin%
\definecolor{currentfill}{rgb}{0.000000,0.000000,0.000000}%
\pgfsetfillcolor{currentfill}%
\pgfsetlinewidth{0.803000pt}%
\definecolor{currentstroke}{rgb}{0.000000,0.000000,0.000000}%
\pgfsetstrokecolor{currentstroke}%
\pgfsetdash{}{0pt}%
\pgfsys@defobject{currentmarker}{\pgfqpoint{-0.138889in}{0.000000in}}{\pgfqpoint{-0.000000in}{0.000000in}}{%
\pgfpathmoveto{\pgfqpoint{-0.000000in}{0.000000in}}%
\pgfpathlineto{\pgfqpoint{-0.138889in}{0.000000in}}%
\pgfusepath{stroke,fill}%
}%
\begin{pgfscope}%
\pgfsys@transformshift{7.200000in}{2.444869in}%
\pgfsys@useobject{currentmarker}{}%
\end{pgfscope}%
\end{pgfscope}%
\begin{pgfscope}%
\definecolor{textcolor}{rgb}{0.000000,0.000000,0.000000}%
\pgfsetstrokecolor{textcolor}%
\pgfsetfillcolor{textcolor}%
\pgftext[x=0.562500in, y=2.368697in, left, base]{\color{textcolor}{\rmfamily\fontsize{16.000000}{19.200000}\selectfont\catcode`\^=\active\def^{\ifmmode\sp\else\^{}\fi}\catcode`\%=\active\def%{\%}0.03}}%
\end{pgfscope}%
\begin{pgfscope}%
\pgfsetbuttcap%
\pgfsetroundjoin%
\definecolor{currentfill}{rgb}{0.000000,0.000000,0.000000}%
\pgfsetfillcolor{currentfill}%
\pgfsetlinewidth{0.803000pt}%
\definecolor{currentstroke}{rgb}{0.000000,0.000000,0.000000}%
\pgfsetstrokecolor{currentstroke}%
\pgfsetdash{}{0pt}%
\pgfsys@defobject{currentmarker}{\pgfqpoint{0.000000in}{0.000000in}}{\pgfqpoint{0.138889in}{0.000000in}}{%
\pgfpathmoveto{\pgfqpoint{0.000000in}{0.000000in}}%
\pgfpathlineto{\pgfqpoint{0.138889in}{0.000000in}}%
\pgfusepath{stroke,fill}%
}%
\begin{pgfscope}%
\pgfsys@transformshift{1.000000in}{3.447386in}%
\pgfsys@useobject{currentmarker}{}%
\end{pgfscope}%
\end{pgfscope}%
\begin{pgfscope}%
\pgfsetbuttcap%
\pgfsetroundjoin%
\definecolor{currentfill}{rgb}{0.000000,0.000000,0.000000}%
\pgfsetfillcolor{currentfill}%
\pgfsetlinewidth{0.803000pt}%
\definecolor{currentstroke}{rgb}{0.000000,0.000000,0.000000}%
\pgfsetstrokecolor{currentstroke}%
\pgfsetdash{}{0pt}%
\pgfsys@defobject{currentmarker}{\pgfqpoint{-0.138889in}{0.000000in}}{\pgfqpoint{-0.000000in}{0.000000in}}{%
\pgfpathmoveto{\pgfqpoint{-0.000000in}{0.000000in}}%
\pgfpathlineto{\pgfqpoint{-0.138889in}{0.000000in}}%
\pgfusepath{stroke,fill}%
}%
\begin{pgfscope}%
\pgfsys@transformshift{7.200000in}{3.447386in}%
\pgfsys@useobject{currentmarker}{}%
\end{pgfscope}%
\end{pgfscope}%
\begin{pgfscope}%
\definecolor{textcolor}{rgb}{0.000000,0.000000,0.000000}%
\pgfsetstrokecolor{textcolor}%
\pgfsetfillcolor{textcolor}%
\pgftext[x=0.562500in, y=3.371215in, left, base]{\color{textcolor}{\rmfamily\fontsize{16.000000}{19.200000}\selectfont\catcode`\^=\active\def^{\ifmmode\sp\else\^{}\fi}\catcode`\%=\active\def%{\%}0.04}}%
\end{pgfscope}%
\begin{pgfscope}%
\pgfsetbuttcap%
\pgfsetroundjoin%
\definecolor{currentfill}{rgb}{0.000000,0.000000,0.000000}%
\pgfsetfillcolor{currentfill}%
\pgfsetlinewidth{0.803000pt}%
\definecolor{currentstroke}{rgb}{0.000000,0.000000,0.000000}%
\pgfsetstrokecolor{currentstroke}%
\pgfsetdash{}{0pt}%
\pgfsys@defobject{currentmarker}{\pgfqpoint{0.000000in}{0.000000in}}{\pgfqpoint{0.138889in}{0.000000in}}{%
\pgfpathmoveto{\pgfqpoint{0.000000in}{0.000000in}}%
\pgfpathlineto{\pgfqpoint{0.138889in}{0.000000in}}%
\pgfusepath{stroke,fill}%
}%
\begin{pgfscope}%
\pgfsys@transformshift{1.000000in}{4.449904in}%
\pgfsys@useobject{currentmarker}{}%
\end{pgfscope}%
\end{pgfscope}%
\begin{pgfscope}%
\pgfsetbuttcap%
\pgfsetroundjoin%
\definecolor{currentfill}{rgb}{0.000000,0.000000,0.000000}%
\pgfsetfillcolor{currentfill}%
\pgfsetlinewidth{0.803000pt}%
\definecolor{currentstroke}{rgb}{0.000000,0.000000,0.000000}%
\pgfsetstrokecolor{currentstroke}%
\pgfsetdash{}{0pt}%
\pgfsys@defobject{currentmarker}{\pgfqpoint{-0.138889in}{0.000000in}}{\pgfqpoint{-0.000000in}{0.000000in}}{%
\pgfpathmoveto{\pgfqpoint{-0.000000in}{0.000000in}}%
\pgfpathlineto{\pgfqpoint{-0.138889in}{0.000000in}}%
\pgfusepath{stroke,fill}%
}%
\begin{pgfscope}%
\pgfsys@transformshift{7.200000in}{4.449904in}%
\pgfsys@useobject{currentmarker}{}%
\end{pgfscope}%
\end{pgfscope}%
\begin{pgfscope}%
\definecolor{textcolor}{rgb}{0.000000,0.000000,0.000000}%
\pgfsetstrokecolor{textcolor}%
\pgfsetfillcolor{textcolor}%
\pgftext[x=0.562500in, y=4.373732in, left, base]{\color{textcolor}{\rmfamily\fontsize{16.000000}{19.200000}\selectfont\catcode`\^=\active\def^{\ifmmode\sp\else\^{}\fi}\catcode`\%=\active\def%{\%}0.05}}%
\end{pgfscope}%
\begin{pgfscope}%
\definecolor{textcolor}{rgb}{0.000000,0.000000,0.000000}%
\pgfsetstrokecolor{textcolor}%
\pgfsetfillcolor{textcolor}%
\pgftext[x=0.506944in,y=2.970000in,,bottom,rotate=90.000000]{\color{textcolor}{\rmfamily\fontsize{16.000000}{19.200000}\selectfont\catcode`\^=\active\def^{\ifmmode\sp\else\^{}\fi}\catcode`\%=\active\def%{\%}$\chi^2$}}%
\end{pgfscope}%
\begin{pgfscope}%
\pgfpathrectangle{\pgfqpoint{1.000000in}{0.660000in}}{\pgfqpoint{6.200000in}{4.620000in}}%
\pgfusepath{clip}%
\pgfsetrectcap%
\pgfsetroundjoin%
\pgfsetlinewidth{1.505625pt}%
\definecolor{currentstroke}{rgb}{0.827451,0.827451,0.827451}%
\pgfsetstrokecolor{currentstroke}%
\pgfsetdash{}{0pt}%
\pgfpathmoveto{\pgfqpoint{1.281818in}{2.993539in}}%
\pgfpathlineto{\pgfqpoint{1.338751in}{2.891562in}}%
\pgfpathlineto{\pgfqpoint{1.395684in}{2.792094in}}%
\pgfpathlineto{\pgfqpoint{1.452617in}{2.695135in}}%
\pgfpathlineto{\pgfqpoint{1.509550in}{2.600685in}}%
\pgfpathlineto{\pgfqpoint{1.566483in}{2.508744in}}%
\pgfpathlineto{\pgfqpoint{1.623416in}{2.419313in}}%
\pgfpathlineto{\pgfqpoint{1.680349in}{2.332391in}}%
\pgfpathlineto{\pgfqpoint{1.737282in}{2.247978in}}%
\pgfpathlineto{\pgfqpoint{1.794215in}{2.166074in}}%
\pgfpathlineto{\pgfqpoint{1.851148in}{2.086680in}}%
\pgfpathlineto{\pgfqpoint{1.908081in}{2.009795in}}%
\pgfpathlineto{\pgfqpoint{1.965014in}{1.935419in}}%
\pgfpathlineto{\pgfqpoint{2.021947in}{1.863552in}}%
\pgfpathlineto{\pgfqpoint{2.078880in}{1.794194in}}%
\pgfpathlineto{\pgfqpoint{2.135813in}{1.727346in}}%
\pgfpathlineto{\pgfqpoint{2.192746in}{1.663007in}}%
\pgfpathlineto{\pgfqpoint{2.249679in}{1.601177in}}%
\pgfpathlineto{\pgfqpoint{2.306612in}{1.541856in}}%
\pgfpathlineto{\pgfqpoint{2.363545in}{1.485045in}}%
\pgfpathlineto{\pgfqpoint{2.420478in}{1.430743in}}%
\pgfpathlineto{\pgfqpoint{2.477410in}{1.378949in}}%
\pgfpathlineto{\pgfqpoint{2.534343in}{1.329666in}}%
\pgfpathlineto{\pgfqpoint{2.591276in}{1.282891in}}%
\pgfpathlineto{\pgfqpoint{2.648209in}{1.238626in}}%
\pgfpathlineto{\pgfqpoint{2.705142in}{1.196870in}}%
\pgfpathlineto{\pgfqpoint{2.762075in}{1.157623in}}%
\pgfpathlineto{\pgfqpoint{2.819008in}{1.120885in}}%
\pgfpathlineto{\pgfqpoint{2.875941in}{1.086656in}}%
\pgfpathlineto{\pgfqpoint{2.932874in}{1.054937in}}%
\pgfpathlineto{\pgfqpoint{2.989807in}{1.025727in}}%
\pgfpathlineto{\pgfqpoint{3.046740in}{0.999026in}}%
\pgfpathlineto{\pgfqpoint{3.103673in}{0.974835in}}%
\pgfpathlineto{\pgfqpoint{3.160606in}{0.953152in}}%
\pgfpathlineto{\pgfqpoint{3.217539in}{0.933979in}}%
\pgfpathlineto{\pgfqpoint{3.274472in}{0.917315in}}%
\pgfpathlineto{\pgfqpoint{3.331405in}{0.903160in}}%
\pgfpathlineto{\pgfqpoint{3.388338in}{0.891515in}}%
\pgfpathlineto{\pgfqpoint{3.445271in}{0.882379in}}%
\pgfpathlineto{\pgfqpoint{3.502204in}{0.875751in}}%
\pgfpathlineto{\pgfqpoint{3.559137in}{0.871634in}}%
\pgfpathlineto{\pgfqpoint{3.616070in}{0.870025in}}%
\pgfpathlineto{\pgfqpoint{3.673003in}{0.870926in}}%
\pgfpathlineto{\pgfqpoint{3.729936in}{0.874335in}}%
\pgfpathlineto{\pgfqpoint{3.786869in}{0.880254in}}%
\pgfpathlineto{\pgfqpoint{3.843802in}{0.888683in}}%
\pgfpathlineto{\pgfqpoint{3.900735in}{0.899620in}}%
\pgfpathlineto{\pgfqpoint{3.957668in}{0.913067in}}%
\pgfpathlineto{\pgfqpoint{4.014601in}{0.929023in}}%
\pgfpathlineto{\pgfqpoint{4.071534in}{0.947488in}}%
\pgfpathlineto{\pgfqpoint{4.128466in}{0.968462in}}%
\pgfpathlineto{\pgfqpoint{4.185399in}{0.991946in}}%
\pgfpathlineto{\pgfqpoint{4.242332in}{1.017939in}}%
\pgfpathlineto{\pgfqpoint{4.299265in}{1.046441in}}%
\pgfpathlineto{\pgfqpoint{4.356198in}{1.077452in}}%
\pgfpathlineto{\pgfqpoint{4.413131in}{1.110972in}}%
\pgfpathlineto{\pgfqpoint{4.470064in}{1.147002in}}%
\pgfpathlineto{\pgfqpoint{4.526997in}{1.185541in}}%
\pgfpathlineto{\pgfqpoint{4.583930in}{1.226589in}}%
\pgfpathlineto{\pgfqpoint{4.640863in}{1.270146in}}%
\pgfpathlineto{\pgfqpoint{4.697796in}{1.316213in}}%
\pgfpathlineto{\pgfqpoint{4.754729in}{1.364789in}}%
\pgfpathlineto{\pgfqpoint{4.811662in}{1.415874in}}%
\pgfpathlineto{\pgfqpoint{4.868595in}{1.469468in}}%
\pgfpathlineto{\pgfqpoint{4.925528in}{1.525571in}}%
\pgfpathlineto{\pgfqpoint{4.982461in}{1.584184in}}%
\pgfpathlineto{\pgfqpoint{5.039394in}{1.645306in}}%
\pgfpathlineto{\pgfqpoint{5.096327in}{1.708937in}}%
\pgfpathlineto{\pgfqpoint{5.153260in}{1.775077in}}%
\pgfpathlineto{\pgfqpoint{5.210193in}{1.843727in}}%
\pgfpathlineto{\pgfqpoint{5.267126in}{1.914885in}}%
\pgfpathlineto{\pgfqpoint{5.324059in}{1.988553in}}%
\pgfpathlineto{\pgfqpoint{5.380992in}{2.064731in}}%
\pgfpathlineto{\pgfqpoint{5.437925in}{2.143417in}}%
\pgfpathlineto{\pgfqpoint{5.494858in}{2.224613in}}%
\pgfpathlineto{\pgfqpoint{5.551791in}{2.308317in}}%
\pgfpathlineto{\pgfqpoint{5.608724in}{2.394532in}}%
\pgfpathlineto{\pgfqpoint{5.665657in}{2.483255in}}%
\pgfpathlineto{\pgfqpoint{5.722590in}{2.574487in}}%
\pgfpathlineto{\pgfqpoint{5.779522in}{2.668229in}}%
\pgfpathlineto{\pgfqpoint{5.836455in}{2.764480in}}%
\pgfpathlineto{\pgfqpoint{5.893388in}{2.863240in}}%
\pgfpathlineto{\pgfqpoint{5.950321in}{2.964510in}}%
\pgfpathlineto{\pgfqpoint{6.007254in}{3.068288in}}%
\pgfpathlineto{\pgfqpoint{6.064187in}{3.174576in}}%
\pgfpathlineto{\pgfqpoint{6.121120in}{3.283373in}}%
\pgfpathlineto{\pgfqpoint{6.178053in}{3.394679in}}%
\pgfpathlineto{\pgfqpoint{6.234986in}{3.508495in}}%
\pgfpathlineto{\pgfqpoint{6.291919in}{3.624820in}}%
\pgfpathlineto{\pgfqpoint{6.348852in}{3.743654in}}%
\pgfpathlineto{\pgfqpoint{6.405785in}{3.864997in}}%
\pgfpathlineto{\pgfqpoint{6.462718in}{3.988849in}}%
\pgfpathlineto{\pgfqpoint{6.519651in}{4.115211in}}%
\pgfpathlineto{\pgfqpoint{6.576584in}{4.244081in}}%
\pgfpathlineto{\pgfqpoint{6.633517in}{4.375462in}}%
\pgfpathlineto{\pgfqpoint{6.690450in}{4.509351in}}%
\pgfpathlineto{\pgfqpoint{6.747383in}{4.645749in}}%
\pgfpathlineto{\pgfqpoint{6.804316in}{4.784657in}}%
\pgfpathlineto{\pgfqpoint{6.861249in}{4.926074in}}%
\pgfpathlineto{\pgfqpoint{6.918182in}{5.070000in}}%
\pgfusepath{stroke}%
\end{pgfscope}%
\begin{pgfscope}%
\pgfsetrectcap%
\pgfsetmiterjoin%
\pgfsetlinewidth{0.803000pt}%
\definecolor{currentstroke}{rgb}{0.000000,0.000000,0.000000}%
\pgfsetstrokecolor{currentstroke}%
\pgfsetdash{}{0pt}%
\pgfpathmoveto{\pgfqpoint{1.000000in}{0.660000in}}%
\pgfpathlineto{\pgfqpoint{1.000000in}{5.280000in}}%
\pgfusepath{stroke}%
\end{pgfscope}%
\begin{pgfscope}%
\pgfsetrectcap%
\pgfsetmiterjoin%
\pgfsetlinewidth{0.803000pt}%
\definecolor{currentstroke}{rgb}{0.000000,0.000000,0.000000}%
\pgfsetstrokecolor{currentstroke}%
\pgfsetdash{}{0pt}%
\pgfpathmoveto{\pgfqpoint{7.200000in}{0.660000in}}%
\pgfpathlineto{\pgfqpoint{7.200000in}{5.280000in}}%
\pgfusepath{stroke}%
\end{pgfscope}%
\begin{pgfscope}%
\pgfsetrectcap%
\pgfsetmiterjoin%
\pgfsetlinewidth{0.803000pt}%
\definecolor{currentstroke}{rgb}{0.000000,0.000000,0.000000}%
\pgfsetstrokecolor{currentstroke}%
\pgfsetdash{}{0pt}%
\pgfpathmoveto{\pgfqpoint{1.000000in}{0.660000in}}%
\pgfpathlineto{\pgfqpoint{7.200000in}{0.660000in}}%
\pgfusepath{stroke}%
\end{pgfscope}%
\begin{pgfscope}%
\pgfsetrectcap%
\pgfsetmiterjoin%
\pgfsetlinewidth{0.803000pt}%
\definecolor{currentstroke}{rgb}{0.000000,0.000000,0.000000}%
\pgfsetstrokecolor{currentstroke}%
\pgfsetdash{}{0pt}%
\pgfpathmoveto{\pgfqpoint{1.000000in}{5.280000in}}%
\pgfpathlineto{\pgfqpoint{7.200000in}{5.280000in}}%
\pgfusepath{stroke}%
\end{pgfscope}%
\begin{pgfscope}%
\pgfsetbuttcap%
\pgfsetmiterjoin%
\definecolor{currentfill}{rgb}{1.000000,1.000000,1.000000}%
\pgfsetfillcolor{currentfill}%
\pgfsetfillopacity{0.800000}%
\pgfsetlinewidth{1.003750pt}%
\definecolor{currentstroke}{rgb}{0.800000,0.800000,0.800000}%
\pgfsetstrokecolor{currentstroke}%
\pgfsetstrokeopacity{0.800000}%
\pgfsetdash{}{0pt}%
\pgfpathmoveto{\pgfqpoint{1.155556in}{4.156563in}}%
\pgfpathlineto{\pgfqpoint{4.397049in}{4.156563in}}%
\pgfpathquadraticcurveto{\pgfqpoint{4.441493in}{4.156563in}}{\pgfqpoint{4.441493in}{4.201008in}}%
\pgfpathlineto{\pgfqpoint{4.441493in}{5.124444in}}%
\pgfpathquadraticcurveto{\pgfqpoint{4.441493in}{5.168889in}}{\pgfqpoint{4.397049in}{5.168889in}}%
\pgfpathlineto{\pgfqpoint{1.155556in}{5.168889in}}%
\pgfpathquadraticcurveto{\pgfqpoint{1.111111in}{5.168889in}}{\pgfqpoint{1.111111in}{5.124444in}}%
\pgfpathlineto{\pgfqpoint{1.111111in}{4.201008in}}%
\pgfpathquadraticcurveto{\pgfqpoint{1.111111in}{4.156563in}}{\pgfqpoint{1.155556in}{4.156563in}}%
\pgfpathlineto{\pgfqpoint{1.155556in}{4.156563in}}%
\pgfpathclose%
\pgfusepath{stroke,fill}%
\end{pgfscope}%
\begin{pgfscope}%
\pgfsetbuttcap%
\pgfsetroundjoin%
\definecolor{currentfill}{rgb}{0.000000,0.000000,0.545098}%
\pgfsetfillcolor{currentfill}%
\pgfsetlinewidth{1.505625pt}%
\definecolor{currentstroke}{rgb}{0.000000,0.000000,0.545098}%
\pgfsetstrokecolor{currentstroke}%
\pgfsetdash{}{0pt}%
\pgfsys@defobject{currentmarker}{\pgfqpoint{-0.041667in}{-0.041667in}}{\pgfqpoint{0.041667in}{0.041667in}}{%
\pgfpathmoveto{\pgfqpoint{-0.041667in}{-0.041667in}}%
\pgfpathlineto{\pgfqpoint{0.041667in}{0.041667in}}%
\pgfpathmoveto{\pgfqpoint{-0.041667in}{0.041667in}}%
\pgfpathlineto{\pgfqpoint{0.041667in}{-0.041667in}}%
\pgfusepath{stroke,fill}%
}%
\begin{pgfscope}%
\pgfsys@transformshift{1.422222in}{4.982517in}%
\pgfsys@useobject{currentmarker}{}%
\end{pgfscope}%
\end{pgfscope}%
\begin{pgfscope}%
\definecolor{textcolor}{rgb}{0.000000,0.000000,0.000000}%
\pgfsetstrokecolor{textcolor}%
\pgfsetfillcolor{textcolor}%
\pgftext[x=1.822222in,y=4.924184in,left,base]{\color{textcolor}{\rmfamily\fontsize{16.000000}{19.200000}\selectfont\catcode`\^=\active\def^{\ifmmode\sp\else\^{}\fi}\catcode`\%=\active\def%{\%}Raie de OH en 15391.057 Å}}%
\end{pgfscope}%
\begin{pgfscope}%
\pgfsetrectcap%
\pgfsetroundjoin%
\pgfsetlinewidth{1.505625pt}%
\definecolor{currentstroke}{rgb}{0.827451,0.827451,0.827451}%
\pgfsetstrokecolor{currentstroke}%
\pgfsetdash{}{0pt}%
\pgfpathmoveto{\pgfqpoint{1.200000in}{4.686901in}}%
\pgfpathlineto{\pgfqpoint{1.422222in}{4.686901in}}%
\pgfpathlineto{\pgfqpoint{1.644444in}{4.686901in}}%
\pgfusepath{stroke}%
\end{pgfscope}%
\begin{pgfscope}%
\definecolor{textcolor}{rgb}{0.000000,0.000000,0.000000}%
\pgfsetstrokecolor{textcolor}%
\pgfsetfillcolor{textcolor}%
\pgftext[x=1.822222in,y=4.609123in,left,base]{\color{textcolor}{\rmfamily\fontsize{16.000000}{19.200000}\selectfont\catcode`\^=\active\def^{\ifmmode\sp\else\^{}\fi}\catcode`\%=\active\def%{\%}Ajustement quadratique}}%
\end{pgfscope}%
\begin{pgfscope}%
\pgfsetbuttcap%
\pgfsetroundjoin%
\definecolor{currentfill}{rgb}{1.000000,0.000000,0.000000}%
\pgfsetfillcolor{currentfill}%
\pgfsetlinewidth{1.505625pt}%
\definecolor{currentstroke}{rgb}{1.000000,0.000000,0.000000}%
\pgfsetstrokecolor{currentstroke}%
\pgfsetdash{}{0pt}%
\pgfsys@defobject{currentmarker}{\pgfqpoint{-0.041667in}{-0.041667in}}{\pgfqpoint{0.041667in}{0.041667in}}{%
\pgfpathmoveto{\pgfqpoint{-0.041667in}{-0.041667in}}%
\pgfpathlineto{\pgfqpoint{0.041667in}{0.041667in}}%
\pgfpathmoveto{\pgfqpoint{-0.041667in}{0.041667in}}%
\pgfpathlineto{\pgfqpoint{0.041667in}{-0.041667in}}%
\pgfusepath{stroke,fill}%
}%
\begin{pgfscope}%
\pgfsys@transformshift{1.422222in}{4.352179in}%
\pgfsys@useobject{currentmarker}{}%
\end{pgfscope}%
\end{pgfscope}%
\begin{pgfscope}%
\definecolor{textcolor}{rgb}{0.000000,0.000000,0.000000}%
\pgfsetstrokecolor{textcolor}%
\pgfsetfillcolor{textcolor}%
\pgftext[x=1.822222in,y=4.293846in,left,base]{\color{textcolor}{\rmfamily\fontsize{16.000000}{19.200000}\selectfont\catcode`\^=\active\def^{\ifmmode\sp\else\^{}\fi}\catcode`\%=\active\def%{\%}Minimum en 8.60}}%
\end{pgfscope}%
\end{pgfpicture}%
\makeatother%
\endgroup%
} % Insérer le fichier PGF du quatrième graphique
%   \end{subfigure}
  
% \end{figure}

\section{Métallicité}

log $\varepsilon_{\rm Fe}$ = 7.21
solaire 7.45 
donc [Fe/H] = -0.24 à l'ETL 


\section{Teff}
pente de quasi 0 donc 4000K bonne Teff à l'ETL 

\section{logg}
on test plusieurs log g avec différents modèles, +1, +2, +0, +3, +4 -> best fit +1 à l'ETL 
sur des raies profondes principalement de Mg et Ca (à tester encore)
checker avec les isochrones et les tracés évolutifs 
l'un suppose l'âge de l'étoile l'autre sa masse

\section{vitesse de micro}
regarder avec eqwuidt la largeur équivalente sinon regarder avec les synthèses et plot python

\section{NLTE}
faire tourner le code poiur les atomes
pas possible de faire de hors etl pour les molécules, même pas pour tous les atomes
\section{abondance d'éléments lourds}
regarder ce que je peux identifier

\clearpage

\bibliographystyle{abbrvnat}
\bibliography{bibtex.bib}


\clearpage
\appendix
\section{Table}
\begin{table}[h!]
  %\caption{Listes de molécules et contribution de chacune d'elle dans la bande H et K}
  \vspace{0.3cm}
  \begin{minipage}[t]{.4\linewidth}
  \begin{center}
    \begin{tabular}{ccc}
          \hline
      \hline
      Fe I & $\lambda_{\mathrm{min}}$ & $\lambda_{\mathrm{max}}$ \\
      \hline
          14745.39 & 14745.01 & 14745.93 \\
          14814.73 & 14814.25 & 14815.41 \\
          14826.41 & 14825.87 & 14826.95 \\
          14897.41 & 14896.89 & 14897.82 \\
          14988.78 & 14988.28 & 14989.29 \\
          15017.70 & 15017.25 & 15018.10 \\
          15194.49 & 15194.00 & 15194.93 \\
          15343.79 & 15343.33 & 15344.26 \\
          15394.67 & 15394.21 & 15395.14 \\
          15395.72 & 15395.22 & 15396.31 \\
          15591.49 & 15590.94 & 15591.95 \\
          15648.51 & 15648.03 & 15648.96 \\
          15723.59 & 15723.00 & 15724.01 \\
          15741.92 & 15741.43 & 15742.36 \\
          15818.14 & 15817.57 & 15818.58 \\
          15821.71 & 15821.21 & 15822.14 \\
          15822.82 & 15822.30 & 15823.30 \\
          15911.30 & 15910.82 & 15911.83 \\
          15964.86 & 15964.27 & 15965.43 \\
          16040.65 & 16040.17 & 16041.10 \\
          16042.72 & 16042.18 & 16043.19 \\
          16125.90 & 16125.45 & 16126.45 \\
          16165.03 & 16164.40 & 16165.57 \\
          16180.90 & 16180.44 & 16181.29 \\
          \hline
      \end{tabular}
  \end{center} 
  \end{minipage} 
  \hspace{2.cm}
  \begin{minipage}[t]{.4\linewidth}
  \begin{center}
     \begin{tabular}{ccc}
          \hline
      \hline
      Fe I & $\lambda_{\mathrm{min}}$ & $\lambda_{\mathrm{max}}$ \\
      \hline
      16198.50 & 16197.86 & 16198.95 \\
      16284.77 & 16284.22 & 16285.39 \\
      16316.32 & 16315.75 & 16316.91 \\
      16318.69 & 16318.07 & 16319.23 \\
      16324.45 & 16323.96 & 16325.04 \\
      16436.62 & 16436.19 & 16437.04 \\
      16466.92 & 16466.24 & 16467.55 \\
      16486.67 & 16486.07 & 16487.23 \\
      16506.29 & 16505.74 & 16506.75 \\
      16517.22 & 16516.66 & 16517.67 \\
      16645.87 & 16645.31 & 16646.39 \\
      17204.30 & 17203.74 & 17204.83 \\
      17706.62 & 17705.95 & 17707.11 \\
      17721.09 & 17720.51 & 17721.75 \\
      17721.37 & 17720.51 & 17721.75 \\
      17771.12 & 17770.46 & 17771.70 \\
      17932.60 & 17931.95 & 17933.19 \\
      19923.34 & 19922.62 & 19923.86 \\
      21238.47 & 21237.74 & 21239.13 \\
      22257.11 & 22256.07 & 22257.87 \\
      22260.18 & 22259.25 & 22260.91 \\
      22392.88 & 22392.15 & 22393.39 \\
      22419.98 & 22419.14 & 22420.67 \\
      22832.36 & 22831.40 & 22833.06 \\
          \hline
      \end{tabular}
    \end{center}
  \end{minipage}
  \vspace{0.3cm}
  
  \textbf{Notes.}
  \end{table}

  \begin{table}[h!]
    %\caption{Listes de molécules et contribution de chacune d'elle dans la bande H et K}
    
    \begin{minipage}[t]{.4\linewidth}
    \begin{center}
      \vfill
        \begin{tabular}{ccc}
        \hline
        \hline
        $^{16}$OH & $\lambda_{\mathrm{min}}$ & $\lambda_{\mathrm{max}}$ \\
        \hline
        14613.49 & 14612.95 & 14613.96 \\
        14661.08 & 14660.58 & 14661.51 \\
        15002.17 & 15001.76 & 15002.54 \\
        15003.15 & 15002.69 & 15003.54 \\
        15130.89 & 15130.41 & 15131.42 \\
        15266.20 & 15265.56 & 15266.65 \\
        15278.53 & 15278.11 & 15279.12 \\
        15391.13 & 15390.57 & 15391.74 \\
        15409.18 & 15408.54 & 15409.70 \\
        15428.45 & 15427.98 & 15428.91 \\
        15429.64 & 15428.99 & 15430.07 \\
        15505.69 & 15505.13 & 15506.52 \\
        15568.78 & 15568.25 & 15569.26 \\
        15651.91 & 15651.36 & 15652.29 \\
        15719.68 & 15719.13 & 15720.14 \\
        15755.53 & 15754.91 & 15755.92 \\
        15756.52 & 15756.15 & 15756.85 \\
        16052.77 & 16052.25 & 16053.03 \\
        16247.89 & 16247.51 & 16248.21 \\
        16312.72 & 16311.87 & 16313.35 \\
        16347.49 & 16347.27 & 16347.89 \\
        16368.13 & 16367.49 & 16368.65 \\
        16448.05 & 16447.49 & 16448.50 \\
        16662.19 & 16661.81 & 16662.73 \\
        16729.78 & 16729.50 & 16730.04 \\
        16904.29 & 16903.69 & 16904.62 \\
        17096.38 & 17095.77 & 17096.70 \\
        17104.72 & 17104.29 & 17105.22 \\
        17239.75 & 17239.29 & 17240.15 \\
        17322.25 & 17321.94 & 17322.71 \\
        17423.86 & 17423.32 & 17424.33 \\
        17618.89 & 17618.35 & 17619.43 \\
        17772.70 & 17772.01 & 17773.10 \\
        17818.48 & 17818.10 & 17818.95 \\
        17825.20 & 17824.60 & 17825.61 \\
        \hline
        \end{tabular} \par
        \vspace{1cm}
        \begin{tabular}{ccc}
          \hline
          \hline
          $^{12}$C$^{16}$O & $\lambda_{\mathrm{min}}$ & $\lambda_{\mathrm{max}}$ \\
          \hline
          15780.09 & 15779.54 & 15780.63 \\
          16237.90 & 16237.36 & 16238.29 \\
          16314.39 & 16313.89 & 16314.89 \\
          17026.00 & 17025.60 & 17026.45 \\
          17081.80 & 17081.37 & 17082.30 \\
          17129.35 & 17128.77 & 17129.85 \\
          23073.90 & 23073.11 & 23074.49 \\
          23109.40 & 23108.55 & 23110.21 \\
          \hline
          \end{tabular}
    \end{center} 
    \end{minipage} 
    \hspace{2.0cm}
    \begin{minipage}[t]{.4\linewidth}
    \begin{center}
      \vfill
      \begin{tabular}{ccc}
        \hline
        \hline
        $^{12}$C$^{14}$N & $\lambda_{\mathrm{min}}$ & $\lambda_{\mathrm{max}}$ \\
        \hline
        14744.25 & 14743.77 & 14744.54 \\
        14747.20 & 14746.71 & 14747.72 \\
        14757.56 & 14757.17 & 14757.94 \\
        14759.84 & 14759.41 & 14760.42 \\
        14763.50 & 14762.97 & 14764.06 \\
        14833.25 & 14832.84 & 14833.69 \\
        15192.10 & 15191.68 & 15192.45 \\
        15447.11 & 15446.57 & 15447.58 \\
        16056.95 & 16056.43 & 16057.44 \\
        16167.23 & 16166.88 & 16167.50 \\
        16317.60 & 16317.14 & 16317.92 \\
        16358.15 & 16357.65 & 16358.58 \\
        16590.70 & 16590.39 & 16591.01 \\
        19833.59 & 19833.05 & 19834.02 \\
        19865.84 & 19865.17 & 19866.41 \\
        19906.49 & 19906.00 & 19906.97 \\
        19913.57 & 19912.92 & 19914.17 \\
        22220.20 & 22219.66 & 22220.63 \\
        22400.50 & 22399.90 & 22401.01 \\
        22593.83 & 22593.16 & 22594.40 \\
        22753.90 & 22753.19 & 22754.57 \\
        \hline
        \end{tabular}
      \end{center}
    \end{minipage}
    \vspace{0.3cm}
    
    \textbf{Notes.}
    \end{table}



\end{document}  